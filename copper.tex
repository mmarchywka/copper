
%%%%%%%%%%%%%%%%%%%%%%%%%%%%%%%%%%%%%%%%%%%%%%%%%%%%%%%%%%%%%%%%%%
% Sample template for MIT Junior Lab Student Written Summaries
% Available from http://web.mit.edu/8.13/www/Samplepaper/sample-paper.tex
% Last Updated April 12, 2007
% Adapted from the American Physical Societies REVTeK-4 Pages
% at http://publish.aps.org

\setlength{\paperheight}{11in}
% http://tex.stackexchange.com/questions/74636/mla-package-and-thumbpdf
\makeatletter
\@namedef{ver@thumbpdf.sty}{}
\makeatother
\documentclass[aps,secnumarabic,balancelastpage,amsmath,amssymb,nofootinbib]{revtex4}

%http://tex.stackexchange.com/questions/119905/insert-multiple-figures-in-latex

\usepackage[nomessages]{fp}    %mjm   needed for chemfig and mol2chemfig computed angles  
\usepackage{siunitx}        %mjm  appendix table subsections  
%\usepackage{morefloats}        %mjm   saving up figs for the end   
% fking incompatibvle fuxk ing floatrow 
%\usepackage{float}        %mjm  appendix table subsections  
\usepackage{pbox}        %mjm box off junk  
\usepackage{comment}        %mjm  see the build options  
\usepackage{framed}        %mjm box off junk  
\usepackage{lgrind}        % convert program listings to a form includable in a LaTeX document
% comment out for biblatex test
\usepackage{chapterbib}    % allows a bibliography for each chapter (each labguide has it's own)
%\usepackage{biblatex}   
\usepackage{color}         % produces boxes or entire pages with colored backgrounds
\usepackage{graphics}      % standard graphics specifications
\usepackage[pdftex]{graphicx}      % alternative graphics specifications
%\usepackage{graphicx}      % alternative graphics specifications
\usepackage{longtable}     % helps with long table options
\usepackage{epsf}          % old package handles encapsulated post script issues
\usepackage{bm}            % special 'bold-math' package
\usepackage{url}            % path for stupid jobname   
%\usepackage{asymptote}     % For typesetting of mathematical illustrations
\usepackage{thumbpdf}
\usepackage[colorlinks=true]{hyperref}  % this package should be added after all others
%\usepackage[draft=false, x-bib-pages=\input{\mjmbasename.last_page}, colorlinks=true]{hyperref}  % this package should be added after all others
%\usepackage[draft=false, x-bib-pages=\input{allbib.last_page}, colorlinks=true]{hyperref}  % this package should be added after all others
                                        % use as follows: \url{http://web.mit.edu/8.13}

%http://tex.stackexchange.com/questions/12676/add-notes-under-the-table
%\usepackage{booktabs,caption,fixltx2e}
% not work with subfig????
\usepackage[CaptionAfterwards]{fltpage}
\usepackage{lipsum}

% this does not  work .... 
%\usepackage[utf8]{inputenc}
%\usepackage{tabulary}
\usepackage[para,online,flushleft]{threeparttable}
%\usepackage{threeparttable}

%\usepackage{chemfig}
\usepackage[version=3]{mhchem}        %  
\usepackage{mol2chemfig}
% chemfig vriables maybe
\usepackage{xstring}        %mjm  appendix table subsections  
%\usepackage{chemformula}
% \usepackage{chemmacros}
\usepackage{floatrow}
\usepackage{fancyhdr}
%  underscore in jobmane f 
% https://latex.org/forum/viewtopic.php?t=2975
% this also messes up pdftotext  
%\usepackage[T1]{fontenc}
\usepackage{dcolumn} % https://tex.stackexchange.com/questions/2746/aligning-numbers-by-decimal-points-in-table-columns

\usepackage{catchfile} % mjmaddbib needs to read page count file
\pagestyle{fancy}




\newcolumntype{.}[1]{D{.}{.}{#1}}
%\usepackage[maxfloats=30]{morefloats}   %mjm   saving up figs for the end   
% no param on old version stuck at 36
\usepackage{morefloats}        %mjm   saving up figs for the end   
\usepackage{graphicx}        %mjm   saving up figs for the end   

% will need modificaitons 

% 2020-10-18 extract some new boilerplate 

% https://tex.stackexchange.com/questions/121601/automatically-wrap-the-text-in-verbatim
\usepackage{listings}
\lstset{
basicstyle=\small\ttfamily,
columns=flexible,
breaklines=true
}
%%%%%%%%%%%%%%%%%%%%%%%%%%%%% utilitites

\newcommand{\mjmblackbox}[2]{
 \fbox{
% thi does not ing work right 
\begin{minipage}[t]{\textwidth}
{ \centering{\bf{#1 : }} }
\par
#2
\end{minipage}
}
}
\newcommand{\mjmblackboxno}[2]{
 \fbox{
% thi does not ing work right 
\begin{minipage}[t]{\textwidth}
{ \centering{\bf{#1 : }} }
#2
\end{minipage}
}
}





%%%%%%%%%%%%%%%%%%%%%%%% biblio stuff



\newcommand{\checkrel}[1]{%
  \ifcsname#1\endcsname%
\newcommand{\mjmstatus}{  public NOTES }
\newcommand{\mjmversion}{\mjmrelease}
  \else%
\newcommand{\mjmstatus}{ NOT public NOTES }
\newcommand{\mjmversion}{0.00}
  \fi%
}




% https://tex.stackexchange.com/questions/18089/are-there-any-command-for-producing-the-bibtex-logo
%\def\BibTeX{{\rm B\kern-.05em{\sc i\kern-.025em b}\kern-.08em
%    T\kern-.1667em\lower.7ex\hbox{E}\kern-.125emX}}

\newcommand{\biblogo}{
{{\rm B\kern-.05em{\sc i\kern-.025em b}\kern-.08em
    T\kern-.1667em\lower.7ex\hbox{E}\kern-.125emX}}
{ }  }
%\newcommand{\biblogo}{ Bibte{\it X}  { }  }
\newcommand{\latexlogo}{ \LaTeX  { }  }
\newcommand{\bomtexlogo}{ BomTe{\it X}  { }  }






\newcommand{\mjmvirus}{SARS-Cov-2 }
\newcommand{\mjmdisease}{covid-19 }
\newcommand{\Mjmdisease}{Covid-19 }
\newcommand{\mjmlogo}{ MUQED { }  }
\newcommand{\mjmlinkedin}{ {\bf LinkedIn} { }  }








% https://tex.stackexchange.com/questions/121601/automatically-wrap-the-text-in-verbatim
\usepackage{listings}
\lstset{
basicstyle=\small\ttfamily,
columns=flexible,
breaklines=true
}

% none of this fking fking works for a f F 
\newcommand{\mjmverbatim}{lstlisting}
\newcommand{\mjmbeginverbatim}{\begin{lstlisting}}
\newcommand{\mjmendverbatim}{\end{lstlisting}}

\newcommand{\mjmmangle}[1]{keep/#1}



%# CHANGE VERSION AND STATUS MANUALLY 
% need a draft/notes/release flag

% https://tex.stackexchange.com/questions/5894/latex-conditional-expression
%At the command-line, you can do \def\MYFLAG{} and then test if \MYFLAG is defined in your document (or an included style file) with \ifdefined\MYFLAG ... \else ... \fi.
% needs trailing space for the sample bibtex doh
% leading spaces mess up the entry thought 
\def\xxmjmrelease{0.10 }
\ifdefined\mjmrelease
\newcommand{\mjmstatus}{ PUBLIC NOTES }
\newcommand{\mjmversion}{\mjmrelease} %%%%%%%%%%%%%
\newcommand{\mjmtrno}{MJM-2024-010}
\newcommand{\mjmbib}{\mjmtrno-\mjmversion}
\newcommand{\mjmstatuswarn}{{\bf{  }}   }
% 2021-09-29 wanted version wth for brownie
%\newcommand{\mjmbib}{\mjmtrno-\mjmversion-\mjmrelease}
%\newcommand{\mjmbib}{\mjmtrno}
\else
\newcommand{\mjmstatus}{ NOT public NOTES }
\newcommand{\mjmversion}{0.00} %%%%%%%%%%%%%
\newcommand{\mjmtrno}{MJM-2024-010}
%\newcommand{\mjmbib}{\mjmtrno-\mjmversion}
\newcommand{\mjmbib}{\mjmtrno}
%\newcommand{\mjmstatuswarn}{  }
\newcommand{\mjmstatuswarn}{{\bf{This document is a non-public DRAFT and contents may be speculative or undocumented or simple musings and should be read as such.  }}   }
\fi

%\newcommand{\mjmstatus}{ NOT public NOTES }



\newcommand{\mjmtitle}{Utility of Copper Supplementation in Dogs: Listen to Her Heart }



\ifdefined\mjmlegacy
\newcommand{\expandableinput}[1]{\input{#1}}
\else
% https://tex.stackexchange.com/questions/583927/misplaced-noalign-error-with-input-in-a-table-after-the-2020-fall-latex-releas/583939#583939
\ExplSyntaxOn % providing \expandableinput
\cs_new:Npn \expandableinput #1
  { \use:c { @@input } { \file_full_name:n {#1} } }
\ExplSyntaxOff
\fi



\ifdefined\grad
\else
\newcommand{\grad}{\nabla}
\fi
\newcommand{\mjmquote}[1]{{\centering {\it #1 }}}
\newcommand{\mjmwisdom}[1]{{\centering {\it #1 }}}
\newcommand{\laplace}{\nabla^{2}}
\newcommand{\mjmdx}[2]{\left(\frac{\partial #1 }{\partial #2} \right) }
\newcommand{\mjmdxop}[2]{\frac{\partial  }{\partial #2}\left( #1 \right) }
\newcommand{\mjmdxdx}[2]{\left(\frac{\partial^{2} #1 }{\partial #2^{2}} \right) }
\newcommand{\mjmdxo}[2]{\frac{\partial #1 }{\partial #2} }
\newcommand{\mjmdxx}[2]{\left(\frac{\partial^{2} #1 }{\partial {#2}^{2}} \right) }
\newcommand{\mjmdxy}[3]{\left(\frac{\partial^{2} #1 }{\partial {#2}\partial{#3}} \right) }
\newcommand{\mjmdxyn}[5]{\left(\frac{\partial^{#1} }{\partial^{#2} {#3}\partial^{#4}{#5}} \right) }
\newcommand{\mjmdsq}[2]{\left(\frac{\partial #1 }{\partial #2} \right) ^{2}}
\newcommand{\mjmupdated}[2]{\p  Updated on $#1$ from source #2 \p }
% see tug mail archives, based on discussion 
%\bool_new:N \l_tmpa_bool
\newif\ifbibstarted
\newif\ifbibnamed
\bibstartedfalse
\bibnamedfalse
\newcommand{\mjmtotalbib}{}
%\def\foo{}
\newcommand{\mjmsummabib}[2]{
%\renewcommand{\mjmtotalbib}{ \mjmtotalbib, #1 = #2 }
%\let\foo{ #1 = #2}
}
%%%%%%%%%%%%%%%%%%%%%%%%%%%%%%%%%%%%%%%%%%%%%%%%%%%%%%%%%%%%%%%%

\iffalse
@software{,
  author = {Michael J Marchywka},
  city = {Jasper GA 30143 USA},
  title = { A one-file library for adding machine and human readable bibtex to an article },
abstract={ A simple include file to make bibtex available 
in a document in both machine and human readable format. 
Machine readable is added to extended information 
which can be read with tools such as exiftool ( https://exiftool.org/ ).
While typically not complete at time of pdf creation,
other tools such as toobib can be used to complete the citation
and of course publishers may be able to modify it too as 
more is known. The human readable form need not beincluded
in document types not suited for that but then automated citation
may still be easy.  
See also some exchanges on the Texhax mailing list @tug.org },
institution={},
license={Knowledge sir should be free to all },
publisher={Mike Marchywka},
email={marchywka@hotmail.com},
authorid={orcid.org/0000-0001-9237-455X},
  filename = {mjmaddbib.tex},
  url = {},
  version = {0.0.0},
  date-started = {}
}

<one line to give the program's name and a brief idea of what it does.>


Conceived and written by Mike Marchywka from 2019 to present.
See dates in individual code pieces as they were 
generated from my wizards. 
Copyright (C) <year> <name of author>


This program is free software: you can redistribute it and/or modify it under
the terms of the GNU General Public License as published by the Free Software
Foundation, either version 3 of the License, or (at your option) any later
version.

This program is distributed in the hope that it will be useful, but WITHOUT ANY
WARRANTY; without even the implied warranty of  MERCHANTABILITY or FITNESS FOR
A PARTICULAR PURPOSE. See the GNU General Public License for more details.

You should have received a copy of the GNU General Public License along with
this program.  If not, see <http://www.gnu.org/licenses/>.

   THIS SOFTWARE IS PROVIDED BY THE COPYRIGHT HOLDERS AND CONTRIBUTORS
   "AS IS" AND ANY EXPRESS OR IMPLIED WARRANTIES, INCLUDING, BUT NOT
   LIMITED TO, THE IMPLIED WARRANTIES OF MERCHANTABILITY AND FITNESS FOR
   A PARTICULAR PURPOSE ARE DISCLAIMED.  IN NO EVENT SHALL THE COPYRIGHT OWNER OR
   CONTRIBUTORS BE LIABLE FOR ANY DIRECT, INDIRECT, INCIDENTAL, SPECIAL,
   EXEMPLARY, OR CONSEQUENTIAL DAMAGES (INCLUDING, BUT NOT LIMITED TO,
   PROCUREMENT OF SUBSTITUTE GOODS OR SERVICES; LOSS OF USE, DATA, OR
   PROFITS; OR BUSINESS INTERRUPTION) HOWEVER CAUSED AND ON ANY THEORY OF
   LIABILITY, WHETHER IN CONTRACT, STRICT LIABILITY, OR TORT (INCLUDING
   NEGLIGENCE OR OTHERWISE) ARISING IN ANY WAY OUT OF THE USE OF THIS
   SOFTWARE, EVEN IF ADVISED OF THE POSSIBILITY OF SUCH DAMAGE.


\fi

%\usepackage[pdftex]{graphicx}      % alternative graphics specifications
\usepackage{hyperref}      %


\newcommand{\mjmstartbib}[2]
{
\def\mjmbibentry{@#1\{#2}
\def\mjmbiboneentry{@#1\{#2}
\def\mjmbibpre{x-bib{-}}
\def\mjmday{\day}
%\mjmaddbib{run-day}{\mjmday}
%\mjmaddbib{run-month}{\expandafter\month}
%\mjmaddbib{run-year}{\year}
\mjmaddbib{filename}{\jobname}
\mjmaddbib{run-date}{\today}
}
% modify toobib and change this from "x-bib" to something like bibtex or mjm- whatever 
\newcommand{\mjmbibmunge}[1] {x-bib-#1}

\newcommand{\mjmaddbib}[2]
{
{\hypersetup{pdfinfo={{\mjmbibmunge{#1}}={#2}}}}
\edef\mjmbibentry{\unexpanded\expandafter{\mjmbibentry,}

    #1 =\{#2\} }
%\edef\mjmbiboneentry{\unexpanded\expandafter{\mjmbiboneentry,}
% this works but gives warning and linebreak is removed.
%\edef\mjmbiboneentry{\unexpanded\expandafter{\mjmbiboneentry,\linebreak}
\edef\mjmbiboneentry{\unexpanded\expandafter{\mjmbiboneentry, }
#1 =\{#2\}
}

} % mjmaddbib


\newcommand{\mjmshowbib}
{

\mjmbibentry

\}
} % mjmshowbib

\newcommand{\mjmshowbibone}
{
\mjmbiboneentry \}
} % mjmshowbibone

\newcommand{\mjmdonebib}
{

{\hypersetup{pdfinfo={\mjmbibmunge{bibtex}={\mjmshowbibone}}}}

} % mjmdonebib


% af 
\newcommand{\mjminputlisting}[2]
{
%\begin{figure}[H]
\lstinputlisting{#1}
%\caption{#2}


#2

%\end{figure}
}

%%%%%%%%%%%%%%%%%%%%%%%%%%%%%%%%%%%%%%%%%%%%%%%%%%%%%%%%%%%%%%%%%%

%\newcommand{\mjmaddbib}[2]{\hypersetup{ pdfinfo={ x-bib-#1 = {#2}}}\mjmsummabib{#1}{#2}}

\newcommand{\mjmaddbibonly}[2]{\hypersetup{ pdfinfo={ x-bib-#1 = {#2}}}}
\newcommand{\mjmaddbibe}[2]{\hypersetup{ pdfinfo= x-bib-#1 = #2}}

\newcommand{\mjmtable}[2]{
\begin{table}[H] \centering
\begin{tabular}{#1}
#2
\end{tabular}
\end{table}

}
% for archiving and file list 
% David Carlisle You can use   \textbf{\detokenize{ pmg_ratios.svg}} \IfFileExists{ pmg_ratios.sv}{yes}{no} to make the tokens safe for typesetting in text mode.


\newcommand{\mjmusesitem}[2]{

%\item {\bf $ #1 $  } : #2  % 1 is #1 and 2 is  #2 xxx 
\item {{\detokenize{ #1 } }} : #2  % 1 is #1 and 2 is  #2 xxx 
\IfFileExists{#1}{}{{\bf not found} }

} % mjmuses

\newcommand{\mjmreleasewarning}
{
{\bf This is a draft and has not been peer reviewed or completely proof 
read but released in some state where it seems worthwhile given 
time or other constraints. Typographical errors are quite likely 
particularly in manually entered numbers. This work may 
include  output from software which has not been fully debugged.  
For information only, not for use for any particular purpose see
fuller disclaimers in the text.  Caveat Emptor.}
}

\newcommand{\mjmwarningtoo}
{\bf {This is a draft which may not have been fully proofread and certainly
not peer reviewed. Read the disclaimers and take them seriously.
The reader is assumed familiar with the related literature and
controversial issues. For information and thought only not intended
for any particular purpose. Caveat Emptor  }}

\newcommand{\mjmwarntopic}
{\bf {  This work addresses a controversial topic and likely advances one
or more viewspoints that are not well accepted in an attempt to
resolve confusion.   
The reader is assumed familiar with the related literature and
controversial issues and in any case should seek additional 
input from sources the reader trusts likely with differing opinions. 
For information and thought only not intended
for any particular purpose. Caveat Emptor  }}


\newcommand{\mjmwarnfeed}
{\bf { Note that any item given to a non-human must be checked for safety alone and in combination with other ingredients or medicines  for that animal. Animals including dogs and cats have decreased tolerance for many common ingredients in things meant for human consumption. }}


\newcommand{\mjmwarnme}
{\bf { I am not a veterinarian or a doctor or health care professional
 and this is not particular advice
for any given situation.  Read the disclaimers in the appendicies or text, take them seriously and take prudent steps 
to evaluate this information. 
 }}

\newcommand{\mjmexplainbib}
{{The release may use an experimental bibliography code that
is not designed to achieve a particular format but to
allow multiple links to reference works with modifications
to the query string to allow identification of the citing
work for tracking purposes. This may be useful for a bill-of-materials
and purchases later.
}}








\newcommand{\mjmauthor}{Mike J Marchywka }
\newcommand{\mjmmakedate}{2024-04-12 }
\newcommand{\mjmbasename}{\jobname}
\newcommand{\mjmaddbio}{mjm_tr,releases,potusebib,../casesum/casesum,sbenzbib}
%\newcommand{\mjmversion}{0.00}
%\newcommand{\mjmtrno}{MJM-2024-010}
%\newcommand{\mjmbibday}{12}
%\newcommand{\mjmbibmo}{04}
%\newcommand{\mjmbibyear}{2024}

% the build script changes these to creation day doh 
\newcommand{\mjmbibday}{12}
\newcommand{\mjmbibmo}{04}
\newcommand{\mjmbibyear}{2024}


\newcommand{\mjmmakebibday}{\number\day}
\newcommand{\mjmmakebibmo}{\number\month}
\newcommand{\mjmmakebibyear}{\number\year}

\newcommand{\mjmbibtype}{techreport}

\newcommand{\mjmbibname}{marchywka-\mjmbib}
\mjmstartbib{\mjmbibtype}{\mjmbibname}


\newcommand{\mjmemail}{marchywka@hotmail.com}
%\newcommand{\mjmaddr}{44 Crosscreek Trail Jasper GA 30143 USA}
\newcommand{\mjmaddr}{44 Crosscreek Trail Jasper GA 30143 USA}
\mjmaddbib{title}{\mjmtitle}
\mjmaddbib{author}{\mjmauthor}
\mjmaddbib{type}{\mjmbibtype}
%\mjmaddbib{name}{marchywka-\mjmbib}
\mjmaddbib{name}{\mjmbibname}
\mjmaddbib{number}{\mjmtrno}
\mjmaddbib{version}{\mjmversion}
\mjmaddbib{institution}{not institutionalized, independent }
\mjmaddbib{address}{ \mjmaddr}
\mjmaddbib{date}{\today}
\mjmaddbib{startdate}{\mjmbibyear -\mjmbibmo -\mjmbibday }
%\mjmaddbib{day}{\mjmbibday}
%\mjmaddbib{month}{\mjmbibmo}
%\mjmaddbib{year}{\mjmbibyear}
\mjmaddbib{day}{\mjmmakebibday}
\mjmaddbib{month}{\mjmmakebibmo}
\mjmaddbib{year}{\mjmmakebibyear}

\mjmaddbib{author1email}{\mjmemail}
\mjmaddbib{contact}{\mjmemail}
\mjmaddbib{author1id}{orcid.org/0000-0001-9237-455X}
\CatchFileEdef\mjmpages{\mjmbasename.last_page}{\endlinechar=-1\relax}
% TODO FIXME add this to the skeleton text 
%\mjmaddbib{pages}{ \input{\mjmbasename.last_page}}
\mjmaddbib{pages}{ \mjmpages}
%\mjmaddbib{filename}{\mjmbasename}

\begin{comment}
\newcommand{\checkfortwo}[1]{%
  \ifcsname#1\endcsname%
  VERSION =        \{\mjmversion \today \mjmstatus \},
  \else%
  VERSION =        \{\mjmversion\},
  \fi%
}

\end{comment}
%\mjmaddbib{bibtex}{\mjmfullbib}

\mjmdonebib



\lhead{\mjmauthor,  \mjmtrno }


%\lhead{M Marchywka,  \mjmtrno }
%\rhead{ \mjmversion not for public release}
%\rhead{ { \today }  v. \mjmversion for release without review }
%\rhead{ { \today }  v. \mjmversion NOT public DRAFT }
%\rhead{ { \today }  v. \mjmversion { }  NOT public NOTES }
\rhead{ { \today }  v. \mjmversion { }  \mjmstatus }

\newfloatcommand{capbtabbox}{table}[][\FBwidth]


%%%% build flags 
%\newlength{\desttabw}  \setlength{\desttabw}{4in}
\newlength{\desttabw}  \setlength{\desttabw}{\textwidth}
\newlength{\chainwidth}  \setlength{\chainwidth}{.4\textwidth }
\newlength{\slantwidth}  \setlength{\slantwidth}{.2\textwidth }
\newlength{\subfigwidth}  \setlength{\subfigwidth}{.3\textwidth }
\newlength{\fullfigwidth}  \setlength{\fullfigwidth}{.8\textwidth }
\newlength{\subwfigwidth}  \setlength{\subwfigwidth}{.75\textwidth }
\newlength{\subwfigwidthrot}  \setlength{\subwfigwidthrot}{\textwidth }
\newlength{\myboxwidth}  \setlength{\myboxwidth}{.3\textwidth }
\newlength{\picwidth}  \setlength{\picwidth}{.4\textwidth }
% set to center for nowmal output 
\newcommand{\destflushtab}{flushleft}
\includecomment{mdpicomment}
\excludecomment{draftcomment}
\excludecomment{badmathcomment}
\excludecomment{showworkcomment}
% this does not fing work ... 

\newcommand{\mjmed}[1]{
%\begin{mjmedx} 
[ mjm : #1   ]
%\end{mjmedx} 
}  

% thinking outload
\newcommand{\mjmtolx}[1]{}
\newcommand{\mjmtolxx}[1]{}
\newcommand{\mjmtol}[1]{
 \fbox{  
% thi does not ing work right 
\begin{minipage}[t]{\textwidth}
{ \centering{\bf{Thinking outloud}} }
\par   
#1 
\end{minipage} 
}
}

\newcommand{\mjmpicture}[3]
{
\begin{figure}[H]
{ \includegraphics[height=3in,width=4in]{keep/#1} }
\caption{#2}
\label{fig:#3}
\end{figure}
} % mjmpicture


\newcommand{\mjmaside}[1]{
 \fbox{  
% thi does not ing work right 
\begin{minipage}[t]{\textwidth}
{ \centering{\bf{Aside: }} }
\par   
#1 
\end{minipage} 
}
}




\newcommand{\mjmgraphics}[1]{#1 }
\newcommand{\mjmfullplot}[1]{\includegraphics[width=\fullfigwidth]{keep/#1}}
%\newcommand{\mjmincludeplot}[1]{\includegraphics[width=\fullfigwidth]{#1}}
%\newcommand{\mjmincludeplot}[1]{\includegraphics[width=\subfigwidth]{#1}}
\newcommand{\mjmincludeplot}[1]{\includegraphics[height=3in,width=\fullfigwidth]{#1}}

% include here as likely to be doc specific 
%\newcommand{\mjmreffig}[1]{Fig. \ref{#1}}
\newcommand{\mjmreffig}[1]{Fig. \ref{fig:#1}}
\newcommand{\mjmreftab}[1]{Table  \ref{tab:#1}}
\newcommand{\mjmrefapp}[1]{Appendix  \ref{appendix:#1}}


%cp yyy2.pdf ~/d/latex/keep/pp20171124biotin.pdf
\newcommand{\mjmdatedplot}[1] 
{ \includegraphics[height=3in,width=\fullfigwidth]{keep/pp20171124#1} }
% right now there are too many figs to save for the end apaprently
% hard limit is 36
\newcommand{\mjmbeginfigure}{\begin{figure}[H] }
%\newcommand{\mjmbeginfigure}{\begin{figure}[p] }
\newcommand{\mjmfigure}[1]{
%\begin{figure}[H]
\mjmbeginfigure

#1 

\end{figure}
}
%\extrafloats{100}

\newcommand{\mjmlisting}[1]
{
\begin{lstlisting} 
#1 
\end{lstlisting}
}

\newcommand{\mjmold}[1]{ } 



\newcommand{\mjmeqn}[1]{\begin{equation} #1 \end{equation}  } 


\newcommand{\mjmhappycu}{
\begin{figure}[htb] 
\centering
\mjmfullplot{happy2seeyou.jpg}
\caption{  Copper(white) and Zinc(yellow) dosing per day averaged over 
prior 10 day period as dosing was highly variable due to rotations
of various nutrients. 
18046 is 2019-05-30 when the cough was first noted to be gone for a few weeks. 19823 is  2024-04-10 the last date for which data was obtained. The cough stopped prior to the start of the Zinc and gradually increased to a notable background level over most of this interval although notes were incomplete. 
19531 2023-06-23 notes the start of Cu depletion and chronic cough was noted by late Fall. During this time Zinc greatly exceeded copper dosing.  }
\end{figure}
} % mjmhappycu

\newcommand{\mjmtrixiecu}{
\begin{figure}[htb] 
\centering
\mjmfullplot{trixie_cu.jpg}
\caption{   Trixie copper consumption since arrival. Daily amounts
( white)) and trailing 10 day average( yellow). Copper started to
be significant around day 19730 in response to coughing. Day 19807
marked the end of the copper fast as well as the end of Clavamox
which was prescribed due to worsening when copper stopped days
earlier. 
  }
\end{figure}
} % mjmtrixiecu

\newcommand{\mjmrockycu}{
\begin{figure}[htb] 
\centering
\mjmfullplot{rocky_cu.jpg}
\caption{    Rocky 10 day trailing average copper (white )
. 
  }
\end{figure}
} % mjmrockycu

\newcommand{\mjmrockyi}{
\begin{figure}[htb] 
\centering
\mjmfullplot{rocky_i.jpg}
\caption{    Rocky iodine intake daily and with 10 and 30 day trailing averages. Patterns are difficult to discern with the pulsed dosing.  
  }
\end{figure}
} % mjmrockyi



\newcommand{\mjmmixiecu}{
\begin{figure}[htb] 
\centering
\mjmfullplot{mixie_cu.jpg}
\caption{    Mixie daily copper (white ) intake 
. 
  }
\end{figure}
} % mjmmixiecu

\newcommand{\mjmbrowniecu}{
\begin{figure}[htb] 
\centering
\mjmfullplot{brownie_cu.jpg}
\caption{    Brownie 10 day trailing average copper (white )
. 
  }
\end{figure}
} % mjmrowniecu


\newcommand{\mjmanniecu}{
\begin{figure}[htb] 
\centering
\mjmfullplot{annie_cu.jpg}
\caption{    Annie  
  }
\end{figure}
} % mjmannieycu




\newcommand{\mjmhersheycu}{
\begin{figure}[htb] 
\centering
\mjmfullplot{hershey_cu_i.jpg}
\caption{    Herhsey 10 day trailing average copper (white )
and iodine (yellow). 
  }
\end{figure}
} % mjmhersheycu

\newcommand{\mjmhuntcu}{

\begin{table}[H] \centering
\begin{tabular}{|r|r|c|r|}
\hline
\multicolumn{4}{c}{Title}\\
\hline
Cu &&& \\
Zn&&&\\
Fe&&&\\
Mo&&&\\
H/ OH&&&\\
S and amino acids &&& reduce toxicity \cite{Jensen_Maurice_Influence_Sulfur_Amino_1979}\\
PO4&&&\\
Fructose&&&\\
Tyr&&&\\
Trp&&&\\
Pi-complex&&&\\
Fenton&&&\\
Ammonia, Amides, N&&&\\
Citrate&&&\\
Ascorbate&&&\\
garlic &&& enhancing bioavailability Cu etc \cite{JSFA:JSFA8113}\\
Microbial products&&&\\
phytate&&&\\
\hline
\hline
\end{tabular}
\caption{ Some entities that may interact with copper}
%\label{}
\end{table}


} % mjmhuntcu


\newcommand{\mjmcollapse}{

Collapsed trachea in dogs is not fully understood
although symptoms and features are well known 
and is somewhat age related
\cite{Jeung_Sohn_An_retrospective_study_2019}
although also described as congenital
\cite{PMC5746443}.
This work is predicated on the belief that 
"tracheobronchomalacia" is largely a nutritional issue
with amino acid limitation being one factor but also
physciological crosslinking which depends on copper
largely in lysuyl oxidase \cite{PMC7204390}.
This should be distinguished from pathological crosslinking
which may occur through other mechanisms. 
While much is not known about cartilage crosslinking,
turnover, and remodeling some recent results do point
to unexpected beneficial effects of copper mediated
cross linking over week time scales \cite{PMID25907046}.
Copper deficiency has been associated with lung
development defects in rats \cite{PMC2018314} and airway and arteriole elastin
were at least partially restored after 60 days of additional copper.
In poultry,
infection is known to cause some tracheal symptoms \cite{PMID9835339}
and  relationship to copper status is considered.
Copper compounds may reduce the virulence of some  organisms for short periods
even if not completely able to clear a pathogen\cite{PMC4542191}
and in particular Cu was recently shown to be effective against
the toxin of one anaerobe  \cite{doi:10.1021/jacs.7b01084} .
Collapsed trachea in dogs does not appear to be commonly associated
with copper in the
literature.
\mjmtolx{
  although copper storage disease seems to be an
active area of investigation\cite{PMC4783930}
 and copper status does come up
in other investigations of tracheal damage as highlighted above.
It is possible that the original motivation
 is partially correct and that extra copper combined with garlic did
 allow for better crosslinking to stiffen tracheal cartilage and
 prevent dynamic collapse and "honking".
}
%part2

} % mjmcollapse 
\newcommand{\mjmdcm}{


As early as 1993, copper restriction in rats was observed to 
produce cardiomuuoptahy
\cite{Medeiros_Davidson_Jenkins_Unified_Perspective_1993}.
while reduced cytochrome c oxidase activity was observed
in 1999 paricularly with simultaneously high fat diets
\cite{Jalili_Medeiros_Wildman_Aspects_Cardiomyopathy_1996}.
Altohough confusingly copper chelation with trientine
is also thought to show promise in HCM treatment 
\cite{Reide001803} but its unclear exactly how how
it rearranges copper distribution. 

The heart relies on 4 valves and a rapid contraction subsequent
to depolarization to move blood through the body. Mechanical
properties of the valves as well as energy efficiency may be expected
to be important to maintain performance without desperate remodelling. 
These requirements motivate an interest in accurate protein
translation, and copper dependent physsiological crosslinking, removal of
deposits such as mineralization, and mitochondria performance. 
Empirically, copper deficiency can lead to cardiac hypertrophy with increased
mitochondria \cite{PMID12539966}.
Dilated and hypertrophic cardiac myopathy can both
be related to mitrochondria with  "oxidative stress"
as a signal for more copper at cytochrome oxidase. "Oxidative stress"
has been reported to increase muscle mass while 
reducing performance \cite{Ahn_Ranjit_Premkumar_Mitochondrial_oxidative_stress_impairs_2019}.
"Oxidative stress" is often blamed in the literature as 
a cause of various problems but it may in fact also be a signal.
ROS signalling is well known by now but specifically
it may help get sufficient copper to the mitochondria.

As early as 2008, a pressure overload mouse model of DEM 
regressed with copper supplementation
\cite{PMC2671677}.
An recent 2023 work demonstrated correlations between heart copper
content and cardiac parameters in a group of identically fed mice
while also pointing to the heart as a regulator of copper concentration
overall \cite{10.3389/fcvm.2023.1089963}.

\mjmtolx{
Copper loading
of cytochrome C oxidase relies on an oxidized
Cox11 to interact with Cox19 \cite{Bode01072015} and this may be inhibited
by GSH but is enhanced by GSSG. Redox regulation in the IMS seems to be integral
to proper copper disposition \cite{IUB:IUB1620} and indeed mitochondrial
related signaling \cite{PMC5448071} . The latter reference also points
to tissue specific mitochondrial isoform expression suggesting that maybe
some related diseases are states rather than traits and hence correctable with
signaling.
 Certainly excess antioxidants would be suspicious ( for example reference 24
\cite{PMC3852343} in \cite{PMC5448071}) .
However, the enhancement by GSSG suggests that the presence of oxidized antioxidants
may be beneficial but not in their reduced state.
}

Over the  same time however, concerns about diet linked
DCM in dogs have emerged.  A genetic link is also being
investgated and a recent GWAS in dobermans pointed to two genes,
RNF207 and PRKAA2 as risk factors \cite{PMC10506233}
 but did not mention copper. 
However, RNF207 may mediate degradation of ATP7A
\cite{Zhao_Zeng_Wang_Targeting_2023}
while PRKAA2 comes up in cuproptosis 
\cite{10.3389/fcell.2022.996307}.

Dobermans are at remarkably high risk of DCM 
\cite{Ezer_Saarinen_Katayama_Identification_novel_genetic_2023}.
Interestingly, "standard Dobermans" are also at high risk for hypothyroidism
\cite{PMC9552398}.

Its possible the two concerns are related in more copper
is being absorbed as less is  transported to target organs
such as the heart.
This connection between dysregulated copper metabolism and 
heart disease has been considered recently in humans
\cite{PMC8838622} in a work that reviews many important
aspects of copper metabolism. .


} % mjmdcm 
\newcommand{\mjminfect}{


Combined copper and zinc deficiency was observed to
reduce response to \mjmdisease mRNA vaccines with only
minimal copper deficiency 
\cite{Chillon_Demircan_Hackler_Combined_copper_zinc_2023}
The present work considers copper status in light of
other nutrients notably amino acids such as Trp
and Tyr and with zinc being a possible competitor.


In human health, zinc seems to have taken precedence over copper
most recently with some headlines related to the \mjmdisease
pandemic
\cite{Ata_Bradshaw_Zinc_Supplementation_induced_Copper_2024}
\cite{PMC8970610}
. 
Some studies suggest copper is not a factor in \mjmdisease
due to measurements like serum copper levels
\cite{Rozemeijer_Hamer_Heijboer_Micronutrient_Status_Critically_2024}
although ceruloplasmin as an acute phase protein
can elevate levels  during stress even with a 
deficiency in copper
\cite{Harvey_McArdle_Biomarkers_copper_status_2008}.
In  sample of 70 patients presrribed zinc, many had symptoms consistent
with copper deficiency \cite{PMID26085547}



Copper may antagonize many pathogens including
H pylori
\cite{Bernegger_Brunner_VizoviUx0161badcharvv353ek_novel_FRET_peptide_2020}
and clostridum
\mjmtol{
Repression of toxin production by tryptophan in Clostridium botulinum type E
\cite{PMID2256780}. } 


} % mjminfect 
\newcommand{\mjmthy}{

Dobermans are at remarkably high risk of DCM 
\cite{Ezer_Saarinen_Katayama_Identification_novel_genetic_2023}.
Interestingly, "standard Dobermans" are also at high risk for hypothyroidism
\cite{PMC9552398}.

Inerestlingly, copper deficiency in rats can reduce thyroid
hormone levels and body temeprature \cite{Lukaski_Hall_Marchello_Body_temperature_thyroid_1995}.

Of the dofs discussed here, only Rocky was a concern for hypothyroiism although
Happy came back with very low normal thyrroid hormone result prompting
further consideration. Besides iodinne, other contributros may include
tyrisone and sodium benzoate. 

} % mjmthy 
\newcommand{\mjmmisc}{


Degenerative neurological or muscular diseases
may be intertwined with copper status although 
the relationship is again confusing. Similarly with
arthritis. 

Deficiency seems to effect prefernetially proteins involved
in neuronal projection and diabetes and iron handling
\cite{PMID36583241}.

Rats fed a copper deficient diet shows neurological symptoms
by 7 weeks and had reduced tyrosine hydroxylase and
SOD activity ZZ
\cite{Morgan_ODell_EFFECT_COPPER_DEFICIENCY_1977}.

A 2017 study explored the effects of copper and vitamin C
as well as other molecules such as
clioquninol on abeta and in vitro neurons suggesting
abeta could be cleaved by copper in the presence of oxygen
as well as an anti-oxidant such as vitamin C although
restoration of neuronal functioning was only partial
\cite{C7SC01787A}.
Interestingly, copper-ascorbate oxidation of tryptophan may be
suppressed by Trp chelation of copper at high trp concentrations
\cite{PMID27406076}
suggesting reduced amounts may give copper more ability to damage
an already low supply.
This is interesting in terms of a utrient interaction hypothesis
on copper toxicity.
And in fact as early as 2012 it was determined that
tryptophan intake could reduce copper toxicity at least in carp
\cite{Hoseini_Hosseini_Soudagar_Dietary_tryptophan_changes_serum_2012}.


Some precednce for metal modulated toxicity existed back to 1999
when work with cultured neurons showed a dose dependent
reduction in abeta toxicity with Zn \cite{Lovell_Xie_Markesbery_Protection_against_amyloid_beta_1999}.
By 2005 toxicity of amyloid beta and the metals zinc, iron, and
copper was investigated under conditions that created more toxicity
with iron and zinc but not copper while amyloid beta reduced metal
toxicity in rats
\cite{Bishop_Robinson_Amyloid_Paradox_AmyloidupbetaMetal_2004}.
In 2021, Ni was found in important amounts in a commercial
abeta40 preparation
\cite{Benoit_Maier_nickel_chelator_dimethylglyoxime_2021}
and was found to mediate dityrosine crosslinks
\cite{Berntsson_Vosough_Svantesson_Residue_specific_binding_2023}
similar to the dityrosine crosslinks induced by copper
found in 2004
\cite{Atwood_Perry_Zeng_Copper_Mediates_Dityrosine_Cross_2004}.

A 2013 work found in vitro physiological conditions
caused copper to prevent fibril formation
\cite{Mold_OuroGnao_Wieckowski_Copper_prevents_amyloid_upbeta1_2013}.

By 2022,  work focusing on moving copper into the cell considered
many aspects of copper misallocation and devised a copper specific
shuttle peptide to deliver Cu from abeta  \cite{D2SC02593K}.

One work in  2022 addressed
AD as a consequence of copper deficnecy because
\cite{Klevay_contemporaneous_epidemic_2022}
\begin{quote}
It is hypothesised that copper deficiency is a plausible cause of Alzheimer's
disease(Reference Klevay84). Patients are thinner than normal; weight loss
precedes dementia and is associated with greater dementia and neurobehavioural
symptoms. Nutritional compromise contributes to morbidity. Cytochrome oxidase
depends on copper for activity; at least fourteen publications reveal decreased
activity in brain of Alzheimer's patients. Brain copper and caeruloplasmin
also are decreased. This hypothesis is the only one that explains why Alzheimer's
disease occurs earlier and is more common in Down's syndrome. Superoxide
dismutase (SOD1) depends on copper for activity; its gene is on chromosome
21. This enzyme is elevated in Down's syndrome (trisomy 21) and is decreased
in people with monosomy. It seems likely that people with Down's syndrome
have a higher than average requirement for dietary copper because copper
is incorporated into superoxide dismutase and is unavailable for other uses.
Thus, Alzheimer's disease fulfills the first two of Golden's criteria (above)
for deficiency.

\end{quote}


Folklore regarding copper persists and yet clinical trials
for Cu in arthritis continue to show lack of any benefit
\cite{10.1371/journal.pone.0071529}
even as other controlled tests show some effects of Cu on
processes related to collagen properties
\cite{PMID6110524}

Remarkably, copper sulfae was shown to protect against
ETC damage by MPP \cite{RUBIOOSORNIO20171} suggesting
some activity against toxic insults.
Copper histidine is used to treat Menkes disease
which is a defect in ATP7A
\cite{Munakata_Sakamoto_Kitamura_effects_copper_2005}.



At least one report found a potentially meaningful association
between copper intake and kidney stone odds ratio
with a non-linear but monotonic inverse relationship
\cite{PMC9469499}. This is interesting in the case of Hershey
who was found to have bladder stones.  


\mjmtol{ this may not belong here but relevant to
other Cu stuff,
A recently published work suggests copper delivery is the
important part of a new ALS drug but the work also suggests
a "hyperreductive state" around hypoxic mito that
promote relase of Cu from the drug complex
\cite{Hilton_Kysenius_Liddell_Evidence_disrupted_copper_2024}/
pointing to a possible more general mechnism.
The work goes onto suggest possible role in Parkinson's Disease
but does not address AD.
At least one observational study found a 
negative correlation between odds ratio for Parkinson's and copper
intake \cite{Zeng2024}
}



} % mjmmisc 

\newcommand{\mjmcancer}{
Copper has been studied in relation to cancer being
considreed both as a limiting nutrient and a source
of cuproptosis in the overload state
\cite{PMC10327296}.
Copper is essential for many growth processes and can activate
receptor tyrosine kinases without a ligand making it
a target for cancer \cite{PMC6537018}.
However, these processes are also esential for
maintenance and healing in the organism and any net
benefit would have to consider host fitness as well
as cancer cells again using a clinical endpoint to deciee
on approach. 
In some studies, copper depletion was considered to be
motivated by the observation that copper chelatiors
reduced cancer growth and equating chelation with depletion.
However, role confusion may have applied as the
chelated copper could generate ROS's with obvious
cytotoxic acitivity. 


 Copper storage diseases and in essence "overdose" are
well known \cite{Fuentealba2003} % this also mentions absorption idiosyncracies
 and a role in cancer is suspected \cite{PMC5412452}
.
% TODO find the other comments for here 
However, see the comments below about complexed copper actually
being an active compound similar to other drugs,
perhaps Pt based for example,  which may kill cancer.
Copper toxicity has been noted to differ between host cell types
and may be reducible, at least in  Long Evans Cinnamon rats, with thiamine or lipoic
acid \cite{ANA:ANA10276}.
Lysyl oxidase activation, the goal of this therapy, is also associated
with cancer spread \cite{PMC5399287}\cite{PMC5413586}\cite{PMC5297697}
. Although it is likely to remodel
possible tumor locations, its role in growth or  metastases
in a clinically relevant situation is currently unresolved
( see for example \cite{PMC3280126}
 or \cite{PMC5103911} and the survival curve in figure 1).
Incidence of liver cancer in Wilson's Disease patients is remarkably low
\cite{PMC4803712} and a discussion of possible treatment effects
\cite{doi:10.1093/qjmed/hcg114} points out that the copper per
se rather than removal is likely to help while also mentioning
differences with iron overload. Indeed, extra copper that prevents
iron overload may be therapeutic as originally intended.


We should note that complexed copper is not equivalent
to copper deficiency as the complex may not be inert.
However, when an ROS generating complex has been observed it  effects were
diminished by antioxidants \cite{PMC4223620}.
This also suggests that copper depletion per se
may not kill cancer cells as much as copper complexes
and that concerns about copper supplements and cancer may not
be significant.




} % mjmcancer

\newcommand{\mjmcox}{

Supplemental copper has been noted to improve
symptoms in one case report including symptoms such as hearing loss \cite{PMID15902551}
attributed to defects in cytochrome C oxidase copper loading
and restore function in mutant yeast \cite{doi:10.1093/hmg/ddu069}
.Cytochrome C oxidase levels in rat hearts were shown to
be related to copper deficiency as early as  1998  \cite{Rossi1998}.
It would be interesting to determine if more problems in dogs
are related to specific mutations in mitochondrial copper handling.
We note again that vitamin K could contribute in similar places
and may be synergistic with copper for connective tissue quality
as well as in eukaryotic mitochondria \cite{Vos1306}\cite{PMC5022102}
.

Two types of excess have been identified as potentially
important to pathogenesis- mineral and antioxidant.
Iron overload copper deficiency was identified early on as
a concern with animals getting high iron diets and generally
free of parasites in stark contrast to the likely situation over
evolutionary time scales. A second type emerged on consideration
of the copper response and cytochrome C oxidase copper loading -
that of antioxidant overload. ROS have generally gained more acceptance
as having physiological roles at low concentrations rather than simply
being a source of damage. Literature related to these experiments
suggests a very specific role in mitochondrial function.
The original concern about antioxidant overload was mostly
confined to vitamin E-K antagonism and it is not
clear how or if these concerns relate. Coupled with empirical
deleterious effects of some antioxidant combinations in clinical trials
( one high profile  example \cite{doi:10.1056/NEJM199605023341802} )
,
it is clear that antioxidant excess should be considered
as a problem with some diets.  The antioxidant paradox
now seems to be gaining acceptance , for example see \cite{PMC4412352}
 and \cite{PMC5021979}
.
%label cytoc
%label EK 
Copper is decidedly pro-angiogenic 
and several studies have shown  effects of copper consumption
increasing tumor growth in animals  while both
chelators and ionophores are being investigated for treatments 
\cite{Chen_Min_Wang_Copper_homeostasis_cuproptosis_2022}.
However, pro-growth and angiogenesis could also be expected
during regeneration which itself may require copper. 

\mjmtol{
Copper loading
of cytochrome C oxidase relies on an oxidized
Cox11 to interact with Cox19 \cite{Bode01072015} and this may be inhibited
by GSH but is enhanced by GSSG. Redox regulation in the IMS seems to be integral
to proper copper disposition \cite{IUB:IUB1620} and indeed mitochondrial
related signaling \cite{PMC5448071} . The latter reference also points
to tissue specific mitochondrial isoform expression suggesting that maybe
some related diseases are states rather than traits and hence correctable with
signaling.
 Certainly excess antioxidants would be suspicious ( for example reference 24
\cite{PMC3852343} in \cite{PMC5448071}) .
However, the enhancement by GSSG suggests that the presence of oxidized antioxidants
may be beneficial but not in their reduced state.
}



} % mjmcox 
\newcommand{\mjmlox}{

Lysyl oxidase bad for vessels \cite{PMID38402026} calxification.
but may  be related to metallization issues \cite{PMID9524359}.


Regulation at transcriptional and translation and post-translational
levels is confusing. For example, it has been described in
1998 as \cite{SmithMungo_Kagan_Lysyl_oxidase_Properties_regulation_1998},
\begin{quote}
While enzyme activity levels were decreased in the skin of weanling rats
fed a copper deficient diet, the basal, steady-state levels of LO specific
mRNA or immunodetectable LO protein were not significantly reduced (Rucker
et al., 1996). These results suggest both that the biosynthesis of the enzyme
is not markedly affected by copper deficient diets and that the increasing
percentage of copper-deficient, catalytically compromised enzyme molecules
presumed to accumulate during this dietary treatment remain relatively stable.
Notably, copper-deficient diets significantly reduced cardiac LO activity
and induced cardiac pathology in male but not in female rats (Werman et
al., 1995). 
\end{quote}
or more to the point from the same year,
\cite{Rucker_Kosonen_Clegg_Copper_lysyl_oxidase_1998}
\begin{quote}
Although nutritional copper status does not influence the accumulation of
lysyl oxidase as protein or lysyl oxidase steady state messenger RNA concentrations,
the direct influence of dietary copper on the functional activity of lysyl
oxidase is clear. The hypothesis is based on the possibility that copper
efflux and lysyl oxidase secretion from cells may share a common pathway.
The change in functional activity is most likely the result of posttranslational
processing of lysyl oxidase.
\end{quote}

It has been observed to upregulate in the injured
newborn lung
\cite{Zhong_Mahoney_Khatun_Lysyl_oxidase_regulation_2022}
sugesting increased levels may be a response to an insult
rather than a cause of damage. 



In 2001, it was observed that bovine lyxyl oxidase had enzymatic
activity without copper but was less stable
\cite{Tang_Klinman_Catalytic_Function_2001}
although details on reactions catalyzed could not have
been fully explored. 


However, metallization may not be complete
and feedback systems may increase expression to achieve
an activity level. Note too that "crosslinking" is
a variable modification and physiological as well as
pathological crosslinking can occur. 
While "quantity versus quality" will be the subject of
another work, its important to remember that increased
expression of lysyl oxidase genes and more patholoical
crosslinking could occur in the absence of sufficient copper.
Mature functional lysyl oxidase contains an unusual 
 lysine tyrosylquinone (LTQ) which itself is formed in 
a copper dependent process \cite{PMC2749888}.
Depence on multiple tyrosines or tryptophans   can
increase the odss of generating dysfunctional
enzymes which may be inactive or perform unintended functions 
when these amino acids are limited. This theme of amino
acid starvation also appears in concerns about ceruloplasmin
and more generally with aging. 

Lysyl oxidase  expression has been associated with degenerative
mitral valve disease in humans\cite{PMC5541772}.

} % mjmlox 
\newcommand{\mjmcp}{
Ceruloplasmin contains 6 coppers
% and two tytosines
and  histidines needed for metallization  
\cite{Hellman_Kono_Mancini_Mechanisms_Copper_Incorporation_2002}.
as well as 2 tyrosines. Independent errors in metallization
will multiple and reduce likelyhood of functional enzyme.  

The blood leveels of ceruloplasmin may correlate well
with copper blood levels but it can vary in quality too. 
A chain  of W and Y  are thought important
for enzyme preservation  with ceruloplasmin containing a chain of
two tyrosines \cite{Tian_Jones_Solomon_Role_Tyrosine_2020}.
As iron accumulation is related to AD, there is a question about
the quality of the circulating ceruloplasmin.  If there is high-infidelity
translation due to W and Y depletion, there is also the question of
how feedback mechanisms control the overall amount.
First is has to metalize right which requires 6 coppers and
6 histidines and then survive with the 2 tyrosines. 
If any components are deficient quantity or quality may suffer. 
Ceruloplasmin KO mice gained weight and showed increased
scatter in weight with lipid dysregulation only partially corrected
wth exogenous replacement
\cite{Raia_Conti_Zanardi_Ceruloplasmin_Deficient_Mice_Show_2023}
sugesting tight control may matter. 


} % mjmcp 
\newcommand{\mjmmacro}{

While well known for a role in infection control,
macrophages also have a complicated role in cardiovascular
disease
\cite{Chen_Zhang_Tang_Macrophages_cardiovascular_diseases_2024}
although the impact of copper may be less apparent.

The importance of copper for host-pathogen relations is 
well known  as is a prticular usage by macrophages
\cite{10.1039/c4mt00327f}
\cite{PMC5535265}. Note that nutritional immunity 
may involve either an excess or limittation of copper to the
patogen hinting at a common source of myths about
which direction to move a quantity to achieve a given result
 ( see a list in \mjmrefapp{errors} )
.
Much is known about interaactions of copper in macrophages with
particular  organisms such as Mycobacterium tuberculosis
and Salmonella typhimurium 
\cite{PMC3340201}
as well as Streptococcus pneumoniae 
\cite{ACamilli_Role_Copper_Efflux_2015}
.
As with most other sites, copper in macrophages can be
considered pathological and attempts may be made
to limit rather than enhance it.
For example, targeting mitochondrial copper with
"rationally" designed metformin dimers 
\cite{Solier_Muller_Caneque_druggable_copper_signalling_2023}.
As with attempts to control amyloid beta in AD, there may be some
passing benefits but taking all the information together
this is likely of limited value if the simpler approach of
enabling the copper based response in a well regulated way
is feasible. 
} % mjmmacro 

\newcommand{\mjmlig}{

Copper can of course form associations, with many chemical
groups and diverse roles.  Role confusion however seems
to be a recurring issue as specific componunds may not
sequester or activate copper as may be expected. 
One recent review on anti-cancer mechanisms of copper dicussed
various ligands and roles for both copper depletion
and overload
\cite{Babak_Ahn_Modulation_Intracellular_Copper_2021}.
The combination
of the garlic and copper, while sounding like a medieval concoction
( suggested as a result from Bald's Leechbook \cite{PMC4542191} )
,
has been described as synergistic against fungus \cite{PMID16102883}
and seems plausible for generating perhaps more volatile and diffusable
copper that could find otherwise inaccessible lysyl
oxidase and other targets.

Dogs fed a histidine deficienct diet eventually developed
feeding resistance and lower whole blood copper and zinc
\cite{Cianciaruso_Jones_Kopple_Histidine_Essential_Amino_1981}.


} % mjmlig 

\newcommand{\mjmdogpic}[1]
{
\includegraphics[height=2in,width=2in]{keep/#1} 
}
\newcommand{\mjmdogsone}
{
\begin{figure}[H]
\mjmdogpic{happycu.jpeg}
\mjmdogpic{rockycu.jpeg}
\mjmdogpic{annie.jpeg}

\mjmdogpic{mixiepic.jpeg}
\mjmdogpic{trixiepic.jpeg}
\mjmdogpic{browniepic.jpeg}
\caption{Some of the dogs described from left to right then top to bottom: Happy, Rocky, Annie,Mixie,Trixie,Brownie.  As breed may be important but unknown visual inspection may be helpful for guessing about genetic background.  }
\label{fig:dogsone}
\end{figure}
} % mjmpicture





\begin{document}

\title{\mjmtitle}
\author         {Mike Marchywka}
\email          {\mjmemail}
\thanks{ to cite  or credit this work, see bibtex in \ref{appendix:citing} } 
\date{\today}
\affiliation{\mjmaddr}

\mjmblackboxno{Release Notes  xxxx-xx-xx : }{
Copper was an early part of my interest in optimization of 
supplements for dogs and humans. Recent literature has expressed
concern about copper so I thought I would get out generally
supportive results to date although omitting much of my own
personal experiences ( I'm a human not a dog ) that seem similarly
beneficial. It seems that often the popular press led by science
catches onto incomplete or "close but not quite" ideas and reversals
in recommendations are common. Curious to see how attitudes towards
copper evolve. It may be worth noting there seems to be a trend
to get away from copper plumbing lol.  
Actually looking at the old, unpublished work, "casesum",
that includes Little Man, most of the text is still
useful today and has been copied and pasted without attribution
( since it was a never-published work I authored ).

A lot of introductory material may be more commonly put into
a discussion of later section but its important to motivate the
rather simple work of desining a set of supplements for a group
of dogs. The data re of course ambiguous and the discussion tries
to tie the introduction to the new data without a lot of new citations. 

{\bf ToDo : } Known problems: no refs yet, diettables have unit problems for recent
noun additions


\mjmreleasewarning

\mjmstatuswarn

\mjmwarnfeed

\mjmwarnme

\mjmwarntopic

\mjmexplainbib




}




\begin{abstract}
The known roles of copper suggest it is needed to perform
functions that could mitigate several common diseases
yet it  is not currently a trendy supplement for dogs or
humans. Part of the concern with dogs is the variable
genetics and observations of copper accumulation in the liver
becoming more common. However, with most genetics lacking
a recognized  copper storage disease, copper distribution is
regulated by a complex system that appears to consider
locale based demand as part of the uptake and excretion control 
algorithm.  Such a system may produce bottlenecks leading to
uncommonly large accumulation in an organ such as the liver
while another organ such as the heart is starved.
Surch a situation could occur due to some other nutrient
limitation that fools the feedback mechanisms.
Without knowing the specific bottleneck, the decision
to supplement would be based on the overall benefits
to the starving location versus any harms to the 
accumulating organ. 
This work describes copper supplementation to a group
of dogs without obvious significant harm while 
in some cases coinciding with benefits such as
increased energy or reduced coughing. 
Supplementation in the range of .3-1 mg/kgBW/day was most
commonly explored  with .1 thought to be too low and  3
being an old literature NOAEL. 
The "background diet"
is discussed in terms of making extra copper intake useful
to the hosts. While copper interacts with just about everything,
specific interactions with amino acids, B-6, and zinc are
considered in more detail.  
The overall diet may help mitigate two contemporary
issues, hepatic copper accumulation and diet related
DCM, in dogs as well as highlight rrecurring issues
with hidden assumptions and logical fallacies when dealing
with non-obvious regulatory feedback systems. Hopefully
this workd leads to a starting point for 
a diet-related disease mitigation strategy and helps
turn paradoxes into paradigms be elucidating some
feedback mechanisms that may be quite common. 
 
\mjmtolx{
Two contemporary diet related dog health issues,
, hepatitic copper accumulation
and dilated cardiomyopathy, may be symptoms
of the same set of nutrient deficiencies with the latter
reflecting cardiac copper deficiency.
The inability to attack each is probably due to 
unexamined assumptions about cause and effect leading
from diet to clinican outcomes. 
In particular, homeostatic mechanisms 
and roles %  ( sequestering vs activating for example )
may need more attention.  Role may come up
in cause and effect but also in action such as sequestering
or depleting versus activating. This type of role confusion
comes up in the copper literature.  
Local nutrient related feedback signals
may cause more copper uptake but be unable to 
deliver it to the heart where it is needed.
Consideration of homeostatic mechanisms,
feedback loops, is often neglected and that may be the case
in the present condundrum.  
This work describes copper supplementation to a group of dogs
over several years with no robust deletious effect established
although some suspicious observations are discussed.
Bnefits associated with copper supplementation,
in the context of more complete supplementation, include 
reduced coughing likely due to infectious
respiratory disease, collapsed trachea,, or dilated / hypertrophic
heart. 
Copper in these dogs may be beneficial through accumulation
in macrophages and other locations, use by lysyl oxidase to
stiffen trachea and other structural organs, and for
mitochondrial energy production notably by the heart leading to greater
volumetric efficiency and reduced size. 
Particular nutrients that may aid transport out of the liver
would likely be those which enhance ceruloplasmin quantity or quality
or othewise modify copper handling. Tryptophan and tyrosine are
two candidates for limiting copper toxicity  and both were generally
supplemented in this group of dogs. Both amino acids have
unique functions and distribution may be modified by many
factors from GI health to overall food chemistry and
microbial metabolism.  Some works implicating copper
quantity in pathogenesis of these or other conditions
is examined for hidden assumptions. 
Of particular note may be that the mitochondrial copper pools
are regulated by oxidized glutathione suggesting
the possibility that antioxidant overdose could
damage this signal leading to insufficient energy
output.  Another is mistaking adequate blood copper levels as a sign
of overall sufficiency rather than a sacrificial accute phase
response. 
This work may be useful in mitigating these two unrelated
issues. 
}

\end{abstract}

\maketitle
\tableofcontents
\newpage
\newcommand{\mjmguide}[1]{ \subsection{#1}}

\section{Introduction  }
\mjmguide{ Copper situation is confusing and deficiency often ignored by doctors}
Copper is  currently not a trendy supplement
for humans or dogs.
This may be due to  deficiency being  a neglected  diagnosis  by
physicians 
\cite{PMC10733163}
\cite{Fong_Vij_Vijayan_Copper_deficiency_important_2007}
\cite{Wahab_Mushtaq_Borak_Zincinduced_copper_deficiency_sideroblastic_2020}
\cite{Urtiaga_Terrero_Malumbres_Myelopathy_secondary_copper_2018}
\cite{Goez_Jacob_Fealey_Unusual_Presentation_2011}
\cite{Abdolmohammadi_Sharma_Copper_Deficiency_Forgotten_2011}
.
In dogs however a specific issue with liver copper accumulation has
been noted. 
An expert consensus statement on hepatitis in dogs suggested 
an increased incidence  that conincided with a change in copper supplementation
premix in commercial dog foods \cite{PMC6524396} pointing
to increased intake as a cause. 
However,  known and expected functions
make it a good   match to several possible problems 
motivating an interestt in supplementation  even if contrary
indicators also exist.  
In fact, isolated authors do sometimes  point to it as a major
cause of diseases such as  heart disease
\cite{DiNicolantonioe000784}
\cite{PMC8838622}
.
Copper appears to be of recognized importance in dogs
but 
The question of dietary deficiency or excess  
has been a topic of controversy for many years and likely
copper intake alone is not the dominant issue but rather
distribution in the body deterined by other dietary components
as well as genetics.
The question of supplementation has to be anwered within the context
of a specific overall diet  as well as recognized  factors like genetics.
If an unusually high concentration if measured  in one location, the question
remains if there  would be 
a net benefit reducing deficiency in some other place even if 
more local excess occurs.  
Thsis has to be answered in terms of clinical outcomes,
things relevant to the host, rather than lab values.  
Even in the case of known genetic diseases,
copper removal to improve liver values may create clinical
dopper deficiency. 
Treatment emergent copper deficiency has been observed in 
a Bedlington Terrier with an excretion defect 
that clinically resolved with cessation of chelation 
\cite{Seguin_Bunch_Iatrogenic_copper_deficiency_associated_2001}.

\mjmguide{ In many cases confusion may be due to interaction with overall diet }
Ideally the optimal fate of the attention getting copper hoard
could be identified and manipulated somewhat with dietary components.
This work includes consideration of 
a background dietary context including real foods
and controlled supplements  that  has been described
previously
\cite{marchywka-MJM-2021-018-0.50rg}
.
The copper supplementations described here, about .3mg/kgBW/day ,  is likely
well in excess of background levels in most known natural diets  
although its also worth noting that other sources may 
be accidentally significant as with water and air exposure 
\cite{Georgopoulos_Wang_Georgopoulos_Assessment_human_exposure_2006}.
A 2016 study in mice suggested adding copper to water was
"worse" than adding it to food and supplementation at 6,15, and 30 ppm
with increases in soluble abeta and decreased growth rate and
GSH/SOD activity
\cite{Wu_Han_Gong_effect_copper_2016}.
With a high dose of about 100 micrgograms/day ( from CuSO4) and
a body weight of about 30grams, the dose was about 3.3mg/kg or
about 10 times higher than considered here.

\mjmguide{ Copper excess observed in dogs misleading  }

One increasingly common concern with dogs is copper associated
hepatitis
\cite{10.1093/tas/txad147}
\cite{PMC10787350}
\cite{Center_Richter_Twedt_time_2021}.
Cppper handling genetic diseases  are well known in dogs 
\cite{PMC6531654} and some concern  about excess intake may be warranted
for specific animals. 
Some noisy association between intake and hepatic copper
content was shown in a small study
\cite{Fieten_HooijerNouwens_Biourge_Association_Dietary_Copper_2012}
but intake  was probably not the major determinant of copper content
.
The causal role of copper in any clinical disease remains unclear. 
One study produced a histogram of liver Cu content (ppm dry weight )
for a few hundred dogs but did not conlucde there was a particular
cutoff level for health vs disease
\cite{Thornburg_Rottinghaus_McGowan_Hepatic_Copper_Concentrations_1990}.
Work continues to focus on concepts of deficiency or
excess even though both may be matters of degree, location,  and context. 
 Deficiency   in commercial dog foods was    suspected years
earlier  \cite{HuberTLa}
while works as early as 2000 suggested supplementation
would be unhelpful\cite{doi:10.1177/104063870001200201}.
A recent work comparing mineral content in groups of commericial
dog foods found some patterns and advised against high copper
foods  
\cite{KUx0119badchar_Biel_Witkowicz_Mineral_heavy_metal_2023}.
Recognized low copper status is  known in humans
and may be related to dietary issues such as  
high fructose intake \cite{PMID7825519}. 
Copper deficient liver patients 
may be  notable for
"steatohepatitis, iron overload, malnutrition, and recurrent infections "
\cite{Yu_Liou_Biggins_Copper_Deficiency_Liver_2019}.

Ideally observation of a pathological copper 
buildup in one location with symptoms of deficiency
in other systems would lead to a search for 
the bottleneck prohibiting a better allocation which may
include excretion or controlled uptake.
With copper this is quite complicated. 
A number of regulatory systems are known to control
copper levels in different compartments  in healthy mammals.
Uptake regulation is not completely known but response to
dietary or ambient levels is only part of the control
loop in vivo and likely cytosolic sensors exist \cite{PMC6365104}.

Copper related signalling is such that remote signals may exist from the heart
to liver and intestines to make more available
\cite{Kim_Turski_Nose_Cardiac_Copper_Deficiency_Activates_2010}
\cite{xxx_Mechanism_regulation_2010}. In this scenario, local
shortage could induce blood stream or liver excess due to added
uptake with frustrated  cardiac specific uptake.
A similarly confusing  feedback scheme 
has been suggested for other nutrients such as tryptophan
\cite{mmarchywka-MJM-2021-007-.1-table-rg}
and biotin
\cite{marchywka-MJM-2022-010-0.80}
\cite{marchywka-MJM-2021-015-0.50-rg}
.  Feedback systems like genetics are likely 
selected by evolution but that need not make them simple
or  completely robust to defects in the control loop.
One recurring source of defects to consider is  high-infidelity
translation or a specific pattern of defects likely related
to the charge state of tRNA   for paricularly problematic
amino acids.   
Motivation for the current work comes from observations on
conditional KO or mutational diseases relevant to copper handling. 
For example, copper transporter CTR1 , important for
intestinal uptake  
\cite{Nose_Kim_Thiele_Ctr1_drives_intestinal_copper_2006}
and distribution  
\cite{Lee_Prohaska_Thiele_Essential_role_mammalian_2001}
\cite{PMC9759326}
\cite{PMC34439}
has  reduced function  if some
histidines are mutated in either the C or N terminal tails. 
Intestinal loss of CTR1 causes cardiac hypertrophy
among other issues that are partially correctable with copper supplementation
\cite{Nose_Kim_Thiele_Ctr1_drives_intestinal_copper_2006}
This suggests cardiac copper deficiency, possibly due to histidine 
deficiency,  as an issue for cardiac hypertrophy 
in animals but does not exclude simultaneous excess in the liver.
%"Jigh-infidelity transcription" or the synthesis of proteins defective at specific amino acid codons largely due to lack of a required acid, such as histidine,  and appropriately charged tRNA 

Even with excessive intake, in most genetics
 biliary excretion increases when pathological
amounts of copper begin to accumulate
\cite{Chen_Min_Wang_Copper_homeostasis_cuproptosis_2022}
\cite{Hamza_Gitlin_Hepatic_Copper_Transport_2013} pointing
to defects in excretion as another possible problem.
However, many cuase and effect relationships or
control paths exist allowing other factors to control
local copper concentrations.  
Senescent cells may accumulate copper in the absence
of autophagy
\cite{Masaldan_Clatworthy_Gamell_Copper_accumulation_senescent_2018}.
Copper elevation may commit 
cord-blood derived cells  to differentitaion \cite{PMID11849228}
and it may be regulat4ed during myogenic differntiation
\cite{PMC5824686} raisng the possibility that accumulation
is due to confused signalling. 
The major copper transporting protein ceruloplasmin is an 
acute phase protein 
and blood levels may be associated with pathological conditions
\cite{PMID15668644} but a protective role is generally recognized.
As a major transporter of copper out of the liver, 
stresses then could export copper  out of the  liver.
In fact, as substantiated later, it would make sense
for copper to be redeployed from some mitrochondria to the blood
in response to pathogens or trauma. This will lead to hydrogen
peroxide formation and potentially anti-pathogen copper both conditions
likely to control infection. 
Emerging mechanisms such as extracellular veiscles 
\cite{Bellingham_Guo_Hill_secret_life_2015} suggest
that uncharacterized mechanisms of metal homeostasis
exist.


\mjmtolx{
Other dietary components can impact uptake, signalling,  and 
distribution in many ways. 
The dietary and genetic factors that influence copper 
distribution  and health outcomes are complex and probably not fully known. 
While genetics are highly variable in dogs, some other factors
include many dietary components, GI health,  and signalling stimuli. 

}
%%%%%%%%%%%%%%%%%%%%%%%%%%%%%%%%%%%%%%%%%%%%%%%%%%%%%%%%%%%%%

% \\\\\\\\\\\\\\\\\\\\\\\
\mjmguide{ Copper status is ambiguous from biochemisty and clinical outcomes   }
There is an attempt to distinguish symptoms or biomarkers  from disease although the
two may themselves be ambiguous. The main diagnostic here is coughing.
This is certainly relevant to the host and would hvae to be considered
a clinical outcome. However, the underlying diseases that provokes the
cough may not monotonically correspond to coughins features.  
A following section of this work will describe coughing characteristics
as a clinical outcome associated with copper supplementation.
Careful consideration of the diseases and locations in the following
section
sugests that disease and this symptom
could decouple with copper supplementation. That is, coughing may 
depend on arousal state
of the host and copper may increase overall energy levels and
coughing frequency.  However, at least one pathological state
may be improved with anticipation that another pathology will
recover but more slowly leading to coughing reduction. 



\mjmguide{ Copper related outcomes depend on everything starting with GI tract   }

One problem making reproducible statements about 
copper supplementation may be the interaction
of copper compounds with  almost everything else.
Literature exists on some intreractions but the versaility
of copper probably makes a copmlete list intractable.
Some of the 'usual suspects" are listed in \mjmrefapp{interactions}
merely to indicate a lot of what is left out of the current analysis.

Thie confusion begins in the GI tract with chemisty ranging from 
the problems sorting and regulating metals  to the
non-enzymatic reactions copper can undergo with a variety
of organics in food. Once in the body, the ability to 
distribute and use copper depends on other nutrients
maybe more than is the case with other vitamins or minerals. 

As many conditions linked to copper may be considered age
related,  quality of the GI tract as a function of age may be 
relevant. 
GI health and in particular stomach acidity may be important
facotrs in copper uptake but also distribution if other
nutrient deficiencies are created. 
\mjmtolx{Copper uptake may depend on anions such as chloride
at least in some fish 
\cite{Handy_Musonda_Phillips_Mechanisms_Gastrointestinal_Copper_2000}
,suggesting GI chloride per se may require more investigation.
Content in drinking water is variable and may be perceived
by humans depending on factors like pH \cite{Hong_Duncan_Dietrich_Effect_copper_speciation_2010} sugesting it may be perceived by dogs too.
}
Copper solubility is pH dependent \cite{10.1093/chemse/bjl010}
similar to the competing element zinc for which absorption
 has been shown to depend on salt type and
gastric pH \cite{PMID8577018}.
Interaction with food components such as polyphenols is significant
and pH dependent 
\cite{PMC3401972} motivating a larger interest in food interactions
and in particular rings such as in tyrosine. 
Speciation gradients
may be large in the range of possible stomach acid levels.  
A 2021 study did in fact explore copper speciation in simulated
gastic juices with food components such as 
tyrosine and citric acid among others \cite{PMC8441336}.
Impact of GI pH on broiler chicks has been studied due to impact
on nutrition and micrbial populations 
and Cu-Zn antagonism in the digestive system was also observed
\cite{Pang_Applegate_Effects_Dietary_Copper_2007}.

% \mjmtol{ put somewhere }
In humans, PPI usage has become common. Empirically thate
may be a tumor protective effect and  there is a suggestion
that  pH 6 encourages cancer progression
versus pH 8
\cite{PMC7085403}
yet alkaline stomach pH is observed is commonly observed
in gastric carcinoma
\cite{10.1007/978-4-431-68246-2_26}.
Probably the dominant effect on tumors is unrelated to ambient
pH although increased pH may reduce nutrient
accumulation by many cells. An absolute apoptosis rate
at near neutral pH is probably not indicative of the
overall fitness in the stomach. 

within and beyond the GI tract, a variety of other
factors likely matter. 
Competition between Cu, Fe, and Zn was observed in Caco-2
cells \cite{ARREDONDO_MARTINEZ_NUNEZ_Inhibition_iron_2006}.
Iron intake in feed has also been observed to decrease
copper uptake in ruminants 
\cite{Clarkson_Paine_Kendall_Evaluation_solubility_2021} 
and rats
\cite{Lee_Ha_Collins_Dietary_Iron_Intake_2021}
.
One work suggested  iron disrupts copper homeosstasis
independent of uptate
\cite{Ha_Doguer_Collins_Consumption_high_2017}.
Dietary cholesterol appears to distrurb copper homeostasis 
with atherosclerosis thought to involve copper dysregulation
\cite{PMID33661473}
and high fat diets in other species \cite{PMC5167165} are an issue.
Fructose also inhibits relative copper absorption
\cite{ODell_Fructose_mineral_metabolism_1993}.
Zinc is a known inhibitor of cuppoer uptake
and cases of zinc induced copper deficiency in humans
are known\cite{PMC7495772}.
While concerns exist about copper content in dog foods, a
recent survey of some zinc content shous many foods 
contain amount above recommended maxima and few are
low or deficient \cite{PMC8066201}.
Age related absorption problems in people
are known \cite{PMC5133110} and other apparent deficiencies 
could be a consequence of
insufficient B-6 alone \cite{PMID7814236}.
Interestinly, B-6 is added to penicillamine treatment of
Wilson's disease to avoid neurological effects
\cite{PMC3526418}.
\mjmtol{
Since paradoxes are a concern here, its worth noting that pyridoxine
is an inhibitor of PLP and high dosese result in functional B-6
deficiency \cite{Vrolijk_Opperhuizen_Jansen_vitamin_paradox_2017}.
}
% sled dog exercize comment omitted for no good reason to include
Body stores of copper increase with excess tyrisone in the diet
of rats
\cite{Yang_Noda_Kato_Elevated_Intestinal_Absorption_}.
Some reports show specific issues when combinted with vitamin C.
A small trial with copper sulfate indicated kidney problems
result 
\cite{Jiang_Sui_Hong_Combined_Administration_2023}
although alternatives with copper gluconate 
suggested use as food preservative
\cite{Graf_Copper_Ascorbate_1994}
. Dose probably matter among other factors.  


\mjmguide{ This kind of confusion is common for recurring reasons   }
This work describes inclusion of copper into a set
of supplements for dogs with different conditions and unknown genetics 
showing generally beneficial results  or at least
no obvious harm with added copper.
This result is in contrast to some of the popular expactations
cited above.
A descrepancy between expectations and outcome 
of this type seems to be common in medicine
even as late as clinical trials 
o understanding the causes of that
in this partciular case may help optimize
dog supplements and avoid delays in understanding the
limitations of data and theory  to design 
therapeutic interventions.  
Interpretation of the copper literature may be  limited by 
unqestioned assumptions and logidcal fallacies 
that are quite common.  Some types of problems
are listed in  \mjmrefapp{errors} as a general giuide. 
A 2010 work suggested that high copper and iron intake were
particularly dangerous in older people observing that
one study concluded high intake of both was assoicated with increased cognitive
decline 
\cite{Brewer_Risks_Copper_2010}.

%%%%%%%%%%%%%%%%%%%%%%%%%%%%%%%%%%%%%%%%%%%


Part of the reluctance to supplement "high" copper doses
is the accumulation in the livers of some dogs but
relation to any clinical disease is not clear. This
may be similar to amyloid beta in Alzheimer's
Disease and pointing to the need to understand cause
and effect before an all out attack on one molecular
entity.  

A recent example may be the identification of amyloid
beta as a nominally protective substance
\cite{Yu_Wu_Amyloid_upbeta_double_2021}
\cite{PMID35673950}
 instead of
the cause of Alzheimer's pathology and target
for intervention. 
It may still be possible to optimize the amount creating
passing benefits that are inherently limited and difficult
to control. 
That state of affairs is well documented
in the works
that hint at it unravelling\cite{PMID37833948}  such as a 2002 work
suggesting that  "tauists" and "baptists" could 'shake hands"
and look for other causes \cite{PMID11801334}.
Interestingly, related to copper,
is the emerging role of lysyl oxidase in Alzheimer's
as a possible target 
where it associates with cerebral amyloid angiopathy and
is thought to be a drug target 
\cite{Kelly_Sharp_Thomas_Targeting_lysyl_oxidase_2023}
\cite{PMC11042178}. However,  upregulation would have to be
suspected as a part of regeneration attempting to fix
degeneration. 
Previously, heartworm positive dogs had been given
significant amounts of vitamin K
\cite{mmarchywka-MJM-2021-003-v0.50rg}
\cite{mmarchywka-MJM-2019-001-.1li}
 although severe
case may be treated with anitcoagulants. In thise
case, it may not be clear that vitamin K effects clot
quality and consequently may limite pathological quantities
allowing for beneficial clots to form. 


\mjmguide{ Coppr and blood pressure   }
Blood peressure, which is not a clinical endpoint per se ,
has been associated with coppers status. While it is not 
central to this work, it illustrates some of the issues
introduced above and is good context for the current
work on clinical diseases.

A 2021 work exploring the association between serum copper
levels and hypertension in US children and adolescents
stated
\cite{Liu_Liao_Zhu_association_between_serum_2021}
"
As suggested by previous literature, serum copper appears to reflect the status of copper nutrition in both depleted and replete populations. "
The present work however tends to accept that ceruloplasmin and
blood copper levels are a reaction to stresses such as infection
and a high level simply indicates some other pathological process
likely exists.  While blood levels can be elevated in response to
a stress, its not known how well other functions 
such as electron transport are able to continue. 

As erly as 1993, one work with Dahl salt-sensitive rats
suggested that indeed copper blood levels were a response
to hypertension 
\cite{Garrow_Clegg_Metzler_Influence_hypertension_1991}.

Things that do matter to the host  related to blood pressure include
cardiac workload,
vascular pressure hhandling limits and adequate nutrient delivery.
Any of these are likely complicated functions of long term diet
details. 

\mjmguide{ Copper interactions with WHY  and clinical impact of Cu   }

The present work gives details of most of the dietary components
but analysis is largely confined to a select few such as
tryptophan, tyroine, histidine, and vitamin B-6. Its worth
noting that copper is quite reactive and can form 
coompounds with many aromatic rings, nitrogens, sulgurs, or
other common chemical groups.  



Interestingly, 3 amino acids, the ringed "WHY" trinity
( tryptophan, histidine, and tyrosine ) seem to be
the most important. Notably tyrosine protexts ceruloplasmin
and 6 histidines, one for each copper, are required for
a functional enzyme. Mistranslation due to insufficiency
will be amplified byt the higher power if all need to be right. 
There is some indication that "diseases of old age" 
are at least partially medicated by sarcopenia,
most reecently atrial fibrillation
\cite{PMID38739369}
consistent with earlier ideas linking age to
amino acid staration
\cite{mmarchywka-MJM-2021-007-.1-table-rg}
.
Interestingly, Scavenger Rceptor BI contains 8 highly
conserved tryptophans
\cite{Holme_Miller_Nicholson_Tryptophan_Critical_2016}
which appear important for cholesterol transport
suggesting trp deficiency will slow down its translation
or create many imperefect receptors. 
So, deficiency of these amino acids may produce broad 
non-specific problems that appear to be common
in intractable diseases often associated with old age. 

Copper status may help unify other unresolved issues
in dog health. 
More recently, diet associated dilated cardiomyopathy (DCM)
has also become a concern.
Hypothyroidism has been a topic in human health for a while
now and has occured in several dogs here. Interestinly,
that too may be related to copper defieincy.

\mjmtol{ engineering  ahead of medical }
In some other context, engineering efforts to produce
biological products such as bone in a reacor vessel
can offer insights into factors in vivo.
A 2021 work focused on similar systems by modelling the
effect of amino acids and copper on antobody quality in 
a production setting 
\cite{PMID32875683},
\begin{quote}
{Specifically, copper has a significant, positive effect on titer and a
significant, negative effect on lactatea phenomenon consistently observed
in other CHO cell culture processes.{[}12, 13, 23, 24{]} A plausible explanation
is that the increased copper level is known to drive lactate consumption.
Copper deficiency reduces cytochrome c oxidase activity, limiting the ability
of cells to produce ATP via oxidative phosphorylation. As a result, cells
switch to aerobic glycolysis to generate ATP, causing increased lactate
production, which affects other metabolic processes.{[}12, 23{]}}
\end{quote}


Given that many diseased state can be caused by low copper,
and accumlation in some does does not equate to excessive
intake in thse dogs, possible benefits of copper were considered.



\mjmguide{ Recent Similar Work  }
At least one recent work 
\cite{Chen_Cai_Liang_Copper_homeostasis_copper_2023} has
summarized many of the same issues relating to copper
and cardiovasular disease but with better editing and depth.
However, it does not point to some of the cues outlined herein.
For example, the issue of "other" limiting nutrients determining
when copper is beneficial or not. 


\mjmguide{ Copper rich diets  for dogs and people are  worth investigation   }
This work illustrates that empirically clinical deficiency
may be more common in some groups of dogs than meaningful excess
in that supplementation, in the context of the rest of the 
diet,  improves apparent health.
If that hypothesis can be shown to be more generally
true, the coppper literature may illustrate some common
fallacies and errors common to biology related
literature and likely other genres involving complicated
systems. 
However copper intake issues are resolved, the complexity of
copper handling may make it a good topic to understand larger
issues in system identification. 

\mjmtolx{
However, the empirical data and known theory or
biological pathways don't estalish a causal role
for excessive intake of copper as a problem in most cases.
Rather, other nutrients may be lacking to properly
deploy existing copper reources and likely other
things. In the particular case of copper accumulation
in the liver, its important to note that import and
export are controlled by different things. 
While import appears to be controlled by diet and
remote signals from the heart and binding
to albumin, export may be
limited by ceruloplasmin and excretion.


This work describes in more detail the theoretical and empirical
motivation to consider copper supplementation carefully
as part of a lrger supplementation progam. 


}



The remainder of this work describes the
most interesting dogs' histories  % stories are presnted and discussed
 with variable  copper supplementation  and the  
differences between some expectations and outcomes.  


\section{  Diseases of Interest }

The known roles of copper include pathways and functions that
would seem important for many diseases that are common in dogs.
The present works relies mostly on 'cough" and thyroid related
symptoms ( coat quality, weight distribution, energy level etc).
Cough has many possible origins as discussed previously
\cite{mmarchywka-MJM-2019-001-.1rg}.  Among these are
respiratory infection or irritation, collapsed trachea, heart failure,
and CNS stimulation. Infection related cough is well known due
to direct irriation. Collapsed trachea may cause a cough on either
inhalation or exhalation with a distinct honking sound.  
Heart failure may initiate fluid build up in the lungs and
other sources of bulk such as a tumor may cause irritation. 
With these source in mind, it is not difficult to find
pathways and locations that require copper for proper functioning.
These have been tabulatedin in \mjmreftab{causes}
along with time scales for the response to reflect changes in 
copper intake. In most cases the distribution is very broad
an non-specific symptoms may be expected. For example,
mitochondria are eveywhere although the heart may be
the first to show symptoms of deficiency. Its also 
important to note that cough monitoring is not
going to be monotonic with improvement as increased
energy may occur quickly resulting in more coughing
until trachea and heart can remodel.   

\begin{table}[H] \centering
\begin{tabular}{|l|r|l|r|r|}
%\multicolumn{6}{|c|}{Title}\\
\hline
Disease & Host & Effect & time scale &  \\
\hline
DCM/HCM & dog  &  && \\ 
heart failure  &  &  &4 weeks \cite{Chen_Cai_Liang_Copper_homeostasis_copper_2023}& \\ 
Collapssed Trachea & dog  & && \\ 
airway  defects & rats && months, 60days  \cite{PMC2018314} &  \\
Hypothyroid & dog  & && \\ 
Infection & dog  & && \\ 
\mjmdisease & human  &
serum levels irrelevant
\cite{Rozemeijer_Hamer_Heijboer_Micronutrient_Status_Critically_2024}
 && \\ 
Parkinson's & human  & assoc  benefit \cite{Zeng2024} && \\ 
Alzheimer's & human  & && \\ 
Infection & human  & && \\ 
Arthritis & human  &anecdotes, no trial verfiication  && \\ 
\hline
\hline
\end{tabular}
\caption{ Some  diseases and conditions with copper involvement that
may be illustrative of less obvious issues. 
   }
\label{tab:disease}
\end{table}


\subsection{ Collapsed Trachea }

\mjmcollapse

\subsection{ DCM/HCM}

\mjmdcm

\subsection{ Infection }

\mjminfect

\subsection{ Thyroid  }

\mjmthy

\subsection{ AD, PD, ALS, Arthritis  }

\mjmmisc

\subsection{ Cancer }

\mjmcancer

\section{ Functions of Interest   }

\begin{table}[H] \centering
\begin{tabular}{|l|r|l|r|r|}
%\multicolumn{6}{|c|}{Title}\\
\hline
Location & site & Effect & time scale &  \\
\hline
Heart & mitochondria & energy production & maybe days & \\ 
Heart & mitochondria & remodelling  & weeks or months & \\ 
Heart valves & lysyl oxidase & crosslinking \cite{PMC5541772}  & weeks or months & \\ 
Trachea & lysyl oxidase & proper crosslinking   & months & \\ 
Macrophage&  & infection  & days & \\ 
ceruloplasmin&  & distribution  & days & \\ 
foreign ligand  & variable  & variable && \\ 
\hline
\hline
\end{tabular}
\caption{ Some expected benefits of copper that guided the original
interest and observations although sometimes the goals were lost
in the details of the diet and outcomes. 
   }
\label{tab:causes}
\end{table}


Copper in these dogs may be beneficial through accumulation
in macrophages and other locations, use by lysyl oxidase to
stiffen trachea and other structural organs, and for
energy production notably by the heart leading to greater
volumetric efficiency. 





\subsection{ Cytochrome C Oxidase  }

\mjmcox

\subsection{ Lysyl Oxidase  }

\mjmlox

\subsection{ Ceruloplasmin   }

\mjmcp

\subsection{ Macrophage et al   }

\mjmmacro

\subsection{ Associated Moieties - histidine and garlic etc   }

\mjmlig


%%%%%%%%%%%%%%


%%%%%%%%%%%%%% cutoff

Copper homeostasis is a much larger issue including in human
health with regard to such unresolved diseases as 
Alzheimer's where the decades of work on amyloid beta
is becoming more clearly futile. 


In humans lysyl oxidase is sometimes discussed as
a drug target as its quantity seems to increase in pathological
situations. 
This may suggest that additional copper would not likely
help. 




To help with this apparent conundrum, this work describes
variable copper supplementation to a group of dogs
over several years including one pregnant pit bull with
uterine fibroids. Generally beneficial results 
were associated with copper supplementation in the context
of broader rationally designed supplements.
Apparent benefits to a group of puppies included infection 
control. Additional respiratory infections were
thought to be modulated in older dogs described
here as Cookie( AKA Mixie ) and Trixie. It may have
reduced transmission to the larger group in the latter case.
Also an association with likely non-infectious coughing
was seen in the case of Happy. 

Given the varied canine genetics and known copper related
diseases, vigiliance for adverse reactions was maintained
but to date only questionable events, such as reduced
appetite, remain. 

Copper and vitamin K have both seen literature suggesting
a role for liver health under some conditions. Vitamin K is
note worthy because of many efforts to antagonitze
its effects similar to the present  concerns with copper. 

 
These cases are described in more detail with the hope of
sorting out cause and effect between diet and clinical
outcomes as fixation on one nutrient at  a time may not
be productive. 


The original motivation for copper was based on notions
similar to the table below which reflects current
thinking based on results presented here. 



\section{Cases and Observations}

\mjmdogsone

A series of rescue dogs were fed food and vitamin supplements
in addition to commercial kibble products. Diet and outcomes
were recorded  in MUQED format
\cite{mmarchywka-MJM-2020-004-.012rg} immediately
after feeding.
While most supplements and medicines were recorded 
in sufficient detail to reproduce, the meal or snack   
most dogs received additional meals of commercial dog food
and unfortuntately uncontrolled scraps or treats while others
routinely ate toys or yard debris. However, some 
results appear to relate to the vitamin mix and notably
inclusion of  copper. Total calorie intake is not reflected in
the data although some details of the products are indicated
in the MUQED data ( see Supplementary Information ).   
In paricular normalization of the "dinner" amounts is variable
between dogs.

Some of the discussion will also reference unpublised notes
on "Little Man" who was given copper and other 
nutrients before the MUQED system was operating.

Many of the dogs in this setting have had either symptoms
that could be due to hypothyroidism or overt lab confirmed
low thyronid hormone levels. As iodine intake flucuated 
wildly, it is included in some of the graphs as 
a putative factor confounding inferences about copper. 
The premise of this work is that the total diet is
a "confounding factor" and readers are encouraged to check
the MUQED data for other patterns. 

Generally copper dosing was rotated leading to high doses
some days with none on intervening days. 
A hard upper limit of
3mg/kg body weight of supplemental copper per day was maintained  based 
on quoted NOAEL's from an  
original source dated  1972 \cite{NAP9782}.
All of the dogs currently living here received increased copper
shortly after the arrival of Trixie due to apparent
spread of coughing. Trixie was later treated with Clavamox
leading to cure suggesting indeed an infectious cause existed. 
Some of the dogs, notably Happy and Rocky, had varying cough levels
previously as described below. 



\begin{table}[H] \centering
\begin{tabular}{|l|r|l|r|r|r|r|}
%\multicolumn{6}{|c|}{Title}\\
\hline
Dog & \multicolumn{1}{|c|}{Dates} & Condition & weight(lbs) & Cu(mg/day)&Cu(mg/kg)& Outcomes \\
\hline
Cookie &21-09-10 22-01-21&Resp infection/azithromycin &13.5&2&.33&cleared \\
Happy  &18-09-07 24-04-10&several&13.4 - 17.7&&& \\
Happy  &18-09-07 19-05-30&heartworm/doxycycline&13.4 &2&.29&cough gone \\
Happy  &24-03-26 &coughs&15.2 - 15.5&2&.29&rare coughing \\
Brownie  & 21-01-12 23-02-22& &49 - 64&1 variable&& pts due to cancer  \\
Brownie  & 21-01-12 21-02-14&pregnant, fibroids, heartworm & $\approx 60$ &1.5-2.5&& uneventful  \\
puppies  &21-03-23 21-06-09&cough&104&4.5&.095&cleared \\
Trixie  &23-12-16 24-04-10&resp infection/Clavamox&37.6 - 44.6&5&.276&cleared \\
Trixie  &24-05-15 24-06-01&deep cough returned & 44.2 &6.7& .3  & greatly reduced  \\
Rocky  &22-02-05 24-04-10&&4.4 - 8.3 &1&.37&subjectively better  \\
Hershey  &17-04-22 19-08-27&multiple&8.2 -  9 &.2-.6&.1& heart failure  \\
Hershey  &17-04-22 19-08-27&multiple&8.2 - 9 &2&.52& transient improvements  \\
LittleMan  &2016&multiple& &&.8 & honking stopped  \\
Annie  &2022-09-21 &excessive sleep, apathy &7.74- 9.9  &2 & .5 & active again  \\
\hline
\hline
\end{tabular}
\caption{List of dogs most effected by copper supplementation.
Cu amount given is largest thought to be therapeutic and in case
of Herhsy amount near death in ().
 The puppies were born on 2021-02-14 but only recorded as weaning began. Puppie weight reflects total as they were placed elsewhere and food shares are unknown   }
\label{tab:dogs}
\end{table}

\subsection{Cookie or Mixie}

\mjmmixiecu

Arrived with diagnosed respiratory infection and prescribed azithromycin.
Copper and other nutirents were added and eventually infection
resolved well. Contribution of any nutrient is unknown but recovery
seemed uneventful.

\subsection{Brownie and puppies}

\mjmbrowniecu

Brownie was the subject of a prior work 
where her uneventful pregnancy was notable
for vitamin K consumption while heartworm posivite
\cite{mmarchywka-MJM-2021-003-v0.50rg}. 
Subsequent to that work, her abdominal tumor
was removed and diagnosed as fibroid. 
Briefly, 
Brownie was determined to be pregnant shortly after arrival
. Her  heartworm
was  treated with Diroban after weaning  and fibroids
removed 2021-11-15 well after the puppies were gone. She was
unevntful until being doganosed with cancer and killed
2023-02-22. Prior to her diganosis, she was observed to 
cough occasionally for no obvious reason and appeared to
have some movement issues. X-ray confirmed tumours likely
responsible for both problems.   

Her copper supplementation was fairly minimal but given her
overall state of health and pregnancy she may be
an interesting case for context.  As the role of copper in cancer
progression is not clear, her tumors  may also be a concern. 


\subsection{Happy}

\mjmhappycu

Happy, like Brownie, was also the subject of two prior works.
Initially her heartworm recovery was described
with vitamin K and other supplements
\cite{mmarchywka-MJM-2019-001-.1li}
followed by an unusual episode which resulted in some
investigation of possible role of vitamins B-2 and B-3
in her health
\cite{marchywka-MJM-2022-009-0.50rg}
.
To summarize, Happy arrived heartworm positive coughing to varying degrees.
She was treated with a slow kill approach including ivermectin
and doxycycline as previously described. She later was acting
sick but appeared to recover well with B vitamin supplements.
Her coughing never returned to the very low levels
seen after heartworm recovery until copper doses were increased
with elimination of any zinc and care with tryptophan.
As copper was increased due to widespread coughing after Trixie's
arrival, her cough was noted to decrease to lower than prior levels.
Review of the copper dosing suggested it had fallen prior to Tricie's
arrival.
Often her excited cough appeared to be a honk on exhale suggestive of
trachea collapse. She also would cough early in the morning
when curled up. 
As outlined in the introductiory material, it was later realized 
that earlier conserns about aromatic amino acids could be
due to simple excitability rather than any disease worsening.
This most recent effort considered some coughing increase normal
and now she seems to have reduced coughing, good energy level,
and maybe some  aromatic amino acid sensitivity. 

During the increased copper interval it was noted that
her frantic "rubbing", where she tries to rub her lower
back on fucniture or rolled over on the floor,
had stopped. This was most severe years ago when it
was treated successfully with Apoquel between 2019-08-27
and 2019-11-03. After that, it had come and gone but never
appeared distressing enough to treat again. A review of the
copper and zinc dosing during the Apoquel treated spell
suggests very low copper ( maybe less than 1mg per day )
and substantial zinc ( 3mg on some days). The resolution
with either copper of Apoquel points to the possibility of
a pathogen involvement as will be discussed with Trixie
and Annie's coat problems. 

Currently
she is coughing when excited, now several days after   significant
aromatic supplementation, but her exercize capacity from
jumping during dinner to trotting in the bark is very good.
It remains to be seen if continued copper supplementation can
eliminate the cough presumed due to defects or limitations in 
trachea properties.  


\mjmtolx{
Also since starting the increased copper she does not appear to
have disruptive itching around her tail area. 
Prior rubbing episodes, treated with Apoquel, did not appear
after increasing copper. This suggests pathogenesis via 
immune response to either allergen or actual patogen 
that becomes resolving with more copper although symptoms
could be treated with Apoquel known and likely side effects
make this a cleaner solution. 
}


\subsection{Trixie}

\mjmtrixiecu

Trixie began coughing shortly after arrival and was
very low energy. Her coat was notable for a sticky
feeling that seemeed to recur after a bath.  
Many other dogs began to cough or
hack suggesting that she brought a communicable infectious
disease. Nutrient mix was modified to add more copper and
most dogs' coughing returned to normal quanity and qualitiy
although her's did not entirely resolve. Copper stopped for a couple
day ( I was gone ) and owner took he to the vet as she
began coughing more. Clavamox was prescribed and her coughing
stopped within a few days. Her energy level has improved
but she did not  run until 2024-05-02 ( 19845 ) .  
She had a renwed cough around 2024-05-22 (19865)  subsequent to
copper withdrawl on 2024-05-15 (19858). 
Cough seems deep on excitement not honk/wheeze like Happy.
Copper resumed 2024-05-27 and is greatly reduced
by 2025-06-01 . 
She appeared more energetic than when I left and walked well
although maybe still tired at end of walk. 
Her coat now feels clean. 

\subsection{Rocky}

\mjmrockycu

\mjmrockyi

Rocky will hopefully be the subject of another work as
he responded significantly to iodine and sodium benzoate
which was attributed to, but never lab confirmed, low
thyroid output. His "plastic" body type changed into
a more normal "flexible" type and he began to feel like 
the other dogs when picked up rather than stiff.
The addition of copper may have reduced his morning cough
but he continued to have apparent congestion after eating
sometimes breathing through his mouth and sneezing.
That too has cleared up by 2024-06-01.
Most recently he had notable muscle tone which had been lacking.
His overall activity increased but that may be due to social factors
such as feeding ritual. 
Probably most coughing now coincides with exposure to cigarette smoke.

As benzoate may be an important coundounding factor,
a brief digression here may be helpful. 
A recent study on rats fed benzoate demonstrated
some insignificant indication of increased T4 and decreased
TSH which the authors summarize as,
\cite{TURNBULL2021104897}
,
\begin{quote}
Minor variations in T4 and TSH levels were not considered treatment-related because they were not noted in a dose responsive manner, were not generally statistically significant, or were observed in a direction that would be generally not be considered toxicologically relevant. These minor variations also fell within the range of levels noted for historical controls."
\end{quote}

If there is a beneficial effect, it likely depends on many factors
such as overall diet and may vary across species.


\subsection{Annie}

\mjmanniecu

Annie probably showed the clearest improvement
coincident with increased copper ( and a few other components
described below). 
Annie arrived in generally good condition although 
seemed old for stated age of 6 years. She generally ate well
but had some limitations in sight and hearing. She
seemed to get skin or paw irritations easily.
She began to sleep excessively in early 2024 
and a  special copper snack was started on 2024-05-01 ( 19844).
This consisted of 2mg of copper along with sodium benzoate, KCl,
and B-6 in chicken broth which was more readily accepted than the 
full snacks for everyone else. 
As these other components had been quite common they were not
considered significant although the sodium benzoate had also been 
stopped earlier. 
Lost vigor suddently regained in a few days
and seemed restored to her more typical self by
 2024-05-05 ( 10948 ) with continuing improvements in 
energy and appetite.
 % of 2mg cu. 
Her exercise interest improved and she walked and cried
with good energy. 
%Panting up hill at park but quite active again. 
%Started special copper snack on 2024-05-01 ( 19844)

\subsection{Hershey}

\mjmhersheycu


\begin{table}[H] \centering
\begin{tabular}{|c|r|p{4in}|}
%\multicolumn{6}{|c|}{Title}\\
\hline
Date & Day number & Comment   \\
\hline
2017-09-25 & &developed skin problem, vet prescribed clavamox and miconazole chlorhexidine shampoo Malaseb   \\
2017-10-13 & & blotches mostly gone yesterday  Barb still notes some  \\
2017-11-01 & &stumbled down steps did not come up until after PMSNACK restart lipoicacid   \\
2017-11-13 & &struggles up deck steps but finally made it \\
2017-12-12 & 17512 & seems to be coughing a lot \\
2018-03-06 & 17596 & seems to cough less, continue copper \\
2018-04-19 & 17640 & came up steps on own again cadence sounded good \\
2018-04-20 & & fur seems thicker except for small area on back behind neck. Still coughing though \\
2018-04-28 & 17649 &appears alert more flexible and good up steps while still planning although he did stumble the other day  \\
2018-04-29 & 17650 & seems ok on steps, hair filling in.  \\
2018-05-29 & 17680 & not coughing much and energetic but refused to eat and diarrhea. Ate small amount indicated around 830AM. He seems ok at noon not coughing much but subdued.    \\
2018-07-02 & 17714 &lighter and not coughing except when really agitated. Made it up steps good. Could be just weight although not that much lost, something in yard wiped out with spraying, or something like potassium chlorde or the lysine making him worse  \\
2018-07-15 & 17727 & had rear leg problem, Barb gave rimadyl \\
2018-09-04 & 17778 &seems to be stumbling more on steps last day or 2 but yesterday later came up good with fish at top ...  \\
2018-09-19 & 17793 & left rear leg bad had to help up steps still limping in kitchen 5 mintues of so. Gave some b7ngnnc rested ok. Made it up steps after PMSNACK ok although aborted on attempt but it is 95F out.   \\
2018-10-12 & 17816 &planning and cirling on bottom step then doing good up steps  \\
2018-11-02 & 17837 & walked up first few steps in the rain and then faster up last few no slipping. Rear left leg may be more useful now.  \\
2018-11-26 & 17861 & better again on steps walking up but coughing still  \\
2018-12-04 & 17869 &coughing a lot again try stopping Cu for day or two  \\
2018-12-17 & 17882 &probably made it up steps on own, saw he was gone then heard barking and clumsy step noise. Seems good still coughing on and off, went around shed today.  \\
2018-12-27 & 17892 &coughing less and went out to pee, maybe the extra copper yesterday helped quickly  \\
2019-01-08 & 17904 & seems generally more active maybe coughing less for all the barking with the other 2 BCAA's \\
2019-03-12 & 17967 & pretty good up steps almost back to recent bests. Coughing like always but darted out the door to deck quite well went around shed etc.  \\
2019-03-14 & 17969 &leapt up some steps then stumbled near top, went around near side of shed ok and wandered yard for a while  \\
2019-04-02 & 17988 &good on steps leaping not slipping  \\
2019-04-26 & 18012 &came up steps without crying on his own.  \\
2019-05-13 & 18029 & Hershey slower than yesterday more normal  \\
2019-06-03 & 18050 &Vet found heart failure on X-ray and bladder stones.   \\
\hline
\hline
\end{tabular}
\caption{ An abbreviated set of note on Herhsey. Increased coughing may have occured due to increased excitability and energy prior to heart remodelling and 
may have been mistaken as a sign of pathology rather than recovery leading
to some confusion. 
   }
\label{tab:hershey}
\end{table}

Hershey quickly demonstrated several problems not long after arrival.
He had problems with his fur, digestion, and coughing while ultimately
being diagnosed with heart failure and bladder stones.
During his time here, his diet was varied and he was observed
for overall eating and bahvior with specific interest in 
coughing and ability to make it up a short flight of steps
from the backyard to the deck. Again, the notes were not sufficient
to fully captures the dynamics of his condition but some representative
ones have been edited into the above list. Some correspondence with
copper intake is noted. In the last few days of his life, he
would pass out and quickly regain consciousness until one day 
he did not recover presumably due to heart failure. 
As many initial features could be rationalized as related to thyroid output,
his iodine intake was elevated. 
In tretropsect, his initial response to increased copper may have been
to be more energetic but that increased his coughing leading
to a dosage reduction. Clearly the data are incomplete but suggestive
now of some benefits. 


\mjmtolx{
\subsection{Miscellaneous Observations}
Rocky seemed to do better but had  also been responding
to benzoate and iodine. 

In initial attempts to formulate a vitamin mix,
copper was added but no zinc. There was some possible
feeding hesitance that went away when zinc was added.
However a causal link was not established although 
copper was moderated afterwards. Annie may lose
some appetite with excessive copper. However, a
causal link was not eastablished. 

In a quick review of blood tests, 
chloride tended to be low and bicarbonate high.
}

\section{Discussion }


Copper supplementation averaging around .2 to as high 
as .8 mg/kgBW/day appeared beneficial in this group of
dogs as symptom improvement often followed weeks of dosing.
Several suspicious observations were noted but nothing
robustly correlated with copper intake. 
As some symptoms such as coughing are the product
of many factors, interpretation is difficult during the 
initial response phase. Most notably, increased general
energy may increase coughing in the presence of a weak
or collapsed trachea or with some heart conditions. 
Initial response would be to increased energy production
and persumably a feeling of wellbeing but remodelling
and stiffening 
may take longer to  achieve  more complete  
resolution. Another difficulty is  due to pulsed dosing.
This was chosen to allow rotation of various
nutrients and make it easier to see short term effects
of comparatively high doses. However, it did require 
that time series be averaged  for analysis.
Happy's copper intake dropped much lower than normal
preceeding a period of excessive coughing. It was only
after copper was started due to Trixie's  respiratory infection
that the data were examined more carefully and the pattern
noted. Ideally the filtering would be a lagging average
weighted to some suspected biological parameters but
even uniform moving average was useful.   

The overall concern with Happy and aromatic amino acids
ist still not clear. Increased energy may lead to more
coughing but 
pulmonary hypertension may be controlled by serotonin
\cite{PMID27927914} and therefore tryptophan intake. 

Happy's prior rubbing episodes, initially treated with Apoquel but 
subsequently just "waited out" , did not appear
after increasing copper even though she had been doing that on and
off recently. This "allergy" could be  pathogenic via 
immune response to either allergen or actual patogen 
that becomes resolving with more copper although symptoms
could be treated with Apoquel known and likely side effects
make this a cleaner solution. 

Reconciliation with various concerns and failures
relies on the complex interaction with other dietary componets
which may be important over a large range of genetics. 
Copper was just one of many nutrients explored and
it requires some of these for proper handling.
Any optimization strategy would need to conintually
be finding the performance limiting nutrient
in turn as each is tweaked. 


Hypothyroidism is common but may be due to a nutrient
deficiency suzh as iodine or less obviously copper.
Some of the symptoms associated with hypothyroidism
could be explained by copper deficiency alone.
The role of sodium benzoate may be important but
difficult to characterize  but may be a topic of a later
work on Rocky.

Copper was considered to be important early but despite
some successes was lost in the shuffle. While the conditions
for benehficial usage remain unknown, several suspects
can be identified. Histidine addition was the most
obviously correlated with increased
alertness, animation, and aggression although there
may have been some expectation bias considering increased
histamine contributint to these states. These responses
could increase coughing leading to the erroneous
conclusions it was making something worse.    


Several likely benefits of copper suypplementation were observed
but no clear robust clinical symptoms got worse.
This is contrary to some indications from populat concerns
about excessive copper in commercial dog foods. 
Copper use requires uptake and transportation to 
various targets. Transport out of the liver can be hindered
for reasons such as ceruloplasmin defects.

Coughing and other subjective signs were often used
to monitor progress and notes were not always sufficient.
Coughing as described before can be produced by many
causes. Here, we were concerned mostly with infection,
trachea collapse, and heart enlargement. Honking
related to trachea collapse may be more common when the
dog is excited. In this case, improving "energy" may
produce more coughing even though the dog is largely
healthier but the heart is still large or trachea
still soft. This is further complicated with additions
of vitamins that tend to promote alertness as was 
observed with histidine ( indeed there was some concern
about aggression when initiating it ). This may not
have been fully appreciated early on.  


B-6 deficiency has been linked to excess copper excretion \cite{PMID7814236}
possibly making copper a secondary excretion issue
rather than a primary absorption problem. 
However, the concern with excess B-6 intake creating effective
deficiency is a reminder to look more carefully at all data. 

%%%%%%%%%%%%%%
%%%%%%%%%%%%%% cutoff

Iver a broad range of genetics, its likely that copper intake
can be raised as long as other nutirents are also given
to handle the copper beneficially.
Candidate nutrients include tryrosine and tryptophan.


\section{Limitations}
While the other components were mentioned as important,
it needs to be reiterated that the 
the other snack components could have effected copper handling
significantly and supplementation with another diet lacking
these components may not be beneficiail but copper restriction
may not be either. Most food ingredient interact with matals 
to varying degrees and this notably contained citric acid
and spinach along with amino acids. 

The residential setting made it difficult to control or monitor
all of the factors which could effect health. Besides the main
kibble meals not being recorded for some dogs, intake of food
and foriegn objects was common and unpredictable. 
Supplement quantities were often measured by volume using kitchen
utensils known to be poorly calibrated. 
Completely unknown experiences or factors may be involved in their
subjective behaviors.  Cigarette smoke exposure was common
but variable.
As is always the case, despite MUQED's ability to keep strcutured
outcome notes on things like cough, the resulting outcome
data was very sparse and relies on memory in some cases.
The lesson remains that notes and data always need to be
more complete. 



\section{Conclusions}
Copper has to be suspected of being important in dogs for 
functions that likely
include strengthening  of structural elements such as the trachea,
volumetric energy effeiciency of the heart, and infection control.
In the GI tract, it may moderate pathogenic phenotypes and
change community structure of microbiome. 
Accumulation in the liver may reflect export problems
rather than too much intake as signalling exists to
regulate uptake and disposal. Defects may be due to other nutrients
and particularly anything that interfers with ceruloplasmin 
synthesis or quality. 

Internal transport and uptake however may both rely on 
GI defects which limit nutrient avialbility. Low stomach
acid may be one common problem. 

Zinc excess may also interfer with copper
deployment.
Dog genetics are varied and specifics likely vary too.
Similar considerations may apply to humans. 

Liver pathology that includes atypical amounts of copper
may not reflect excess dietary intake but some other problem
that needs to be fixed. 

\section{Supplemental Information}

Dog diet data files are available online at
{\url{https://github.com/mmarchywka/dogdata}}
or other locations as may be required.
The author may also be contacted if onlines sources are not
avialble. Raw MUQED format as well as parsed text formats
are avilable although MUQED software availbility is in the works.


\subsection{Computer Code}
note anything using "snacks\_Collated.ssv" is obsolete as it messed
up adjectives etc. use "linc\_graph -dt-mo"
NB : the "datealias" entries need to be updated not just datemin and datemax
and the latter may not even do anything lol.  A note also
"reporting units" for many new nouns are not right as tsp
 has replaced mg etc. 

\begin{lstlisting}

diet tables, 

2766  ./run_linc_graph -dt-mo txt/happy2cu.txt 
 2767  texfrag -include xxxtable 
 2768  mv xxxtable /home/documents/latex/proj/copper/keep/monthly.tex

datascope output, 

./run_linc_graph -2dscope Iodine "Happy" "filter=lag20"


\end{lstlisting}
\section{Bibliography}


\bibliography{\mjmbasename,\mjmaddbio}
\bibliographystyle{plainurl}


%%%%%%%%%%%%%%%%%%%%%%%%%%%%%%%%%%%%%%%%%%%%%%%%%%%%%%%%%%%%%%%%%%%%%%%%%%%%%
\begin{acknowledgments} 

% \input{generalack.tex}
\begin{enumerate}
\item Pubmed eutils facilities and the basic research it provides. 
\item Free software including Linux, R, LaTex  etc.
\item Thanks everyone who contributed incidental support. 
\end{enumerate}

\end{acknowledgments}

%%%%%%%%%%%%%%%%%%%%%%%%%%%%%%%%%%%%%%%%%%%%%%%%%%%%%%%%%%%%%%%%%%%%%%%%%%%%%
\clearpage
\appendix

\begin{mdpicomment}

\section{ Statement of Conflicts }
 No specific funding was used in this effort and there are no relationships
with others that could create a conflict of interest. I would like to develop
these ideas further and have obvious bias towards making them appear 
successful. Barbara Cade, the dog owner, has worked in the pet food industry
but this does not likely create a conflict. We have no interest in the
makers of any of the products named in this work.  

\end{mdpicomment}

\begin{mdpicomment}
\section{About the Authors and Facility}
This work was performed at a dog rescue run by Barbara Cade and
housed in rural Georgia.  The author of this report 
,Mike Marchywka,
has a background in electrical engineering and 
has done extensive research using free online literature sources.  
I hope to find additional people interested in critically 
examining the results and verify that they can be reproduced
effectively to treat other dogs.

\begin{comment}
\begin{figure}[htb] 
\centering
\mjmed{ picture commented out to save space in drafts...  } 
%\includegraphics[width=\picwidth]{me_on_brick.jpg}
\caption{ 
 }
\end{figure}

\end{comment}



\section{Some Common Logical Fallacies and Misdirections}
\label{appendix:errors}

The analysis errors that confuse the problem with the
solution may be catagorized in a number of way but
consider the following possibilities. No attempt is made
to organize, catagorize, or make entries "orthogonal"
as its unclear how or why to do that right now.  

\newcommand{\myitem}[2]{\item {\bf #1 : } #2 \par}
\begin{enumerate}
\myitem{people over data} { Thjis comes up in essays on long lived metdical myths. Opinions, votes, don't determine natural laws }
\myitem{supply v demand}{ An excess quantity is not inherently due to either supply or demand }
\myitem{false dichotomy}{ good and evil  }
\myitem{non-monotonic curves }{ \mjmquote{ show me a monotonic dose response curve and I'll show you a fool } }
\myitem{fix or remove }{ Often there are 2 approaches to getting the same result with a partially broken system.  }
\myitem{one extreme says nothing about theother  }{ Both invcreasing or decreasing some quantity may produce the same ( beneficial or evil ) effect.  }
\myitem{hidden interaction }{ The observed result requires some interaction partners  }
\myitem{X is a Y   }{ Role confusion leads to more untested assumptions.  }
\myitem{local optimum  }{ prohibits consideration of global optimum over rugged parameter space with bad figure of merit.  }
\myitem{rabbit trail  }{  Successive approcimations, even as advocated in this seris of works, can lead a strategy down the wrong path.  }
\myitem{platitudes  }{ better to be safe than sorry lol   }
\myitem{equations  }{ consider spherical horse linear approx   }
\myitem{proxies   }{ Almost only counts in horseshoes and handgrenades   }
\myitem{models   }{ This usually means cause and effect is tossed out and the premise of the work has nothing to do with the premise of the real disease and that is often not even known well.   }
\myitem{only relies on  empirical obervations  }{ no cuase and effect stablished   }
\myitem{irnores empirical obervations  }{ "sanity  check " Theory based on incomplete system topology needs to be consistent with empirical observations  }
\myitem{irnores empirical obervations  }{ Theory based on incomplete system topology needs to be consistent with empirical observations  }
\myitem{other state variables }{ Hidden variables include a limting nutrient that is needed to beneficially use the entity under test.   }
\myitem{regulatory landscape }{ As life is generally robust control systems have evolved but may have non-obvious failure modes. }
\myitem{trying harder  }{ If a prefered response is frustrated in attaining a survival relevant goal, there may be signals to try a less orderly response more vigorously. The disorganized response can be suppressed by enabling the effective one.  }
\myitem{narrow corridor }{  Given host-other evolution, you could expect dose response curves with many features but more likely a tight tolerance means that the approach is missing something more robust.  }
\end{enumerate}





\section{Background Diet Sumnary}

\newcommand{\mjmdatemin}{2023-10-01}
\newcommand{\mjmdatemax}{2024-04-10}
\newcommand{\mjmsuperscripts}{{\bf a) } SMVT substrate. Biotin, Pantothenate, Lipoic Acid, and Iodine known to compete..{\bf c) } hamburger with varying fat percentages- 7,10,15,20, etc. ..}
\begin{table}[H]
\centering
\begin{tabular}{|l|r|r|r|r|r|}
\hline
Name&2023-10 Oct&2023-11 Nov&2023-12 Dec&2024-01 Jan&2024-02 Feb\\
\hline
{\bf FOOD}&&&&&\\
$\textrm{KCl(tsp~kcl)}$&0.045 ;0.031;23/23&0.047 ;0.031;30/30&0.085 ;0.062;24/24&0.094 ;0.062;31/31&0.093 ;0.062;29/29\\
$\textrm{KibbleAmJrLaPo}$&0.036 ;0.037;22/23&0.065 ;0.075;30/30&0.07 ;0.075;23/24&0.075 ;0.075;31/31&0.071 ;0.098;29/29\\
$\textrm{KibbleLogic}$&0.024 ;0.025;22/23&0.043 ;0.05;30/30&0.047 ;0.05;23/24&0.05 ;0.05;31/31&0.047 ;0.065;29/29\\
$\textrm{b10ngnc}^{\left(c\right)}$&0.019 ;0.25;1/23&0.11 ;0.25;9/30&0.047 ;0.25;3/24&0.11 ;1;7/31&0.067 ;0.25;5/29\\
$\textrm{b15ngnc}^{\left(c\right)}$&&0.044 ;0.25;5/30&0.021 ;0.25;1/24&0.06 ;0.25;4/31&\\
$\textrm{b20ngnc}^{\left(c\right)}$&0.18 ;0.25;14/23&0.13 ;0.25;10/30&0.25 ;0.25;14/24&0.14 ;0.25;11/31&0.28 ;0.25;19/29\\
$\textrm{b25ngnc}$&0.11 ;0.25;9/23&0.067 ;0.25;6/30&0.026 ;0.25;2/24&0.02 ;0.25;2/31&0.039 ;0.25;4/29\\
$\textrm{b7ngnc}^{\left(c\right)}$&0.1 ;0.25;8/23&0.14 ;0.25;11/30&0.14 ;0.25;9/24&0.2 ;0.25;17/31&0.11 ;0.25;7/29\\
$\textrm{blackberry}$&&0.058 ;0.25;5/30&0.3 ;0.25;20/24&&\\
$\textrm{blueberry}$&2.4 ;3.8;23/23&2.4 ;2.2;30/30&1.9 ;2;20/24&0.71 ;1.5;13/31&1.2 ;1.5;29/29\\
$\textrm{carrot}$&0.35 ;0.25;23/23&0.36 ;0.25;30/30&0.36 ;0.25;24/24&0.38 ;0.25;31/31&0.38 ;0.25;29/29\\
$\textrm{cbbrothbs}$&&&&&0.022 ;0.25;3/29\\
$\textrm{cbbroth}$&0.16 ;0.25;10/23&0.071 ;0.25;6/30&&0.21 ;0.25;15/31&0.25 ;0.25;16/29\\
$\textrm{citrate(tsp~citrate)}$&0.045 ;0.031;23/23&0.047 ;0.031;30/30&0.048 ;0.062;24/24&0.058 ;0.062;31/31&0.092 ;0.062;29/29\\
$\textrm{ctbrothbs}$&0.082 ;0.25;5/23&0.4 ;0.25;25/30&0.48 ;0.25;24/24&0.29 ;0.25;19/31&0.22 ;0.25;14/29\\
$\textrm{ctbroth}$&0.17 ;0.25;11/23&&&0.032 ;1;1/31&\\
$\textrm{eggo3}$&0.065 ;0.12;23/23&0.062 ;0.062;30/30&0.055 ;0.12;20/24&0.062 ;0.062;31/31&0.062 ;0.062;29/29\\
$\textrm{eggo}$&&&0.01 ;0.062;4/24&&\\
$\textrm{eggshell}$&0.13 ;0.25;23/23&0.12 ;0.12;30/30&0.11 ;0.25;21/24&&\\
$\textrm{garlic}$&0.022 ;0.25;2/23&0.22 ;0.25;26/30&0.083 ;0.25;8/24&1.2 ;1;27/31&0.99 ;1;22/29\\
$\textrm{marrow}$&0.19 ;0.25;12/23&0.37 ;0.25;30/30&0.083 ;0.25;6/24&&0.078 ;0.25;7/29\\
$\textrm{oliveoil(tsp)}$&0.035 ;0.12;8/23&0.014 ;0.12;4/30&&&0.039 ;0.12;9/29\\
$\textrm{pepper}$&0.36 ;0.25;23/23&0.38 ;0.25;30/30&0.35 ;0.25;24/24&0.36 ;0.25;31/31&0.38 ;0.25;29/29\\
$\textrm{pineapple}$&&&0.021 ;0.25;2/24&&\\
$\textrm{raspberry}$&0.32 ;0.25;23/23&0.28 ;0.25;24/30&&&\\
$\textrm{salmon}$&&0.043 ;0.25;8/30&&0.025 ;0.25;3/31&\\
$\textrm{shrimp(grams)}$&&3 ;38;5/30&4.9 ;16;9/24&2.8 ;16;8/31&1.8 ;13;4/29\\
$\textrm{spinach}$&&0.15 ;0.25;12/30&0.36 ;0.25;24/24&0.38 ;0.25;31/31&0.36 ;0.25;28/29\\
$\textrm{sunflowerseed}$&0.23 ;0.25;21/23&0.25 ;0.25;30/30&0.21 ;0.25;20/24&&0.034 ;0.25;4/29\\
$\textrm{tomato}$&0.36 ;0.25;23/23&0.23 ;0.25;19/30&0.18 ;0.25;12/24&0.17 ;0.25;15/31&0.19 ;0.25;29/29\\
$\textrm{tuna(oz)}$&&&&&\\
$\textrm{turkey}$&0.34 ;0.25;23/23&0.37 ;0.25;30/30&0.35 ;0.25;24/24&0.36 ;0.25;31/31&0.36 ;0.25;29/29\\
$\textrm{vinegar(tsp)}$&0.09 ;0.062;23/23&0.094 ;0.062;30/30&0.09 ;0.062;24/24&0.068 ;0.062;24/31&2.16e-03 ;0.062;1/29\\
{\bf VITAMIN}&&&&&\\
$\textrm{B-1(mg)}$&4.09e-03 ;0.012;15/23&5.87e-03 ;0.0059;30/30&6.12e-03 ;0.012;24/24&5.69e-03 ;0.0059;30/31&5.87e-03 ;0.0059;29/29\\
$\textrm{B-12(mg)}$&0.033 ;0.25;5/23&0.029 ;0.25;5/30&0.047 ;0.25;6/24&0.024 ;0.25;5/31&0.034 ;0.12;8/29\\
$\textrm{B-2(mg)}$&5.7 ;16;15/23&7.9 ;8.1;29/30&8.1 ;16;24/24&21 ;32;30/31&43 ;65;29/29\\
$\textrm{B-3(mg)}$&8.3 ;24;15/23&12 ;12;30/30&12 ;24;23/24&31 ;48;30/31&60 ;48;29/29\\
$\textrm{B-6(mg)}$&6 ;12;11/23&12 ;12;28/30&11 ;12;21/24&8.9 ;12;29/31&5.8 ;12;26/29\\
$\textrm{B-multi(count)}$&0.022 ;0.062;8/23&&&2.02e-03 ;0.062;1/31&\\
$\textrm{Cu(mg)}$&0.11 ;0.25;10/23&0.76 ;2;19/30&0.86 ;2;19/24&1.9 ;2;30/31&1.9 ;2;28/29\\
$\textrm{D-3(iu)}$&91 ;300;7/23&60 ;300;6/30&62 ;300;5/24&58 ;300;6/31&52 ;300;5/29\\
$\textrm{Iodine(mg)}^{\left(a\right)}$&2.3 ;12;8/23&0.1 ;0.78;4/30&0.065 ;0.78;2/24&0.1 ;0.78;4/31&0.13 ;0.78;5/29\\
$\textrm{K1(mg)}$&0.38 ;1.2;7/23&0.92 ;1.2;22/30&1.1 ;1.2;22/24&1.1 ;1.2;27/31&1.2 ;1.2;28/29\\
$\textrm{K2(mg)}$&1 ;1.6;15/23&0.3 ;1.9;7/30&0.47 ;3.8;3/24&0.91 ;3.8;8/31&0.81 ;3.8;8/29\\
$\textrm{K2MK7(mg)}$&1.63e-03 ;0.025;2/23&5.83e-03 ;0.025;7/30&2.08e-03 ;0.025;2/24&&\\
$\textrm{MgCitrate(mg)}$&96 ;200;21/23&100 ;100;30/30&92 ;100;22/24&31 ;100;10/31&76 ;100;22/29\\
$\textrm{Mn(mg)}$&&&0.042 ;1;1/24&0.21 ;0.62;12/31&0.12 ;1;6/29\\
$\textrm{Se(mcg)}$&&0.42 ;12;1/30&&&0.43 ;12;1/29\\
\hline
\end{tabular}
\caption{Part 1 of 2.  Events Summary for Happy   from 2023-10-01 to 2024-04-10A summary of most dietary components and events  for selected months between \mjmdatemin and \mjmdatemax. Format is average daily amount ;maximum; days given/ days in interval . Units are arbitrary except where noted. Any  superscripts are defined as follows:  \mjmsuperscripts}
\end{table}
\begin{table}[H]
\centering
\begin{tabular}{|l|r|r|r|r|r|}
\hline
Name&2023-10 Oct&2023-11 Nov&2023-12 Dec&2024-01 Jan&2024-02 Feb\\
\hline
$\textrm{Zn(mg~zn)}$&1.3 ;5.9;9/23&1.1 ;5.9;10/30&0.73 ;2.9;6/24&0.47 ;2.9;5/31&0.61 ;5.9;5/29\\
$\textrm{arginine(mg)}$&68 ;175;9/23&82 ;350;10/30&51 ;175;7/24&79 ;350;12/31&275 ;350;15/29\\
$\textrm{biotin(mg)}^{\left(a\right)}$&2.4 ;5;11/23&4.3 ;5;26/30&4 ;5;19/24&3.5 ;5;22/31&3.6 ;5;21/29\\
$\textrm{folate(mg)}$&0.022 ;0.12;5/23&0.019 ;0.12;6/30&0.018 ;0.12;4/24&0.016 ;0.12;5/31&0.011 ;0.12;3/29\\
$\textrm{histidine(tsp)}$&&&&&2.42e-03 ;0.016;7/29\\
$\textrm{histidinehcl(mg)}$&3.7 ;85;1/23&1.4 ;42;1/30&1.6 ;38;1/24&&\\
$\textrm{iron(mg)}$&&1 ;4;8/30&1.8 ;4;11/24&1.3 ;4;10/31&2.2 ;4;18/29\\
$\textrm{isoleucine(mg)}$&30 ;200;5/23&47 ;200;8/30&17 ;200;2/24&48 ;200;9/31&45 ;200;8/29\\
$\textrm{lecithin(mg)}$&215 ;225;22/23&225 ;225;30/30&281 ;225;22/24&330 ;225;31/31&338 ;225;29/29\\
$\textrm{lecithin(tsp)}$&0.046 ;0.062;22/23&0.036 ;0.042;30/30&0.012 ;0.062;8/24&&\\
$\textrm{leucine(mg)}$&74 ;162;20/23&76 ;81;28/30&85 ;162;24/24&66 ;81;25/31&67 ;81;24/29\\
$\textrm{leucine}$&&&&&\\
$\textrm{lipoicacid(mg)}^{\left(a\right)}$&3.1 ;25;5/23&7.6 ;25;16/30&24 ;25;21/24&18 ;25;22/31&31 ;25;28/29\\
$\textrm{lysinehcl(mg)}$&170 ;162;23/23&203 ;162;30/30&186 ;162;24/24&218 ;325;30/31&235 ;325;14/29\\
$\textrm{methionine(mg)}$&57 ;62;21/23&46 ;62;22/30&38 ;125;20/24&4 ;62;3/31&9.7 ;62;7/29\\
$\textrm{pantothenate(mg)}^{\left(a\right)}$&22 ;78;12/23&20 ;39;15/30&21 ;39;13/24&32 ;39;25/31&30 ;39;22/29\\
$\textrm{phenylalanine(mg)}$&38 ;125;7/23&23 ;125;6/30&18 ;125;4/24&8.1 ;125;2/31&15 ;125;4/29\\
$\textrm{proline(mg)}$&143 ;100;23/23&35 ;100;7/30&&&\\
$\textrm{taurine(mg)}$&323 ;225;23/23&338 ;225;30/30&323 ;225;24/24&345 ;225;31/31&338 ;225;29/29\\
$\textrm{threonine(mg)}$&95 ;162;23/23&374 ;325;30/30&467 ;325;24/24&488 ;325;31/31&487 ;325;29/29\\
$\textrm{tryptophan(mg)}$&52 ;150;14/23&40 ;150;14/30&25 ;150;6/24&17 ;150;6/31&24 ;75;10/29\\
$\textrm{tyrosine(mg)}$&17 ;100;4/23&6.7 ;100;2/30&12 ;100;3/24&19 ;100;6/31&19 ;100;6/29\\
$\textrm{valine(mg)}$&165 ;200;19/23&160 ;200;24/30&133 ;200;16/24&135 ;200;21/31&159 ;200;23/29\\
$\textrm{vitamina(iu)}$&489 ;2250;5/23&600 ;2250;8/30&656 ;4500;6/24&435 ;2250;6/31&466 ;2250;6/29\\
$\textrm{vitaminc(tsp)}$&3.23e-03 ;0.0078;11/23&3.39e-03 ;0.0078;13/30&8.14e-04 ;0.0039;5/24&5.04e-04 ;0.0039;4/31&5.39e-04 ;0.0078;2/29\\
$\textrm{vitamine(iu)}$&8.2 ;38;5/23&8.8 ;38;7/30&9.4 ;38;6/24&7.3 ;38;6/31&6.5 ;38;5/29\\
{\bf MEDICINE}&&&&&\\
$\textrm{SnAg}$&&&&1.1 ;1;13/31&0.66 ;1;12/29\\
$\textrm{sodiumbenzoate(tsp)}$&0.011 ;0.016;12/23&8.85e-03 ;0.016;12/30&0.012 ;0.031;15/24&0.018 ;0.016;25/31&0.018 ;0.016;24/29\\
$\textrm{wormer}$&&&&&\\
{\bf RESULT}&&&&&\\
$\textrm{weight(lbs)}$&&&0.63 ;15;1/24&&1.1 ;16;2/29\\
&&&&&\\
$\textrm{sorbitol(tsp)}$&0.045 ;0.031;23/23&0.047 ;0.031;30/30&0.045 ;0.031;24/24&0.046 ;0.062;31/31&0.047 ;0.031;29/29\\
&&&&&\\
&&&&&\\
&&&&&\\
&&&&&\\
&&&&&\\
\hline
\end{tabular}
\caption{Part 2 of 2.  Events Summary for Happy   from 2023-10-01 to 2024-04-10A summary of most dietary components and events  for selected months between \mjmdatemin and \mjmdatemax. Format is average daily amount ;maximum; days given/ days in interval . Units are arbitrary except where noted. Any  superscripts are defined as follows:  \mjmsuperscripts}
\end{table}
\begin{table}[H]
\centering
\begin{tabular}{|l|r|r|}
\hline
Name&2024-03 Mar&2024-04 Apr\\
\hline
{\bf FOOD}&&\\
$\textrm{KCl(tsp~kcl)}$&0.084 ;0.062;20/20&0.087 ;0.062;10/10\\
$\textrm{KibbleAmJrLaPo}$&0.034 ;0.037;18/20&0.034 ;0.037;9/10\\
$\textrm{KibbleLogic}$&0.023 ;0.025;18/20&0.022 ;0.025;9/10\\
$\textrm{b10ngnc}^{\left(c\right)}$&0.069 ;0.25;4/20&0.056 ;0.25;2/10\\
$\textrm{b15ngnc}^{\left(c\right)}$&0.022 ;0.25;2/20&\\
$\textrm{b20ngnc}^{\left(c\right)}$&0.33 ;0.25;17/20&0.19 ;0.25;6/10\\
$\textrm{b25ngnc}$&&\\
$\textrm{b7ngnc}^{\left(c\right)}$&&0.16 ;0.25;4/10\\
$\textrm{blackberry}$&&\\
$\textrm{blueberry}$&0.75 ;0.75;20/20&0.9 ;1;10/10\\
$\textrm{carrot}$&0.35 ;0.25;20/20&0.35 ;0.25;10/10\\
$\textrm{cbbrothbs}$&&\\
$\textrm{cbbroth}$&0.1 ;0.25;5/20&\\
$\textrm{citrate(tsp~citrate)}$&0.081 ;0.062;20/20&0.086 ;0.062;10/10\\
$\textrm{ctbrothbs}$&0.33 ;0.25;17/20&0.41 ;0.25;10/10\\
$\textrm{ctbroth}$&&\\
$\textrm{eggo3}$&0.025 ;0.062;8/20&0.062 ;0.062;10/10\\
$\textrm{eggo}$&0.037 ;0.062;12/20&\\
$\textrm{eggshell}$&&\\
$\textrm{garlic}$&1.4 ;1;18/20&1.1 ;1;10/10\\
$\textrm{marrow}$&&\\
$\textrm{oliveoil(tsp)}$&0.042 ;0.12;6/20&\\
$\textrm{pepper}$&0.36 ;0.25;20/20&0.35 ;0.25;10/10\\
$\textrm{pineapple}$&&\\
$\textrm{raspberry}$&&\\
$\textrm{salmon}$&&\\
$\textrm{shrimp(grams)}$&&\\
$\textrm{spinach}$&0.35 ;0.25;20/20&0.35 ;0.25;10/10\\
$\textrm{sunflowerseed}$&0.037 ;0.25;3/20&0.2 ;0.25;8/10\\
$\textrm{tomato}$&0.12 ;0.12;20/20&0.12 ;0.12;10/10\\
$\textrm{tuna(oz)}$&0.062 ;0.25;5/20&0.075 ;0.25;3/10\\
$\textrm{turkey}$&0.33 ;0.25;20/20&0.35 ;0.25;10/10\\
$\textrm{vinegar(tsp)}$&6.25e-03 ;0.062;3/20&3.13e-03 ;0.031;1/10\\
{\bf VITAMIN}&&\\
$\textrm{B-1(mg)}$&5.58e-03 ;0.012;18/20&5.87e-03 ;0.0059;10/10\\
$\textrm{B-12(mg)}$&0.05 ;0.25;6/20&0.025 ;0.12;2/10\\
$\textrm{B-2(mg)}$&47 ;16;20/20&37 ;16;10/10\\
$\textrm{B-3(mg)}$&69 ;24;20/20&55 ;24;10/10\\
$\textrm{B-6(mg)}$&4.7 ;6.2;15/20&3.8 ;6.2;6/10\\
$\textrm{B-multi(count)}$&3.13e-03 ;0.062;1/20&\\
$\textrm{Cu(mg)}$&2.2 ;2;20/20&2.6 ;2;10/10\\
$\textrm{D-3(iu)}$&62 ;350;4/20&60 ;300;2/10\\
$\textrm{Iodine(mg)}^{\left(a\right)}$&0.19 ;0.78;5/20&0.16 ;0.78;2/10\\
$\textrm{K1(mg)}$&1.1 ;1.2;17/20&1.2 ;1.2;10/10\\
$\textrm{K2(mg)}$&0.75 ;3.1;6/20&\\
$\textrm{K2MK7(mg)}$&&\\
$\textrm{MgCitrate(mg)}$&88 ;100;18/20&90 ;100;9/10\\
$\textrm{Mn(mg)}$&0.14 ;1.2;3/20&\\
$\textrm{Se(mcg)}$&&\\
\hline
\end{tabular}
\caption{Part 1 of 2.  Events Summary for Happy   from 2023-10-01 to 2024-04-10A summary of most dietary components and events  for selected months between \mjmdatemin and \mjmdatemax. Format is average daily amount ;maximum; days given/ days in interval . Units are arbitrary except where noted. Any  superscripts are defined as follows:  \mjmsuperscripts}
\end{table}
\begin{table}[H]
\centering
\begin{tabular}{|l|r|r|}
\hline
Name&2024-03 Mar&2024-04 Apr\\
\hline
$\textrm{Zn(mg~zn)}$&0.73 ;5.9;3/20&0.59 ;5.9;1/10\\
$\textrm{arginine(mg)}$&245 ;350;10/20&228 ;350;5/10\\
$\textrm{biotin(mg)}^{\left(a\right)}$&3.4 ;5;14/20&3.5 ;5;7/10\\
$\textrm{folate(mg)}$&0.013 ;0.12;3/20&\\
$\textrm{histidine(tsp)}$&0.021 ;0.016;19/20&0.02 ;0.031;8/10\\
$\textrm{histidinehcl(mg)}$&&\\
$\textrm{iron(mg)}$&2.4 ;5.3;17/20&5.3 ;5.3;8/10\\
$\textrm{isoleucine(mg)}$&25 ;200;3/20&20 ;200;1/10\\
$\textrm{lecithin(mg)}$&315 ;225;20/20&315 ;225;10/10\\
$\textrm{lecithin(tsp)}$&&\\
$\textrm{leucine(mg)}$&73 ;81;18/20&81 ;81;10/10\\
$\textrm{leucine}$&&\\
$\textrm{lipoicacid(mg)}^{\left(a\right)}$&16 ;25;12/20&20 ;25;8/10\\
$\textrm{lysinehcl(mg)}$&228 ;325;10/20&244 ;325;5/10\\
$\textrm{methionine(mg)}$&12 ;62;8/20&25 ;62;4/10\\
$\textrm{pantothenate(mg)}^{\left(a\right)}$&33 ;39;17/20&35 ;39;9/10\\
$\textrm{phenylalanine(mg)}$&28 ;125;5/20&12 ;125;1/10\\
$\textrm{proline(mg)}$&&\\
$\textrm{taurine(mg)}$&315 ;225;20/20&315 ;225;10/10\\
$\textrm{threonine(mg)}$&455 ;325;20/20&422 ;325;10/10\\
$\textrm{tryptophan(mg)}$&26 ;75;7/20&22 ;75;4/10\\
$\textrm{tyrosine(mg)}$&22 ;100;6/20&30 ;100;3/10\\
$\textrm{valine(mg)}$&160 ;200;16/20&160 ;200;8/10\\
$\textrm{vitamina(iu)}$&506 ;2250;5/20&675 ;2250;3/10\\
$\textrm{vitaminc(tsp)}$&8.79e-04 ;0.0039;5/20&1.95e-03 ;0.0039;5/10\\
$\textrm{vitamine(iu)}$&7.5 ;38;4/20&7.5 ;38;2/10\\
{\bf MEDICINE}&&\\
$\textrm{SnAg}$&&\\
$\textrm{sodiumbenzoate(tsp)}$&0.016 ;0.016;14/20&7.81e-04 ;0.0078;1/10\\
$\textrm{wormer}$&0.075 ;1.5;1/20&\\
{\bf RESULT}&&\\
$\textrm{weight(lbs)}$&&\\
&&\\
$\textrm{sorbitol(tsp)}$&0.044 ;0.031;20/20&0.041 ;0.031;10/10\\
&&\\
&&\\
&&\\
&&\\
&&\\
\hline
\end{tabular}
\caption{Part 2 of 2.  Events Summary for Happy   from 2023-10-01 to 2024-04-10A summary of most dietary components and events  for selected months between \mjmdatemin and \mjmdatemax. Format is average daily amount ;maximum; days given/ days in interval . Units are arbitrary except where noted. Any  superscripts are defined as follows:  \mjmsuperscripts}
\end{table}



\section{Notable Food Components with Copper Interactions}

\label{appendix:interactions}
\mjmhuntcu

\section{Symbols, Abbreviations and Colloquialisms}

\begin{comment}
% grep "[A-Z][A-Z]" paradox.tex | sed -e 's/[^A-Z]/\n/g' | grep "[A-Z]" | sort | uniq -c
% cat  paradox.tex | sed -e 's/  */\n/g' | grep "[A-Z][A-Z]"  | grep -v "[^A-Z]" | sort | uniq  |awk '{print $0" &   \\\\"; }'
\end{comment}


%\abbreviations{The following abbreviations are used in this manuscript:\\
%\begin{table}
\noindent
\begin{tabular}{@{}ll}
%SMVT & Sodium dependent Multi-Vitamin Transporter\\
TERM & definition and meaning   \\
\hline
%TLA & Three letter acronym\\
%LD & linear dichroism
\end{tabular} % }
%\end{table}

% https://tex.stackexchange.com/questions/5957/bibtex-entry-for-white-papers-and-technical-reports

\section{General caveats and disclaimer }
\label{appendix:caveats}

%\input{disclaimer-informal.tex}

This document was created in the hope it will be interesting to
someone including me by providing information 
about some topic that may include personal experience or a literature
review or description of a speculative theory or idea.
There is no assurance that the content of this work will be
useful for any paricular purpose. 
%In no case am I claiming to provide useful advice on any matter
%but attempting to describe events in terms of literature known
%to me. 


All statements in this document were true to the best of my knowledge
at the time they were made and every attempt is made to assure
they are not misleading or confusing. However, information provided by
others and observations that can be manipulated by unknown causes  
( "gaslighting" ) may be misleading. Any use of this information should
be preceded by validation including replication where feasible.
Errors may enter into the final work at every step from conception
and research to final editing. 
%No assurance can exist that obvious conclusions will be useful
%and may be misleading. 



Documents labelled "NOTES" or "not public" contain
substantial informal or speculative content that
may be terse and poorly edited or even sarcastic or profane.
Documents labelled as "public" have generally been edited
to be more coherent but probably have not been reviewed
or proof read. 

Generally non-public documents are labelled as such to avoid
confusion and embarassment and should be read with that understanding.


\section{Citing this as a tech report or white paper }
\label{appendix:citing}

Note: This is mostly manually entered and not assured to be error free.

This is tech report \mjmtrno. 

\begin{table}[H] \centering
\begin{tabular}{r|r|c|r}
Version & Date & Comments  &  \\
0.01 & \mjmmakedate  &  Create from empty.tex template  &  \\
-  & \today & version  \mjmversion { }   \mjmtrno  &  \\
1.0 & 20xx-xx-xx & First revision for distribution &  \\
\end{tabular}
\end{table}

Released versions,

build script needs to include empty releases.tex
\begin{table}[H] \centering
\begin{tabular}{|r|r|l|}
Version & Date & URL    \\
\hline
&  &  \\
% version & date & url  \\
%.1 table & 2021-08-17& {\url{https://www.linkedin.com/posts/marchywka_draft-compare-72020-theory-with-interim-activity-6833343119203860480--wJv}} \\
%.1 table & 2021-08-17& {\url{https://www.researchgate.net/publication/353946686_Draft_table_comparing_expectations_to_recent_results_with_covid-19}} \\
%.1 table & 2021-08-17 & {\url{https://www.academia.edu/s/34e160cae9}} \\

\hline
\end{tabular}
\end{table}

\begin{minipage}{\linewidth}
%\input{bibtex2.txt}
%\input{bibtex3.txt}
\mjmshowbib
\end{minipage}

\vspace{1cm}
Supporting files. Note that some dates,sizes, and md5's will change as this is
rebuilt.

This really needs to include the data analysis code 
but right now it is auto generated picking up things from prior
build in many cases 
\lstinputlisting{\mjmbasename.bundle_checksums}
\end{mdpicomment}
\end{document}


%%%%%%%%%%%%%%%%%%%%%%%%%%%%%%%%%%%%%%%%%%%%%%%%%%%%%%%%%%%%%%%%%%
% Sample template for MIT Junior Lab Student Written Summaries
% Available from http://web.mit.edu/8.13/www/Samplepaper/sample-paper.tex
% Last Updated April 12, 2007
% Adapted from the American Physical Societies REVTeK-4 Pages
% at http://publish.aps.org

\setlength{\paperheight}{11in}
% http://tex.stackexchange.com/questions/74636/mla-package-and-thumbpdf
\makeatletter
\@namedef{ver@thumbpdf.sty}{}
\makeatother
\documentclass[aps,secnumarabic,balancelastpage,amsmath,amssymb,nofootinbib]{revtex4}

%http://tex.stackexchange.com/questions/119905/insert-multiple-figures-in-latex

\usepackage[nomessages]{fp}    %mjm   needed for chemfig and mol2chemfig computed angles  
\usepackage{siunitx}        %mjm  appendix table subsections  
%\usepackage{morefloats}        %mjm   saving up figs for the end   
% fking incompatibvle fuxk ing floatrow 
%\usepackage{float}        %mjm  appendix table subsections  
\usepackage{pbox}        %mjm box off junk  
\usepackage{comment}        %mjm  see the build options  
\usepackage{framed}        %mjm box off junk  
\usepackage{lgrind}        % convert program listings to a form includable in a LaTeX document
% comment out for biblatex test
\usepackage{chapterbib}    % allows a bibliography for each chapter (each labguide has it's own)
%\usepackage{biblatex}   
\usepackage{color}         % produces boxes or entire pages with colored backgrounds
\usepackage{graphics}      % standard graphics specifications
\usepackage[pdftex]{graphicx}      % alternative graphics specifications
%\usepackage{graphicx}      % alternative graphics specifications
\usepackage{longtable}     % helps with long table options
\usepackage{epsf}          % old package handles encapsulated post script issues
\usepackage{bm}            % special 'bold-math' package
\usepackage{url}            % path for stupid jobname   
%\usepackage{asymptote}     % For typesetting of mathematical illustrations
\usepackage{thumbpdf}
\usepackage[colorlinks=true]{hyperref}  % this package should be added after all others
%\usepackage[draft=false, x-bib-pages=\input{\mjmbasename.last_page}, colorlinks=true]{hyperref}  % this package should be added after all others
%\usepackage[draft=false, x-bib-pages=\input{allbib.last_page}, colorlinks=true]{hyperref}  % this package should be added after all others
                                        % use as follows: \url{http://web.mit.edu/8.13}

%http://tex.stackexchange.com/questions/12676/add-notes-under-the-table
%\usepackage{booktabs,caption,fixltx2e}
% not work with subfig????
\usepackage[CaptionAfterwards]{fltpage}
\usepackage{lipsum}

% this does not  work .... 
%\usepackage[utf8]{inputenc}
%\usepackage{tabulary}
\usepackage[para,online,flushleft]{threeparttable}
%\usepackage{threeparttable}

%\usepackage{chemfig}
\usepackage[version=3]{mhchem}        %  
\usepackage{mol2chemfig}
% chemfig vriables maybe
\usepackage{xstring}        %mjm  appendix table subsections  
%\usepackage{chemformula}
% \usepackage{chemmacros}
\usepackage{floatrow}
\usepackage{fancyhdr}
%  underscore in jobmane f 
% https://latex.org/forum/viewtopic.php?t=2975
% this also messes up pdftotext  
%\usepackage[T1]{fontenc}
\usepackage{dcolumn} % https://tex.stackexchange.com/questions/2746/aligning-numbers-by-decimal-points-in-table-columns

\usepackage{catchfile} % mjmaddbib needs to read page count file
\pagestyle{fancy}




\newcolumntype{.}[1]{D{.}{.}{#1}}
%\usepackage[maxfloats=30]{morefloats}   %mjm   saving up figs for the end   
% no param on old version stuck at 36
\usepackage{morefloats}        %mjm   saving up figs for the end   
\usepackage{graphicx}        %mjm   saving up figs for the end   

% will need modificaitons 

% 2020-10-18 extract some new boilerplate 

% https://tex.stackexchange.com/questions/121601/automatically-wrap-the-text-in-verbatim
\usepackage{listings}
\lstset{
basicstyle=\small\ttfamily,
columns=flexible,
breaklines=true
}
%%%%%%%%%%%%%%%%%%%%%%%%%%%%% utilitites

\newcommand{\mjmblackbox}[2]{
 \fbox{
% thi does not ing work right 
\begin{minipage}[t]{\textwidth}
{ \centering{\bf{#1 : }} }
\par
#2
\end{minipage}
}
}
\newcommand{\mjmblackboxno}[2]{
 \fbox{
% thi does not ing work right 
\begin{minipage}[t]{\textwidth}
{ \centering{\bf{#1 : }} }
#2
\end{minipage}
}
}





%%%%%%%%%%%%%%%%%%%%%%%% biblio stuff



\newcommand{\checkrel}[1]{%
  \ifcsname#1\endcsname%
\newcommand{\mjmstatus}{  public NOTES }
\newcommand{\mjmversion}{\mjmrelease}
  \else%
\newcommand{\mjmstatus}{ NOT public NOTES }
\newcommand{\mjmversion}{0.00}
  \fi%
}




% https://tex.stackexchange.com/questions/18089/are-there-any-command-for-producing-the-bibtex-logo
%\def\BibTeX{{\rm B\kern-.05em{\sc i\kern-.025em b}\kern-.08em
%    T\kern-.1667em\lower.7ex\hbox{E}\kern-.125emX}}

\newcommand{\biblogo}{
{{\rm B\kern-.05em{\sc i\kern-.025em b}\kern-.08em
    T\kern-.1667em\lower.7ex\hbox{E}\kern-.125emX}}
{ }  }
%\newcommand{\biblogo}{ Bibte{\it X}  { }  }
\newcommand{\latexlogo}{ \LaTeX  { }  }
\newcommand{\bomtexlogo}{ BomTe{\it X}  { }  }






\newcommand{\mjmvirus}{SARS-Cov-2 }
\newcommand{\mjmdisease}{covid-19 }
\newcommand{\Mjmdisease}{Covid-19 }
\newcommand{\mjmlogo}{ MUQED { }  }
\newcommand{\mjmlinkedin}{ {\bf LinkedIn} { }  }








% https://tex.stackexchange.com/questions/121601/automatically-wrap-the-text-in-verbatim
\usepackage{listings}
\lstset{
basicstyle=\small\ttfamily,
columns=flexible,
breaklines=true
}

% none of this fking fking works for a f F 
\newcommand{\mjmverbatim}{lstlisting}
\newcommand{\mjmbeginverbatim}{\begin{lstlisting}}
\newcommand{\mjmendverbatim}{\end{lstlisting}}

\newcommand{\mjmmangle}[1]{keep/#1}



%# CHANGE VERSION AND STATUS MANUALLY 
% need a draft/notes/release flag

% https://tex.stackexchange.com/questions/5894/latex-conditional-expression
%At the command-line, you can do \def\MYFLAG{} and then test if \MYFLAG is defined in your document (or an included style file) with \ifdefined\MYFLAG ... \else ... \fi.
% needs trailing space for the sample bibtex doh
% leading spaces mess up the entry thought 
\def\xxmjmrelease{0.10 }
\ifdefined\mjmrelease
\newcommand{\mjmstatus}{ PUBLIC NOTES }
\newcommand{\mjmversion}{\mjmrelease} %%%%%%%%%%%%%
\newcommand{\mjmtrno}{MJM-2024-010}
\newcommand{\mjmbib}{\mjmtrno-\mjmversion}
\newcommand{\mjmstatuswarn}{{\bf{  }}   }
% 2021-09-29 wanted version wth for brownie
%\newcommand{\mjmbib}{\mjmtrno-\mjmversion-\mjmrelease}
%\newcommand{\mjmbib}{\mjmtrno}
\else
\newcommand{\mjmstatus}{ NOT public NOTES }
\newcommand{\mjmversion}{0.00} %%%%%%%%%%%%%
\newcommand{\mjmtrno}{MJM-2024-010}
%\newcommand{\mjmbib}{\mjmtrno-\mjmversion}
\newcommand{\mjmbib}{\mjmtrno}
%\newcommand{\mjmstatuswarn}{  }
\newcommand{\mjmstatuswarn}{{\bf{This document is a non-public DRAFT and contents may be speculative or undocumented or simple musings and should be read as such.  }}   }
\fi

%\newcommand{\mjmstatus}{ NOT public NOTES }



\newcommand{\mjmtitle}{Utility of Copper Supplementation in Dogs}
\begin{comment}
\newcommand{\mjmmakedate}{2024-04-12 }
\newcommand{\mjmauthor}{Mike J Marchywka }
\newcommand{\mjmbasename}{\jobname}
\newcommand{\mjmaddbio}{mjm_tr,releases}
\newcommand{\mjmversion}{ 0.00 }
\newcommand{\mjmtrno}{MJM-2024-010}
\newcommand{\mjmbibday}{12}
\newcommand{\mjmbibmo}{04}
\newcommand{\mjmbibyear}{2024}
\newcommand{\mjmemail}{marchywka@hotmail.com}
\end{comment}



\ifdefined\mjmlegacy
\newcommand{\expandableinput}[1]{\input{#1}}
\else
% https://tex.stackexchange.com/questions/583927/misplaced-noalign-error-with-input-in-a-table-after-the-2020-fall-latex-releas/583939#583939
\ExplSyntaxOn % providing \expandableinput
\cs_new:Npn \expandableinput #1
  { \use:c { @@input } { \file_full_name:n {#1} } }
\ExplSyntaxOff
\fi



\ifdefined\grad
\else
\newcommand{\grad}{\nabla}
\fi
\newcommand{\mjmquote}[1]{{\centering {\it #1 }}}
\newcommand{\mjmwisdom}[1]{{\centering {\it #1 }}}
\newcommand{\laplace}{\nabla^{2}}
\newcommand{\mjmdx}[2]{\left(\frac{\partial #1 }{\partial #2} \right) }
\newcommand{\mjmdxop}[2]{\frac{\partial  }{\partial #2}\left( #1 \right) }
\newcommand{\mjmdxdx}[2]{\left(\frac{\partial^{2} #1 }{\partial #2^{2}} \right) }
\newcommand{\mjmdxo}[2]{\frac{\partial #1 }{\partial #2} }
\newcommand{\mjmdxx}[2]{\left(\frac{\partial^{2} #1 }{\partial {#2}^{2}} \right) }
\newcommand{\mjmdxy}[3]{\left(\frac{\partial^{2} #1 }{\partial {#2}\partial{#3}} \right) }
\newcommand{\mjmdxyn}[5]{\left(\frac{\partial^{#1} }{\partial^{#2} {#3}\partial^{#4}{#5}} \right) }
\newcommand{\mjmdsq}[2]{\left(\frac{\partial #1 }{\partial #2} \right) ^{2}}
\newcommand{\mjmupdated}[2]{\p  Updated on $#1$ from source #2 \p }
% see tug mail archives, based on discussion 
%\bool_new:N \l_tmpa_bool
\newif\ifbibstarted
\newif\ifbibnamed
\bibstartedfalse
\bibnamedfalse
\newcommand{\mjmtotalbib}{}
%\def\foo{}
\newcommand{\mjmsummabib}[2]{
%\renewcommand{\mjmtotalbib}{ \mjmtotalbib, #1 = #2 }
%\let\foo{ #1 = #2}
}
%%%%%%%%%%%%%%%%%%%%%%%%%%%%%%%%%%%%%%%%%%%%%%%%%%%%%%%%%%%%%%%%

\iffalse
@software{,
  author = {Michael J Marchywka},
  city = {Jasper GA 30143 USA},
  title = { A one-file library for adding machine and human readable bibtex to an article },
abstract={ A simple include file to make bibtex available 
in a document in both machine and human readable format. 
Machine readable is added to extended information 
which can be read with tools such as exiftool ( https://exiftool.org/ ).
While typically not complete at time of pdf creation,
other tools such as toobib can be used to complete the citation
and of course publishers may be able to modify it too as 
more is known. The human readable form need not beincluded
in document types not suited for that but then automated citation
may still be easy.  
See also some exchanges on the Texhax mailing list @tug.org },
institution={},
license={Knowledge sir should be free to all },
publisher={Mike Marchywka},
email={marchywka@hotmail.com},
authorid={orcid.org/0000-0001-9237-455X},
  filename = {mjmaddbib.tex},
  url = {},
  version = {0.0.0},
  date-started = {}
}

<one line to give the program's name and a brief idea of what it does.>


Conceived and written by Mike Marchywka from 2019 to present.
See dates in individual code pieces as they were 
generated from my wizards. 
Copyright (C) <year> <name of author>


This program is free software: you can redistribute it and/or modify it under
the terms of the GNU General Public License as published by the Free Software
Foundation, either version 3 of the License, or (at your option) any later
version.

This program is distributed in the hope that it will be useful, but WITHOUT ANY
WARRANTY; without even the implied warranty of  MERCHANTABILITY or FITNESS FOR
A PARTICULAR PURPOSE. See the GNU General Public License for more details.

You should have received a copy of the GNU General Public License along with
this program.  If not, see <http://www.gnu.org/licenses/>.

   THIS SOFTWARE IS PROVIDED BY THE COPYRIGHT HOLDERS AND CONTRIBUTORS
   "AS IS" AND ANY EXPRESS OR IMPLIED WARRANTIES, INCLUDING, BUT NOT
   LIMITED TO, THE IMPLIED WARRANTIES OF MERCHANTABILITY AND FITNESS FOR
   A PARTICULAR PURPOSE ARE DISCLAIMED.  IN NO EVENT SHALL THE COPYRIGHT OWNER OR
   CONTRIBUTORS BE LIABLE FOR ANY DIRECT, INDIRECT, INCIDENTAL, SPECIAL,
   EXEMPLARY, OR CONSEQUENTIAL DAMAGES (INCLUDING, BUT NOT LIMITED TO,
   PROCUREMENT OF SUBSTITUTE GOODS OR SERVICES; LOSS OF USE, DATA, OR
   PROFITS; OR BUSINESS INTERRUPTION) HOWEVER CAUSED AND ON ANY THEORY OF
   LIABILITY, WHETHER IN CONTRACT, STRICT LIABILITY, OR TORT (INCLUDING
   NEGLIGENCE OR OTHERWISE) ARISING IN ANY WAY OUT OF THE USE OF THIS
   SOFTWARE, EVEN IF ADVISED OF THE POSSIBILITY OF SUCH DAMAGE.


\fi

%\usepackage[pdftex]{graphicx}      % alternative graphics specifications
\usepackage{hyperref}      %


\newcommand{\mjmstartbib}[2]
{
\def\mjmbibentry{@#1\{#2}
\def\mjmbiboneentry{@#1\{#2}
\def\mjmbibpre{x-bib{-}}
\def\mjmday{\day}
%\mjmaddbib{run-day}{\mjmday}
%\mjmaddbib{run-month}{\expandafter\month}
%\mjmaddbib{run-year}{\year}
\mjmaddbib{filename}{\jobname}
\mjmaddbib{run-date}{\today}
}
% modify toobib and change this from "x-bib" to something like bibtex or mjm- whatever 
\newcommand{\mjmbibmunge}[1] {x-bib-#1}

\newcommand{\mjmaddbib}[2]
{
{\hypersetup{pdfinfo={{\mjmbibmunge{#1}}={#2}}}}
\edef\mjmbibentry{\unexpanded\expandafter{\mjmbibentry,}

    #1 =\{#2\} }
%\edef\mjmbiboneentry{\unexpanded\expandafter{\mjmbiboneentry,}
% this works but gives warning and linebreak is removed.
%\edef\mjmbiboneentry{\unexpanded\expandafter{\mjmbiboneentry,\linebreak}
\edef\mjmbiboneentry{\unexpanded\expandafter{\mjmbiboneentry, }
#1 =\{#2\}
}

} % mjmaddbib


\newcommand{\mjmshowbib}
{

\mjmbibentry

\}
} % mjmshowbib

\newcommand{\mjmshowbibone}
{
\mjmbiboneentry \}
} % mjmshowbibone

\newcommand{\mjmdonebib}
{

{\hypersetup{pdfinfo={\mjmbibmunge{bibtex}={\mjmshowbibone}}}}

} % mjmdonebib


% af 
\newcommand{\mjminputlisting}[2]
{
%\begin{figure}[H]
\lstinputlisting{#1}
%\caption{#2}


#2

%\end{figure}
}

%%%%%%%%%%%%%%%%%%%%%%%%%%%%%%%%%%%%%%%%%%%%%%%%%%%%%%%%%%%%%%%%%%

%\newcommand{\mjmaddbib}[2]{\hypersetup{ pdfinfo={ x-bib-#1 = {#2}}}\mjmsummabib{#1}{#2}}

\newcommand{\mjmaddbibonly}[2]{\hypersetup{ pdfinfo={ x-bib-#1 = {#2}}}}
\newcommand{\mjmaddbibe}[2]{\hypersetup{ pdfinfo= x-bib-#1 = #2}}

\newcommand{\mjmtable}[2]{
\begin{table}[H] \centering
\begin{tabular}{#1}
#2
\end{tabular}
\end{table}

}
% for archiving and file list 
% David Carlisle You can use   \textbf{\detokenize{ pmg_ratios.svg}} \IfFileExists{ pmg_ratios.sv}{yes}{no} to make the tokens safe for typesetting in text mode.


\newcommand{\mjmusesitem}[2]{

%\item {\bf $ #1 $  } : #2  % 1 is #1 and 2 is  #2 xxx 
\item {{\detokenize{ #1 } }} : #2  % 1 is #1 and 2 is  #2 xxx 
\IfFileExists{#1}{}{{\bf not found} }

} % mjmuses

\newcommand{\mjmreleasewarning}
{
{\bf This is a draft and has not been peer reviewed or completely proof 
read but released in some state where it seems worthwhile given 
time or other constraints. Typographical errors are quite likely 
particularly in manually entered numbers. This work may 
include  output from software which has not been fully debugged.  
For information only, not for use for any particular purpose see
fuller disclaimers in the text.  Caveat Emptor.}
}

\newcommand{\mjmwarningtoo}
{\bf {This is a draft which may not have been fully proofread and certainly
not peer reviewed. Read the disclaimers and take them seriously.
The reader is assumed familiar with the related literature and
controversial issues. For information and thought only not intended
for any particular purpose. Caveat Emptor  }}

\newcommand{\mjmwarntopic}
{\bf {  This work addresses a controversial topic and likely advances one
or more viewspoints that are not well accepted in an attempt to
resolve confusion.   
The reader is assumed familiar with the related literature and
controversial issues and in any case should seek additional 
input from sources the reader trusts likely with differing opinions. 
For information and thought only not intended
for any particular purpose. Caveat Emptor  }}


\newcommand{\mjmwarnfeed}
{\bf { Note that any item given to a non-human must be checked for safety alone and in combination with other ingredients or medicines  for that animal. Animals including dogs and cats have decreased tolerance for many common ingredients in things meant for human consumption. }}


\newcommand{\mjmwarnme}
{\bf { I am not a veterinarian or a doctor or health care professional
 and this is not particular advice
for any given situation.  Read the disclaimers in the appendicies or text, take them seriously and take prudent steps 
to evaluate this information. 
 }}

\newcommand{\mjmexplainbib}
{{The release may use an experimental bibliography code that
is not designed to achieve a particular format but to
allow multiple links to reference works with modifications
to the query string to allow identification of the citing
work for tracking purposes. This may be useful for a bill-of-materials
and purchases later.
}}








\newcommand{\mjmauthor}{Mike J Marchywka }
\newcommand{\mjmmakedate}{2024-04-12 }
\newcommand{\mjmbasename}{\jobname}
\newcommand{\mjmaddbio}{mjm_tr,releases}
%\newcommand{\mjmversion}{0.00}
%\newcommand{\mjmtrno}{MJM-2024-010}
%\newcommand{\mjmbibday}{12}
%\newcommand{\mjmbibmo}{04}
%\newcommand{\mjmbibyear}{2024}

% the build script changes these to creation day doh 
\newcommand{\mjmbibday}{12}
\newcommand{\mjmbibmo}{04}
\newcommand{\mjmbibyear}{2024}


\newcommand{\mjmmakebibday}{\number\day}
\newcommand{\mjmmakebibmo}{\number\month}
\newcommand{\mjmmakebibyear}{\number\year}

\newcommand{\mjmbibtype}{techreport}

\newcommand{\mjmbibname}{marchywka-\mjmbib}
\mjmstartbib{\mjmbibtype}{\mjmbibname}


\newcommand{\mjmemail}{marchywka@hotmail.com}
%\newcommand{\mjmaddr}{44 Crosscreek Trail Jasper GA 30143 USA}
\newcommand{\mjmaddr}{44 Crosscreek Trail Jasper GA 30143 USA}
\mjmaddbib{title}{\mjmtitle}
\mjmaddbib{author}{\mjmauthor}
\mjmaddbib{type}{\mjmbibtype}
%\mjmaddbib{name}{marchywka-\mjmbib}
\mjmaddbib{name}{\mjmbibname}
\mjmaddbib{number}{\mjmtrno}
\mjmaddbib{version}{\mjmversion}
\mjmaddbib{institution}{not institutionalized, independent }
\mjmaddbib{address}{ \mjmaddr}
\mjmaddbib{date}{\today}
\mjmaddbib{startdate}{\mjmbibyear -\mjmbibmo -\mjmbibday }
%\mjmaddbib{day}{\mjmbibday}
%\mjmaddbib{month}{\mjmbibmo}
%\mjmaddbib{year}{\mjmbibyear}
\mjmaddbib{day}{\mjmmakebibday}
\mjmaddbib{month}{\mjmmakebibmo}
\mjmaddbib{year}{\mjmmakebibyear}

\mjmaddbib{author1email}{\mjmemail}
\mjmaddbib{contact}{\mjmemail}
\mjmaddbib{author1id}{orcid.org/0000-0001-9237-455X}
\CatchFileEdef\mjmpages{\mjmbasename.last_page}{\endlinechar=-1\relax}
% TODO FIXME add this to the skeleton text 
%\mjmaddbib{pages}{ \input{\mjmbasename.last_page}}
\mjmaddbib{pages}{ \mjmpages}
%\mjmaddbib{filename}{\mjmbasename}

\begin{comment}
\newcommand{\checkfortwo}[1]{%
  \ifcsname#1\endcsname%
  VERSION =        \{\mjmversion \today \mjmstatus \},
  \else%
  VERSION =        \{\mjmversion\},
  \fi%
}

\end{comment}
%\mjmaddbib{bibtex}{\mjmfullbib}

\mjmdonebib



\lhead{\mjmauthor,  \mjmtrno }


%\lhead{M Marchywka,  \mjmtrno }
%\rhead{ \mjmversion not for public release}
%\rhead{ { \today }  v. \mjmversion for release without review }
%\rhead{ { \today }  v. \mjmversion NOT public DRAFT }
%\rhead{ { \today }  v. \mjmversion { }  NOT public NOTES }
\rhead{ { \today }  v. \mjmversion { }  \mjmstatus }

\newfloatcommand{capbtabbox}{table}[][\FBwidth]


%%%% build flags 
%\newlength{\desttabw}  \setlength{\desttabw}{4in}
\newlength{\desttabw}  \setlength{\desttabw}{\textwidth}
\newlength{\chainwidth}  \setlength{\chainwidth}{.4\textwidth }
\newlength{\slantwidth}  \setlength{\slantwidth}{.2\textwidth }
\newlength{\subfigwidth}  \setlength{\subfigwidth}{.3\textwidth }
\newlength{\fullfigwidth}  \setlength{\fullfigwidth}{.8\textwidth }
\newlength{\subwfigwidth}  \setlength{\subwfigwidth}{.75\textwidth }
\newlength{\subwfigwidthrot}  \setlength{\subwfigwidthrot}{\textwidth }
\newlength{\myboxwidth}  \setlength{\myboxwidth}{.3\textwidth }
\newlength{\picwidth}  \setlength{\picwidth}{.4\textwidth }
% set to center for nowmal output 
\newcommand{\destflushtab}{flushleft}
\includecomment{mdpicomment}
\excludecomment{draftcomment}
\excludecomment{badmathcomment}
\excludecomment{showworkcomment}
% this does not fing work ... 

\newcommand{\mjmed}[1]{
%\begin{mjmedx} 
[ mjm : #1   ]
%\end{mjmedx} 
}  

% thinking outload
\newcommand{\mjmtolx}[1]{}
\newcommand{\mjmtolxx}[1]{}
\newcommand{\mjmtol}[1]{
 \fbox{  
% thi does not ing work right 
\begin{minipage}[t]{\textwidth}
{ \centering{\bf{Thinking outloud}} }
\par   
#1 
\end{minipage} 
}
}

\newcommand{\mjmpicture}[3]
{
\begin{figure}[H]
{ \includegraphics[height=3in,width=4in]{keep/#1} }
\caption{#2}
\label{fig:#3}
\end{figure}
} % mjmpicture


\newcommand{\mjmaside}[1]{
 \fbox{  
% thi does not ing work right 
\begin{minipage}[t]{\textwidth}
{ \centering{\bf{Aside: }} }
\par   
#1 
\end{minipage} 
}
}




\newcommand{\mjmgraphics}[1]{#1 }
\newcommand{\mjmfullplot}[1]{\includegraphics[width=\fullfigwidth]{keep/#1}}
%\newcommand{\mjmincludeplot}[1]{\includegraphics[width=\fullfigwidth]{#1}}
%\newcommand{\mjmincludeplot}[1]{\includegraphics[width=\subfigwidth]{#1}}
\newcommand{\mjmincludeplot}[1]{\includegraphics[height=3in,width=\fullfigwidth]{#1}}

% include here as likely to be doc specific 
%\newcommand{\mjmreffig}[1]{Fig. \ref{#1}}
\newcommand{\mjmreffig}[1]{Fig. \ref{fig:#1}}
\newcommand{\mjmreftab}[1]{Table  \ref{tab:#1}}


%cp yyy2.pdf ~/d/latex/keep/pp20171124biotin.pdf
\newcommand{\mjmdatedplot}[1] 
{ \includegraphics[height=3in,width=\fullfigwidth]{keep/pp20171124#1} }
% right now there are too many figs to save for the end apaprently
% hard limit is 36
\newcommand{\mjmbeginfigure}{\begin{figure}[H] }
%\newcommand{\mjmbeginfigure}{\begin{figure}[p] }
\newcommand{\mjmfigure}[1]{
%\begin{figure}[H]
\mjmbeginfigure

#1 

\end{figure}
}
%\extrafloats{100}

\newcommand{\mjmlisting}[1]
{
\begin{lstlisting} 
#1 
\end{lstlisting}
}

\newcommand{\mjmold}[1]{ } 



\newcommand{\mjmeqn}[1]{\begin{equation} #1 \end{equation}  } 





\begin{document}

\title{\mjmtitle}
\author         {Mike Marchywka}
\email          {\mjmemail}
\thanks{ to cite  or credit this work, see bibtex in \ref{appendix:citing} } 
\date{\today}
\affiliation{\mjmaddr}

\mjmblackboxno{Release Notes  xxxx-xx-xx : }{
Copper was an early part of my interest in optimization of 
supplements for dogs and humans. Recent literature has expressed
concern about copper so I thought I would get out generally
supportive results to date although omitting much of my own
personal experiences ( I'm a human not a dog ) that seem similarly
beneficial. It seems that often the popular press led by science
catches onto incomplete or "close but not quite" ideas and reversals
in recommendations are common. Curious to see how attitudes towards
copper evolve. It may be worth noting there seems to be a trend
to get away from copper plumbing lol.  

{\bf ToDo : } Known problems: no refs yet, diettables have unit problems for recent
noun additions


\mjmreleasewarning

\mjmstatuswarn

\mjmwarnfeed

\mjmwarnme

\mjmwarntopic

\mjmexplainbib

}

\begin{abstract}
Recent popular news items have suggested a problem with hepatic 
copper accumulation in dogs thought to relate to increased copper
content in some foods. However, that may reflect frustrated attempts
to deliver more copper to the heart or other organs as those
organs signal to absorb more copper with the intent to deliver
it where needed. Consideration of homeostatic mechanisms,
feedback loops, is often neglected and that may be the case
in the present condundrum.  
This work describes copper supplementation to a group of dogs
over several years with no robust deletious effect eastablished
although some suspicious observations are discussed.
Some benefits associated with copper supplementation include 
reduced coughing likely due to infectious
respiratory disease, collapsed trachea,, or dilated / hypertrophic
heart. 
Copper in these dogs may be beneficial through accumulation
in macrophages and other locations, use by lysyl oxidase to
stiffen trachea and other structural organs, and for
mitochondrial energy production notably by the heart leading to greater
volumetric efficiency. 
Particular nutrients that may aid transport out of the liver
would likely be those which enhance ceruloplasmin quantity or quality
or othewise modify copper handling. Tryptophan and tyrosine are
two candidates for being copper toxicity limiting and were generally
supplemented in this group of dogs. Both amino acids have
unique functions and distribution may be modified by many
factors including GI health to overall food chemistry and
microbial metabolism.  

\end{abstract}

\maketitle
\tableofcontents
\newpage


\section{Introduction  }
Copper has become a concern in dog food over the past few years
due to the more common
observation of "copper associated hepatitis"
\cite{PMC10787350}
\cite{Center_Richter_Twedt_time_2021}.
Copper homeostasis is a much larger issue including in human
health with regard to such unresolved diseases as 
Alzheimer's where the decades of work on amyloid beta
is becoming more clearly futile. 


Over the  same time however, concerns about diet linked
DCM in dogs have emerged. 

Its possible the two concerns are related in more copper
is being absorbed as less is  transported to target organs
such as the heart.

To help with this apparent conundrum, this work describes
variable copper supplementation to a group of dogs
over several years including one pregnant pit bull with
uterine fibroids. Generally beneficial results 
were associated with copper supplementation in the context
of broader rationally designed supplements.
Apparent benefits to a group of puppies included infection 
control. Additional respiratory infections were
thought to be modulated in older dogs described
here as Cookie( AKA Mixie ) and Trixie. It may have
reduced transmission to the larger group in the latter case.
Also an association with likely non-infectious coughing
was seen in the case of Happy. 

Given the varied canine genetics and known copper related
diseases, vigiliance for adverse reactions was maintained
but to date only questionable events, such as reduced
appetite, remain. 

Copper and vitamin K have both seen literature suggesting
a role for liver health under some conditions. Vitamin K is
note worthy because of many efforts to antagonitze
its effects similar to the present  concerns with copper. 

 
These cases are described in more detail with the hope of
sorting out cause and effect between diet and clinical
outcomes as fixation on one nutrient at  a time may not
be productive. 

\section{Cases and Observations}

A series of rescue dogs were fed food and vitamin supplements
in addition to commercial kibble products. Diet and outcomes
were recorded after supplemented meals in MUQED format.
Most dogs receieved additional meals of commercial dog food
and unfortuntately uncontrolled scraps or treats while others
routinely ate toys or yard debris. However, some 
results appear to relate to the vitamin mix and notably
this includes copper. 


\begin{table}[H] \centering
\begin{tabular}{|l|r|l|r|r|r|}
%\multicolumn{6}{|c|}{Title}\\
\hline
Dog & \multicolumn{1}{|c|}{Dates} & Condition & weight(lbs) & Cu& Outcomes \\
\hline
Cookie &21-09-10 22-01-21&Resp infection/azithromycin &13.5&&cleared \\
Happy  &18-09-07 24-04-10&several&13.4 - 17.7&& \\
Happy  &18-09-07 19-05-30&heartworm/doxycycline&13.4 - 17.2&&cough gone \\
Happy  &24-03-26 &coughs&15.2 - 15.5&&rare coughing \\
Brownie  & 21-01-12 23-02-22&pregnant , fibroids &49 - 64&& uneventful  \\
puppies  &21-03-23 21-06-09&cough&104&&cleared \\
Trixie  &23-12-16 24-04-10&resp infection/Clavamox&37.6 44.6&&cleared \\
Rocky  &22-02-05 24-04-10&&4.4 8.3 &&subjective better  \\
Hershey  &17-04-22 19-08-27&multiple&8.2 9 &&heart failure  \\
\hline
\hline
\end{tabular}
\caption{List of dogs most effected by copper supplementation. The puppies were born on 2021-02-14 but only recorded as weaning began. Puppie weight reflects total as they were placed elsewhere and food shares are unknown   }
\label{tab:dogs}
\end{table}

\subsection{Cookie or Mixie}
Arrived with diagnosed respiratory infection and azirhtromycin.
Copper and other nutirents were added and eventually infection
resolved well.
\subsection{Brownie and puppies}
Brownie was determined to be pregnant shortly after arrival
and her diet may be notable for includsion of both 
copper and vitamin K. Other conidtions indluce heartworm
positive, treated after weaning  with Diroban, and fibroids
removed 2021-11-15 well after puppies were gone. She was
unevntful until being doganosed with cancer and killed
2023-02-22. 
\subsection{Happy}
Happy arrived heartworm positive coughing to varying degrees.
She was treaed with a slow kill approach including ivermectin
and doxycycline as previously described. She later was acting
sick but appeared to recover well with B vitamin supplements.
But her coughing never returned to the very low levels
seen after heart worm recovery until copper doses were increased
with elimination of any zinc and care with tryptophan.

\subsection{Trixie}
Trixie began coughing shortly after arrival and was
very low energy. Many other dogs began to cough or
hack suggesting that she brought a communicable infectious
disease. Nutrient mix was modified to add more copper and
most dogs coughing returned to normal quanity and qualitiy
although her's did not resolve. Copper stopped for a couple
day ( I was gone ) and owner took he to the vet as she
began coughing more. Clavamox was prescribed and her coughing
stopped within a few days. Her energy level has improved
but she still does not run.  

\subsection{Rocky}
Rocky will hopefully be the subject of another work as
he responded significantly to iodine and sodium benzoate
which was attributed to, but never lab confirmed, low
thyroid output. His "plastic" body type changed into
a more normal "flexible" type and he began to feel like 
the other dogs when picked up rather than stiff.
The addition of copper may have reduced his morning cough
but he continued to have apparent congestion after eating
sometimes breathing through his mouth and sneezing.
Most recently he had notable muscle tone which had been lacking.
His overall activity increased but that may be due to social factors
such as feeding ritual. 





\subsection{Miscellaneous Observations}
Rocky seemed to do better but was also responding
to benzoate and iodine. 

In initial attempts to formulate a vitamin mix,
copper was added but no zinc. There was some possible
feeding hesitance that went away when zinc was added.
However a causal link was not established although 
copper was moderated afterwards. Annie may lose
some appetite with excessive copper. However, a
causal link was not eastablished. 


\begin{figure}[htb] 
\centering
\mjmfullplot{happy2seeyou.jpg}
\caption{  Copper(white) and Zinc(yellow) dosing per day averaged over 
prior 10 day period as dosing was highly variable due to rotations
of various nutrients. 
18046 is 2019-05-30 when the cough was first noted to be gone for a few weeks. 19823 is  2024-04-10 the last date for which data was obtained. The cough stopped prior to the start of the Zinc and gradually increased to a notable background level over most of this interval although notes were incomplete. 
19531 2023-06-23 notes the start of Cu depletion and chronic cough was noted by late Fall. During this time Zinc greatly exceeded copper dosing.  }
\end{figure}


\begin{figure}[htb] 
\centering
\mjmfullplot{trixie_cu.jpg}
\caption{   Trixie copper consumption since arrival. Daily amounts
( white)) and trailing 10 day average( yellow). Copper started to
be significant around day 19730 in response to coughing. Day 19807
marked the end of the copper fast as well as the end of Clavamox
which was prescribed due to worsening when copper stopped days
earlier. 
  }
\end{figure}


\begin{figure}[htb] 
\centering
\mjmfullplot{hershey_cu_i.jpg}
\caption{    Herhsey 10 day trailing average copper (white )
and iodine (yellow). 
  }
\end{figure}


\section{Discussion }

Several likely benefits of copper suypplementation were observed
but no clear robust clinical symptoms got worse.
This is contrary to some indications from populat concerns
about excessive copper in commercial dog foods. 
Copper use requires uptake and transportation to 
various targets. Transport out of the liver can be hindered
for reasons such as ceruloplasmin defects.


ver a broad range of genetics, its likely that copper intake
can be raised as long as other nutirents are also given
to handle the copper beneficially.
Candidate nutrients include tryrosine and tryptophan.

Copper in these dogs may be beneficial through accumulation
in macrophages and other locations, use by lysyl oxidase to
stiffen trachea and other structural organs, and for
energy production notably by the heart leading to greater
volumetric efficiency. 



Some precednce for metal modulated toxicity existed back to 1999
when work with cultured neurons showed a dose dependent
reduction in abeta toxicity with Zn \cite{Lovell_Xie_Markesbery_Protection_against_amyloid_beta_1999}.
By 2005 toxicity of amyloid beta and the metals zinc, iron, and
copper was investigated under conditions that created more toxicity
with iron and zinc but not copper while amyloid beta reduced metal
toxicity in rats
\cite{Bishop_Robinson_Amyloid_Paradox_AmyloidupbetaMetal_2004}.

Copper signalling is such that remote signals may exist from the heart
to liver and intestines to make more available
\cite{xxx_Mechanism_regulation_2010}. In this scenario, local
shortage could induce blood stream excess due to added
inputs with struggling cardiac specific uptake as
has been suggested for other nutrients such as tryptophan
and biotin.  Note this work also suggests copper deficiency
as an issue for cardiac hypertrophy in animals.
Copper uptake may depend on anions such as chloride
at least in some animals
\cite{Handy_Musonda_Phillips_Mechanisms_Gastrointestinal_Copper_2000}
,suggesting GI chloride per se rather than pH may be an  issue.
A series of copper deficient liver patients were notable for
"steatohepatitis, iron overload, malnutrition, and recurrent infections. "
\cite{Yu_Liou_Biggins_Copper_Deficiency_Liver_2019} .
Its interesting that iron overload occurs along with general malnutrition
and sepcific copper deficiency.

\mjmtol{ this may not belong here but relevant to
other Cu stuff,
A recently published work suggests copper delivery is the
important part of a new ALS drug but the work also suggests
a "hyperreductive state" around hypoxic mito that
promote relase of Cu from the drug complex
\cite{Hilton_Kysenius_Liddell_Evidence_disrupted_copper_2024}/
pointing to a possible more general mechnism.
The work goes onto suggest possible role in Parkinson's Disease
but does not address AD.
}
In 2021, Ni was found in important amounts in a commercial
abeta40 preparation
\cite{Benoit_Maier_nickel_chelator_dimethylglyoxime_2021}
and was found to mediate dityrosine crosslinks
\cite{Berntsson_Vosough_Svantesson_Residue_specific_binding_2023}
similar to the dityrosine crosslinks induced by copper
found in 2004
\cite{Atwood_Perry_Zeng_Copper_Mediates_Dityrosine_Cross_2004}.

A 2013 work found in vitro physiological conditions
caused copper to prevent fibril formation
\cite{Mold_OuroGnao_Wieckowski_Copper_prevents_amyloid_upbeta1_2013}.

Copper is essential for many growth processes and can activate
receptor tyrosine kinases without a ligand making it
a target for cancer \cite{PMC6537018}.

Rats fed a copper deficient diet shows neurological symptoms
by 7 weeks and had reduced tyrosine hydroxylase and
SOD activity ZZ
\cite{Morgan_ODell_EFFECT_COPPER_DEFICIENCY_1977}.

A 2017 study explored the effects of copper and vitamin C
as well as other molecules such as
clioquninol on abeta and in vitro neurons suggesting
abeta could be cleaved by copper in the presence of oxygen
as well as an anti-oxidant such as vitamin C although
restoration of neuronal functioning was only partial
\cite{C7SC01787A}.
Interestingly, copper-ascorbate oxidation of tryptophan may be
suppressed by Trp chelation of copper at high trp concentrations
\cite{PMID27406076}
suggesting reduced amounts may give copper more ability to damage
an already low supply.
This is interesting in terms of a utrient interaction hypothesis
on copper toxicity.
And in fact as early as 2012 it was determined that
tryptophan intake could reduce copper toxicity at least in carp
\cite{Hoseini_Hosseini_Soudagar_Dietary_tryptophan_changes_serum_2012}.

Body stores of copper increases with excess tyrisone in the diet
of rats
\cite{Yang_Noda_Kato_Elevated_Intestinal_Absorption_}.

By 2022,  work focusing on moving copper into the cell considered
many aspects of copper misallocation and devised a copper specific
shuttle peptide \cite{D2SC02593K}.
Recognition that the cells need copper is important .

A 2016 study in mice suggested adding copper to water was
worse than adding it to food and supplementation at 6,15, and 30 ppm
with increases in soluble abeta and decreased growth rate and
GSH/SOD activity
\cite{Wu_Han_Gong_effect_copper_2016}.
With a high dose of about 100 micrgograms/day ( from CuSO4) and
a body weight of about 30grams, the dose was about 3.3mg/kg.


One work in  2022 addressed
AD as a consequence of copper deficnecy because
\cite{Klevay_contemporaneous_epidemic_2022}
\begin{quote}
It is hypothesised that copper deficiency is a plausible cause of Alzheimer's
disease(Reference Klevay84). Patients are thinner than normal; weight loss
precedes dementia and is associated with greater dementia and neurobehavioural
symptoms. Nutritional compromise contributes to morbidity. Cytochrome oxidase
depends on copper for activity; at least fourteen publications reveal decreased
activity in brain of Alzheimer's patients. Brain copper and caeruloplasmin
also are decreased. This hypothesis is the only one that explains why Alzheimer's
disease occurs earlier and is more common in Down's syndrome. Superoxide
dismutase (SOD1) depends on copper for activity; its gene is on chromosome
21. This enzyme is elevated in Down's syndrome (trisomy 21) and is decreased
in people with monosomy. It seems likely that people with Down's syndrome
have a higher than average requirement for dietary copper because copper
is incorporated into superoxide dismutase and is unavailable for other uses.
Thus, Alzheimer's disease fulfills the first two of Golden's criteria (above)
for deficiency.

\end{quote}

Lysyl oxidase bad for vessels \cite{PMID38402026} calxification.
but may  be related to metallization issues \cite{PMID9524359}.

Ceruloplasmin contains a chain  of W and Y that are thought important
for enzyme preservation \cite{Tian_Jones_Solomon_Role_Tyrosine_2020}.
As iron accumulation is related to AD, there is a question about
the quality of the circulating ceruloplasmin.  If there is high-infidelity
translation due to W and Y depletion, there is also the question of
how feedback mechanisms control the overall amount.
Ceruloplasmin KO mice gained weight and showed increased
scatter in weight with lipid dysregulation only partially corrected
wth exogenous replacement
\cite{Raia_Conti_Zanardi_Ceruloplasmin_Deficient_Mice_Show_2023}.

Deficiency seems to effect prefernetially proteins involved
in neuronal projection and diabetes and iron handling
\cite{PMID36583241}.

Copper may antagonize many pathogens including
H pylori
\cite{Bernegger_Brunner_VizoviUx0161badcharvv353ek_novel_FRET_peptide_2020}
and clostridum


Combined with vitamin C literature is confusing. It may be bad
\cite{Jiang_Sui_Hong_Combined_Administration_2023}
although alternatives with copper gluconate instead of
sulfate 

Pulmonary hypertension may be controlled by serotonin
\cite{PMID27927914} and therefore tryptophan intake. 


\section{Limitations}
While the other components were mentioned as important,
it needs to be reiterated that the 
the other snack components could have effected copper handling
significantly and supplementation with another diet lacking
these components may not be beneficiail but copper restriction
may not be either. Most food ingredient interact with matals 
to varying degrees and this notably contained citric acid
and spinach along with amino acids. 

The residential setting made it difficult to control or monitor
all of the factors which could effect health. Besides the main
kibble meals not being recorded for some dogs, intake of food
and foriegn objects was common and unpredictable. 
Supplement quantities were often measured by volume using kitchen
utensils known to be poorly calibrated. 
Completely unknown experiences or factors may be involved in their
subjective behaviors.  Cigarette smoke exposure was common
but variable.
As is always the case, despite MUQED's ability to keep strcutured
outcome notes on things like cough, the resulting outcome
data was very sparse and relies on memory in some cases.
The lesson remains that notes and data always need to be
more complete. 



\section{Conclusions}
Copper has to be suspected of being important in dogs for 
functions that likely
include strengthening  of structural elements such as the trachea,
volumetric energy effeiciency of the heart, and infection control.
In the GI tract, it may moderate pathogenic phenotypes and
change community structure of microbiome. 
Accumulation in the liver may reflect export problems
rather than too much intake as signalling exists to
regulate uptake and disposal. Defects may be due to other nutrients
and particularly anything that interfers with ceruloplasmin 
synthesis or quality. 

Internal transport and uptake however may both rely on 
GI defects which limit nutrient avialbility. Low stomach
acid may be one common problem. 

Zinc excess may also interfer with copper
deployment.
Dog genetics are varied and specifics likely vary too.
Similar considerations may apply to humans. 

Liver pathology that includes atypical amounts of copper
may not reflect excess dietary intake but some other problem
that needs to be fixed. 

\section{Supplemental Information}

Dog diet data files are available online at
{\url{https://github.com/mmarchywka/dogdata}}
or other locations as may be required.
The author may also be contacted if onlines sources are not
avialble. Raw MUQED format as well as parsed text formats
are avilable although MUQED software availbility is in the works.


\subsection{Computer Code}
note anything using "snacks\_Collated.ssv" is obsolete as it messed
up adjectives etc. use "linc\_graph -dt-mo"
NB : the "datealias" entries need to be updated not just datemin and datemax
and the latter may not even do anything lol.  A note also
"reporting units" for many new nouns are not right as tsp
 has replaced mg etc. 

\begin{lstlisting}


2766  ./run_linc_graph -dt-mo txt/happy2cu.txt 
 2767  texfrag -include xxxtable 
 2768  mv xxxtable /home/documents/latex/proj/copper/keep/monthly.tex



\end{lstlisting}
\section{Bibliography}


\bibliography{\mjmbasename,\mjmaddbio}
\bibliographystyle{plainurl}


%%%%%%%%%%%%%%%%%%%%%%%%%%%%%%%%%%%%%%%%%%%%%%%%%%%%%%%%%%%%%%%%%%%%%%%%%%%%%
\begin{acknowledgments} 

% \input{generalack.tex}
\begin{enumerate}
\item Pubmed eutils facilities and the basic research it provides. 
\item Free software including Linux, R, LaTex  etc.
\item Thanks everyone who contributed incidental support. 
\end{enumerate}

\end{acknowledgments}

%%%%%%%%%%%%%%%%%%%%%%%%%%%%%%%%%%%%%%%%%%%%%%%%%%%%%%%%%%%%%%%%%%%%%%%%%%%%%
\clearpage
\appendix

\begin{mdpicomment}

\section{ Statement of Conflicts }
 No specific funding was used in this effort and there are no relationships
with others that could create a conflict of interest. I would like to develop
these ideas further and have obvious bias towards making them appear 
successful. Barbara Cade, the dog owner, has worked in the pet food industry
but this does not likely create a conflict. We have no interest in the
makers of any of the products named in this work.  

\end{mdpicomment}

\begin{mdpicomment}
\section{About the Authors and Facility}
This work was performed at a dog rescue run by Barbara Cade and
housed in rural Georgia.  The author of this report 
,Mike Marchywka,
has a background in electrical engineering and 
has done extensive research using free online literature sources.  
I hope to find additional people interested in critically 
examining the results and verify that they can be reproduced
effectively to treat other dogs.

\begin{comment}
\begin{figure}[htb] 
\centering
\mjmed{ picture commented out to save space in drafts...  } 
%\includegraphics[width=\picwidth]{me_on_brick.jpg}
\caption{ 
 }
\end{figure}

\end{comment}

\section{Background Diet Sumnary}

\newcommand{\mjmdatemin}{2023-10-01}
\newcommand{\mjmdatemax}{2024-04-10}
\newcommand{\mjmsuperscripts}{{\bf a) } SMVT substrate. Biotin, Pantothenate, Lipoic Acid, and Iodine known to compete..{\bf c) } hamburger with varying fat percentages- 7,10,15,20, etc. ..}
\begin{table}[H]
\centering
\begin{tabular}{|l|r|r|r|r|r|}
\hline
Name&2023-10 Oct&2023-11 Nov&2023-12 Dec&2024-01 Jan&2024-02 Feb\\
\hline
{\bf FOOD}&&&&&\\
$\textrm{KCl(tsp~kcl)}$&0.045 ;0.031;23/23&0.047 ;0.031;30/30&0.085 ;0.062;24/24&0.094 ;0.062;31/31&0.093 ;0.062;29/29\\
$\textrm{KibbleAmJrLaPo}$&0.036 ;0.037;22/23&0.065 ;0.075;30/30&0.07 ;0.075;23/24&0.075 ;0.075;31/31&0.071 ;0.098;29/29\\
$\textrm{KibbleLogic}$&0.024 ;0.025;22/23&0.043 ;0.05;30/30&0.047 ;0.05;23/24&0.05 ;0.05;31/31&0.047 ;0.065;29/29\\
$\textrm{b10ngnc}^{\left(c\right)}$&0.019 ;0.25;1/23&0.11 ;0.25;9/30&0.047 ;0.25;3/24&0.11 ;1;7/31&0.067 ;0.25;5/29\\
$\textrm{b15ngnc}^{\left(c\right)}$&&0.044 ;0.25;5/30&0.021 ;0.25;1/24&0.06 ;0.25;4/31&\\
$\textrm{b20ngnc}^{\left(c\right)}$&0.18 ;0.25;14/23&0.13 ;0.25;10/30&0.25 ;0.25;14/24&0.14 ;0.25;11/31&0.28 ;0.25;19/29\\
$\textrm{b25ngnc}$&0.11 ;0.25;9/23&0.067 ;0.25;6/30&0.026 ;0.25;2/24&0.02 ;0.25;2/31&0.039 ;0.25;4/29\\
$\textrm{b7ngnc}^{\left(c\right)}$&0.1 ;0.25;8/23&0.14 ;0.25;11/30&0.14 ;0.25;9/24&0.2 ;0.25;17/31&0.11 ;0.25;7/29\\
$\textrm{blackberry}$&&0.058 ;0.25;5/30&0.3 ;0.25;20/24&&\\
$\textrm{blueberry}$&2.4 ;3.8;23/23&2.4 ;2.2;30/30&1.9 ;2;20/24&0.71 ;1.5;13/31&1.2 ;1.5;29/29\\
$\textrm{carrot}$&0.35 ;0.25;23/23&0.36 ;0.25;30/30&0.36 ;0.25;24/24&0.38 ;0.25;31/31&0.38 ;0.25;29/29\\
$\textrm{cbbrothbs}$&&&&&0.022 ;0.25;3/29\\
$\textrm{cbbroth}$&0.16 ;0.25;10/23&0.071 ;0.25;6/30&&0.21 ;0.25;15/31&0.25 ;0.25;16/29\\
$\textrm{citrate(tsp~citrate)}$&0.045 ;0.031;23/23&0.047 ;0.031;30/30&0.048 ;0.062;24/24&0.058 ;0.062;31/31&0.092 ;0.062;29/29\\
$\textrm{ctbrothbs}$&0.082 ;0.25;5/23&0.4 ;0.25;25/30&0.48 ;0.25;24/24&0.29 ;0.25;19/31&0.22 ;0.25;14/29\\
$\textrm{ctbroth}$&0.17 ;0.25;11/23&&&0.032 ;1;1/31&\\
$\textrm{eggo3}$&0.065 ;0.12;23/23&0.062 ;0.062;30/30&0.055 ;0.12;20/24&0.062 ;0.062;31/31&0.062 ;0.062;29/29\\
$\textrm{eggo}$&&&0.01 ;0.062;4/24&&\\
$\textrm{eggshell}$&0.13 ;0.25;23/23&0.12 ;0.12;30/30&0.11 ;0.25;21/24&&\\
$\textrm{garlic}$&0.022 ;0.25;2/23&0.22 ;0.25;26/30&0.083 ;0.25;8/24&1.2 ;1;27/31&0.99 ;1;22/29\\
$\textrm{marrow}$&0.19 ;0.25;12/23&0.37 ;0.25;30/30&0.083 ;0.25;6/24&&0.078 ;0.25;7/29\\
$\textrm{oliveoil(tsp)}$&0.035 ;0.12;8/23&0.014 ;0.12;4/30&&&0.039 ;0.12;9/29\\
$\textrm{pepper}$&0.36 ;0.25;23/23&0.38 ;0.25;30/30&0.35 ;0.25;24/24&0.36 ;0.25;31/31&0.38 ;0.25;29/29\\
$\textrm{pineapple}$&&&0.021 ;0.25;2/24&&\\
$\textrm{raspberry}$&0.32 ;0.25;23/23&0.28 ;0.25;24/30&&&\\
$\textrm{salmon}$&&0.043 ;0.25;8/30&&0.025 ;0.25;3/31&\\
$\textrm{shrimp(grams)}$&&3 ;38;5/30&4.9 ;16;9/24&2.8 ;16;8/31&1.8 ;13;4/29\\
$\textrm{spinach}$&&0.15 ;0.25;12/30&0.36 ;0.25;24/24&0.38 ;0.25;31/31&0.36 ;0.25;28/29\\
$\textrm{sunflowerseed}$&0.23 ;0.25;21/23&0.25 ;0.25;30/30&0.21 ;0.25;20/24&&0.034 ;0.25;4/29\\
$\textrm{tomato}$&0.36 ;0.25;23/23&0.23 ;0.25;19/30&0.18 ;0.25;12/24&0.17 ;0.25;15/31&0.19 ;0.25;29/29\\
$\textrm{tuna(oz)}$&&&&&\\
$\textrm{turkey}$&0.34 ;0.25;23/23&0.37 ;0.25;30/30&0.35 ;0.25;24/24&0.36 ;0.25;31/31&0.36 ;0.25;29/29\\
$\textrm{vinegar(tsp)}$&0.09 ;0.062;23/23&0.094 ;0.062;30/30&0.09 ;0.062;24/24&0.068 ;0.062;24/31&2.16e-03 ;0.062;1/29\\
{\bf VITAMIN}&&&&&\\
$\textrm{B-1(mg)}$&4.09e-03 ;0.012;15/23&5.87e-03 ;0.0059;30/30&6.12e-03 ;0.012;24/24&5.69e-03 ;0.0059;30/31&5.87e-03 ;0.0059;29/29\\
$\textrm{B-12(mg)}$&0.033 ;0.25;5/23&0.029 ;0.25;5/30&0.047 ;0.25;6/24&0.024 ;0.25;5/31&0.034 ;0.12;8/29\\
$\textrm{B-2(mg)}$&5.7 ;16;15/23&7.9 ;8.1;29/30&8.1 ;16;24/24&21 ;32;30/31&43 ;65;29/29\\
$\textrm{B-3(mg)}$&8.3 ;24;15/23&12 ;12;30/30&12 ;24;23/24&31 ;48;30/31&60 ;48;29/29\\
$\textrm{B-6(mg)}$&6 ;12;11/23&12 ;12;28/30&11 ;12;21/24&8.9 ;12;29/31&5.8 ;12;26/29\\
$\textrm{B-multi(count)}$&0.022 ;0.062;8/23&&&2.02e-03 ;0.062;1/31&\\
$\textrm{Cu(mg)}$&0.11 ;0.25;10/23&0.76 ;2;19/30&0.86 ;2;19/24&1.9 ;2;30/31&1.9 ;2;28/29\\
$\textrm{D-3(iu)}$&91 ;300;7/23&60 ;300;6/30&62 ;300;5/24&58 ;300;6/31&52 ;300;5/29\\
$\textrm{Iodine(mg)}^{\left(a\right)}$&2.3 ;12;8/23&0.1 ;0.78;4/30&0.065 ;0.78;2/24&0.1 ;0.78;4/31&0.13 ;0.78;5/29\\
$\textrm{K1(mg)}$&0.38 ;1.2;7/23&0.92 ;1.2;22/30&1.1 ;1.2;22/24&1.1 ;1.2;27/31&1.2 ;1.2;28/29\\
$\textrm{K2(mg)}$&1 ;1.6;15/23&0.3 ;1.9;7/30&0.47 ;3.8;3/24&0.91 ;3.8;8/31&0.81 ;3.8;8/29\\
$\textrm{K2MK7(mg)}$&1.63e-03 ;0.025;2/23&5.83e-03 ;0.025;7/30&2.08e-03 ;0.025;2/24&&\\
$\textrm{MgCitrate(mg)}$&96 ;200;21/23&100 ;100;30/30&92 ;100;22/24&31 ;100;10/31&76 ;100;22/29\\
$\textrm{Mn(mg)}$&&&0.042 ;1;1/24&0.21 ;0.62;12/31&0.12 ;1;6/29\\
$\textrm{Se(mcg)}$&&0.42 ;12;1/30&&&0.43 ;12;1/29\\
\hline
\end{tabular}
\caption{Part 1 of 2.  Events Summary for Happy   from 2023-10-01 to 2024-04-10A summary of most dietary components and events  for selected months between \mjmdatemin and \mjmdatemax. Format is average daily amount ;maximum; days given/ days in interval . Units are arbitrary except where noted. Any  superscripts are defined as follows:  \mjmsuperscripts}
\end{table}
\begin{table}[H]
\centering
\begin{tabular}{|l|r|r|r|r|r|}
\hline
Name&2023-10 Oct&2023-11 Nov&2023-12 Dec&2024-01 Jan&2024-02 Feb\\
\hline
$\textrm{Zn(mg~zn)}$&1.3 ;5.9;9/23&1.1 ;5.9;10/30&0.73 ;2.9;6/24&0.47 ;2.9;5/31&0.61 ;5.9;5/29\\
$\textrm{arginine(mg)}$&68 ;175;9/23&82 ;350;10/30&51 ;175;7/24&79 ;350;12/31&275 ;350;15/29\\
$\textrm{biotin(mg)}^{\left(a\right)}$&2.4 ;5;11/23&4.3 ;5;26/30&4 ;5;19/24&3.5 ;5;22/31&3.6 ;5;21/29\\
$\textrm{folate(mg)}$&0.022 ;0.12;5/23&0.019 ;0.12;6/30&0.018 ;0.12;4/24&0.016 ;0.12;5/31&0.011 ;0.12;3/29\\
$\textrm{histidine(tsp)}$&&&&&2.42e-03 ;0.016;7/29\\
$\textrm{histidinehcl(mg)}$&3.7 ;85;1/23&1.4 ;42;1/30&1.6 ;38;1/24&&\\
$\textrm{iron(mg)}$&&1 ;4;8/30&1.8 ;4;11/24&1.3 ;4;10/31&2.2 ;4;18/29\\
$\textrm{isoleucine(mg)}$&30 ;200;5/23&47 ;200;8/30&17 ;200;2/24&48 ;200;9/31&45 ;200;8/29\\
$\textrm{lecithin(mg)}$&215 ;225;22/23&225 ;225;30/30&281 ;225;22/24&330 ;225;31/31&338 ;225;29/29\\
$\textrm{lecithin(tsp)}$&0.046 ;0.062;22/23&0.036 ;0.042;30/30&0.012 ;0.062;8/24&&\\
$\textrm{leucine(mg)}$&74 ;162;20/23&76 ;81;28/30&85 ;162;24/24&66 ;81;25/31&67 ;81;24/29\\
$\textrm{leucine}$&&&&&\\
$\textrm{lipoicacid(mg)}^{\left(a\right)}$&3.1 ;25;5/23&7.6 ;25;16/30&24 ;25;21/24&18 ;25;22/31&31 ;25;28/29\\
$\textrm{lysinehcl(mg)}$&170 ;162;23/23&203 ;162;30/30&186 ;162;24/24&218 ;325;30/31&235 ;325;14/29\\
$\textrm{methionine(mg)}$&57 ;62;21/23&46 ;62;22/30&38 ;125;20/24&4 ;62;3/31&9.7 ;62;7/29\\
$\textrm{pantothenate(mg)}^{\left(a\right)}$&22 ;78;12/23&20 ;39;15/30&21 ;39;13/24&32 ;39;25/31&30 ;39;22/29\\
$\textrm{phenylalanine(mg)}$&38 ;125;7/23&23 ;125;6/30&18 ;125;4/24&8.1 ;125;2/31&15 ;125;4/29\\
$\textrm{proline(mg)}$&143 ;100;23/23&35 ;100;7/30&&&\\
$\textrm{taurine(mg)}$&323 ;225;23/23&338 ;225;30/30&323 ;225;24/24&345 ;225;31/31&338 ;225;29/29\\
$\textrm{threonine(mg)}$&95 ;162;23/23&374 ;325;30/30&467 ;325;24/24&488 ;325;31/31&487 ;325;29/29\\
$\textrm{tryptophan(mg)}$&52 ;150;14/23&40 ;150;14/30&25 ;150;6/24&17 ;150;6/31&24 ;75;10/29\\
$\textrm{tyrosine(mg)}$&17 ;100;4/23&6.7 ;100;2/30&12 ;100;3/24&19 ;100;6/31&19 ;100;6/29\\
$\textrm{valine(mg)}$&165 ;200;19/23&160 ;200;24/30&133 ;200;16/24&135 ;200;21/31&159 ;200;23/29\\
$\textrm{vitamina(iu)}$&489 ;2250;5/23&600 ;2250;8/30&656 ;4500;6/24&435 ;2250;6/31&466 ;2250;6/29\\
$\textrm{vitaminc(tsp)}$&3.23e-03 ;0.0078;11/23&3.39e-03 ;0.0078;13/30&8.14e-04 ;0.0039;5/24&5.04e-04 ;0.0039;4/31&5.39e-04 ;0.0078;2/29\\
$\textrm{vitamine(iu)}$&8.2 ;38;5/23&8.8 ;38;7/30&9.4 ;38;6/24&7.3 ;38;6/31&6.5 ;38;5/29\\
{\bf MEDICINE}&&&&&\\
$\textrm{SnAg}$&&&&1.1 ;1;13/31&0.66 ;1;12/29\\
$\textrm{sodiumbenzoate(tsp)}$&0.011 ;0.016;12/23&8.85e-03 ;0.016;12/30&0.012 ;0.031;15/24&0.018 ;0.016;25/31&0.018 ;0.016;24/29\\
$\textrm{wormer}$&&&&&\\
{\bf RESULT}&&&&&\\
$\textrm{weight(lbs)}$&&&0.63 ;15;1/24&&1.1 ;16;2/29\\
&&&&&\\
$\textrm{sorbitol(tsp)}$&0.045 ;0.031;23/23&0.047 ;0.031;30/30&0.045 ;0.031;24/24&0.046 ;0.062;31/31&0.047 ;0.031;29/29\\
&&&&&\\
&&&&&\\
&&&&&\\
&&&&&\\
&&&&&\\
\hline
\end{tabular}
\caption{Part 2 of 2.  Events Summary for Happy   from 2023-10-01 to 2024-04-10A summary of most dietary components and events  for selected months between \mjmdatemin and \mjmdatemax. Format is average daily amount ;maximum; days given/ days in interval . Units are arbitrary except where noted. Any  superscripts are defined as follows:  \mjmsuperscripts}
\end{table}
\begin{table}[H]
\centering
\begin{tabular}{|l|r|r|}
\hline
Name&2024-03 Mar&2024-04 Apr\\
\hline
{\bf FOOD}&&\\
$\textrm{KCl(tsp~kcl)}$&0.084 ;0.062;20/20&0.087 ;0.062;10/10\\
$\textrm{KibbleAmJrLaPo}$&0.034 ;0.037;18/20&0.034 ;0.037;9/10\\
$\textrm{KibbleLogic}$&0.023 ;0.025;18/20&0.022 ;0.025;9/10\\
$\textrm{b10ngnc}^{\left(c\right)}$&0.069 ;0.25;4/20&0.056 ;0.25;2/10\\
$\textrm{b15ngnc}^{\left(c\right)}$&0.022 ;0.25;2/20&\\
$\textrm{b20ngnc}^{\left(c\right)}$&0.33 ;0.25;17/20&0.19 ;0.25;6/10\\
$\textrm{b25ngnc}$&&\\
$\textrm{b7ngnc}^{\left(c\right)}$&&0.16 ;0.25;4/10\\
$\textrm{blackberry}$&&\\
$\textrm{blueberry}$&0.75 ;0.75;20/20&0.9 ;1;10/10\\
$\textrm{carrot}$&0.35 ;0.25;20/20&0.35 ;0.25;10/10\\
$\textrm{cbbrothbs}$&&\\
$\textrm{cbbroth}$&0.1 ;0.25;5/20&\\
$\textrm{citrate(tsp~citrate)}$&0.081 ;0.062;20/20&0.086 ;0.062;10/10\\
$\textrm{ctbrothbs}$&0.33 ;0.25;17/20&0.41 ;0.25;10/10\\
$\textrm{ctbroth}$&&\\
$\textrm{eggo3}$&0.025 ;0.062;8/20&0.062 ;0.062;10/10\\
$\textrm{eggo}$&0.037 ;0.062;12/20&\\
$\textrm{eggshell}$&&\\
$\textrm{garlic}$&1.4 ;1;18/20&1.1 ;1;10/10\\
$\textrm{marrow}$&&\\
$\textrm{oliveoil(tsp)}$&0.042 ;0.12;6/20&\\
$\textrm{pepper}$&0.36 ;0.25;20/20&0.35 ;0.25;10/10\\
$\textrm{pineapple}$&&\\
$\textrm{raspberry}$&&\\
$\textrm{salmon}$&&\\
$\textrm{shrimp(grams)}$&&\\
$\textrm{spinach}$&0.35 ;0.25;20/20&0.35 ;0.25;10/10\\
$\textrm{sunflowerseed}$&0.037 ;0.25;3/20&0.2 ;0.25;8/10\\
$\textrm{tomato}$&0.12 ;0.12;20/20&0.12 ;0.12;10/10\\
$\textrm{tuna(oz)}$&0.062 ;0.25;5/20&0.075 ;0.25;3/10\\
$\textrm{turkey}$&0.33 ;0.25;20/20&0.35 ;0.25;10/10\\
$\textrm{vinegar(tsp)}$&6.25e-03 ;0.062;3/20&3.13e-03 ;0.031;1/10\\
{\bf VITAMIN}&&\\
$\textrm{B-1(mg)}$&5.58e-03 ;0.012;18/20&5.87e-03 ;0.0059;10/10\\
$\textrm{B-12(mg)}$&0.05 ;0.25;6/20&0.025 ;0.12;2/10\\
$\textrm{B-2(mg)}$&47 ;16;20/20&37 ;16;10/10\\
$\textrm{B-3(mg)}$&69 ;24;20/20&55 ;24;10/10\\
$\textrm{B-6(mg)}$&4.7 ;6.2;15/20&3.8 ;6.2;6/10\\
$\textrm{B-multi(count)}$&3.13e-03 ;0.062;1/20&\\
$\textrm{Cu(mg)}$&2.2 ;2;20/20&2.6 ;2;10/10\\
$\textrm{D-3(iu)}$&62 ;350;4/20&60 ;300;2/10\\
$\textrm{Iodine(mg)}^{\left(a\right)}$&0.19 ;0.78;5/20&0.16 ;0.78;2/10\\
$\textrm{K1(mg)}$&1.1 ;1.2;17/20&1.2 ;1.2;10/10\\
$\textrm{K2(mg)}$&0.75 ;3.1;6/20&\\
$\textrm{K2MK7(mg)}$&&\\
$\textrm{MgCitrate(mg)}$&88 ;100;18/20&90 ;100;9/10\\
$\textrm{Mn(mg)}$&0.14 ;1.2;3/20&\\
$\textrm{Se(mcg)}$&&\\
\hline
\end{tabular}
\caption{Part 1 of 2.  Events Summary for Happy   from 2023-10-01 to 2024-04-10A summary of most dietary components and events  for selected months between \mjmdatemin and \mjmdatemax. Format is average daily amount ;maximum; days given/ days in interval . Units are arbitrary except where noted. Any  superscripts are defined as follows:  \mjmsuperscripts}
\end{table}
\begin{table}[H]
\centering
\begin{tabular}{|l|r|r|}
\hline
Name&2024-03 Mar&2024-04 Apr\\
\hline
$\textrm{Zn(mg~zn)}$&0.73 ;5.9;3/20&0.59 ;5.9;1/10\\
$\textrm{arginine(mg)}$&245 ;350;10/20&228 ;350;5/10\\
$\textrm{biotin(mg)}^{\left(a\right)}$&3.4 ;5;14/20&3.5 ;5;7/10\\
$\textrm{folate(mg)}$&0.013 ;0.12;3/20&\\
$\textrm{histidine(tsp)}$&0.021 ;0.016;19/20&0.02 ;0.031;8/10\\
$\textrm{histidinehcl(mg)}$&&\\
$\textrm{iron(mg)}$&2.4 ;5.3;17/20&5.3 ;5.3;8/10\\
$\textrm{isoleucine(mg)}$&25 ;200;3/20&20 ;200;1/10\\
$\textrm{lecithin(mg)}$&315 ;225;20/20&315 ;225;10/10\\
$\textrm{lecithin(tsp)}$&&\\
$\textrm{leucine(mg)}$&73 ;81;18/20&81 ;81;10/10\\
$\textrm{leucine}$&&\\
$\textrm{lipoicacid(mg)}^{\left(a\right)}$&16 ;25;12/20&20 ;25;8/10\\
$\textrm{lysinehcl(mg)}$&228 ;325;10/20&244 ;325;5/10\\
$\textrm{methionine(mg)}$&12 ;62;8/20&25 ;62;4/10\\
$\textrm{pantothenate(mg)}^{\left(a\right)}$&33 ;39;17/20&35 ;39;9/10\\
$\textrm{phenylalanine(mg)}$&28 ;125;5/20&12 ;125;1/10\\
$\textrm{proline(mg)}$&&\\
$\textrm{taurine(mg)}$&315 ;225;20/20&315 ;225;10/10\\
$\textrm{threonine(mg)}$&455 ;325;20/20&422 ;325;10/10\\
$\textrm{tryptophan(mg)}$&26 ;75;7/20&22 ;75;4/10\\
$\textrm{tyrosine(mg)}$&22 ;100;6/20&30 ;100;3/10\\
$\textrm{valine(mg)}$&160 ;200;16/20&160 ;200;8/10\\
$\textrm{vitamina(iu)}$&506 ;2250;5/20&675 ;2250;3/10\\
$\textrm{vitaminc(tsp)}$&8.79e-04 ;0.0039;5/20&1.95e-03 ;0.0039;5/10\\
$\textrm{vitamine(iu)}$&7.5 ;38;4/20&7.5 ;38;2/10\\
{\bf MEDICINE}&&\\
$\textrm{SnAg}$&&\\
$\textrm{sodiumbenzoate(tsp)}$&0.016 ;0.016;14/20&7.81e-04 ;0.0078;1/10\\
$\textrm{wormer}$&0.075 ;1.5;1/20&\\
{\bf RESULT}&&\\
$\textrm{weight(lbs)}$&&\\
&&\\
$\textrm{sorbitol(tsp)}$&0.044 ;0.031;20/20&0.041 ;0.031;10/10\\
&&\\
&&\\
&&\\
&&\\
&&\\
\hline
\end{tabular}
\caption{Part 2 of 2.  Events Summary for Happy   from 2023-10-01 to 2024-04-10A summary of most dietary components and events  for selected months between \mjmdatemin and \mjmdatemax. Format is average daily amount ;maximum; days given/ days in interval . Units are arbitrary except where noted. Any  superscripts are defined as follows:  \mjmsuperscripts}
\end{table}



\section{Symbols, Abbreviations and Colloquialisms}

\begin{comment}
% grep "[A-Z][A-Z]" paradox.tex | sed -e 's/[^A-Z]/\n/g' | grep "[A-Z]" | sort | uniq -c
% cat  paradox.tex | sed -e 's/  */\n/g' | grep "[A-Z][A-Z]"  | grep -v "[^A-Z]" | sort | uniq  |awk '{print $0" &   \\\\"; }'
\end{comment}


%\abbreviations{The following abbreviations are used in this manuscript:\\
%\begin{table}
\noindent
\begin{tabular}{@{}ll}
%SMVT & Sodium dependent Multi-Vitamin Transporter\\
TERM & definition and meaning   \\
\hline
%TLA & Three letter acronym\\
%LD & linear dichroism
\end{tabular} % }
%\end{table}

% https://tex.stackexchange.com/questions/5957/bibtex-entry-for-white-papers-and-technical-reports

\section{General caveats and disclaimer }
\label{appendix:caveats}

%\input{disclaimer-informal.tex}

This document was created in the hope it will be interesting to
someone including me by providing information 
about some topic that may include personal experience or a literature
review or description of a speculative theory or idea.
There is no assurance that the content of this work will be
useful for any paricular purpose. 
%In no case am I claiming to provide useful advice on any matter
%but attempting to describe events in terms of literature known
%to me. 


All statements in this document were true to the best of my knowledge
at the time they were made and every attempt is made to assure
they are not misleading or confusing. However, information provided by
others and observations that can be manipulated by unknown causes  
( "gaslighting" ) may be misleading. Any use of this information should
be preceded by validation including replication where feasible.
Errors may enter into the final work at every step from conception
and research to final editing. 
%No assurance can exist that obvious conclusions will be useful
%and may be misleading. 



Documents labelled "NOTES" or "not public" contain
substantial informal or speculative content that
may be terse and poorly edited or even sarcastic or profane.
Documents labelled as "public" have generally been edited
to be more coherent but probably have not been reviewed
or proof read. 

Generally non-public documents are labelled as such to avoid
confusion and embarassment and should be read with that understanding.


\section{Citing this as a tech report or white paper }
\label{appendix:citing}

Note: This is mostly manually entered and not assured to be error free.

This is tech report \mjmtrno. 

\begin{table}[H] \centering
\begin{tabular}{r|r|c|r}
Version & Date & Comments  &  \\
0.01 & \mjmmakedate  &  Create from empty.tex template  &  \\
-  & \today & version  \mjmversion { }   \mjmtrno  &  \\
1.0 & 20xx-xx-xx & First revision for distribution &  \\
\end{tabular}
\end{table}


Released versions,

build script needs to include empty releases.tex
\begin{table}[H] \centering
\begin{tabular}{|r|r|l|}
Version & Date & URL    \\
\hline
&  &  \\
% version & date & url  \\
%.1 table & 2021-08-17& {\url{https://www.linkedin.com/posts/marchywka_draft-compare-72020-theory-with-interim-activity-6833343119203860480--wJv}} \\
%.1 table & 2021-08-17& {\url{https://www.researchgate.net/publication/353946686_Draft_table_comparing_expectations_to_recent_results_with_covid-19}} \\
%.1 table & 2021-08-17 & {\url{https://www.academia.edu/s/34e160cae9}} \\

\hline
\end{tabular}
\end{table}





% 2020-11-30 keep on same page 
%\input{bibtex2.txt}

\begin{minipage}{\linewidth}
%\input{bibtex2.txt}
%\input{bibtex3.txt}
\mjmshowbib
\end{minipage}




\begin{comment}

\end{comment}
\vspace{1cm}
Supporting files. Note that some dates,sizes, and md5's will change as this is
rebuilt.

This really needs to include the data analysis code 
but right now it is auto generated picking up things from prior
build in many cases 
\lstinputlisting{\mjmbasename.bundle_checksums}
\end{mdpicomment}
\end{document}

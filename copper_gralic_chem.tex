
With the preceding discussion of copper chemistry in general,
discussion of specific garlic chemistry is put into context.
Its important to distinguish "garlic sulfur" from the
sulfur sources consumed by cattle leading to disease
given their rumen properties and other sources of sulfur
common in human or dog diets. Cattle may ingest weeds that
accumulate sulfur from high sulfur soil but may also get
sulfates from feed formulations and the local water
sources. Rumen bacteria tend to reduce these to sulfides.
cruciferous vegatables tend to contain sugar modified amino acids
called glucosinolates. Legumes largely contain cysteine.
 



Often, copper supplement was combined with crushd garlic
prior to mixing into the larger snack. While best known
for containing unstable and volatile sulfur compounds,
other components of garlic may be important for overall
effects on copper fate and health effects.

The literature on garlic-metail interactions includes
bioavailability and corrosion inhibition with modern
techniques applied to results analysis. 


The most well known sulfur compounds in garlic tend to 
be organic sulfurs with 1 or 2 sulfures attached to an
alkane or alkene a few carbons long. As these have
an odor they likely have a high vapor pressure and
may diffuse easily in some settings. A copper or
other metal attached to the sulfurs would not likely
be held in an insoluble solid but rather allowed
to move more freely. The question would then remain
about getting to a suitable location and liberating
the copper from the sulfur carrier. 




\mjmtol{ collection of facts thoughts and citations 
coming together well but needs more time for organization }

This work discussed concerns about copper availability and
degradation of other nurients with consideration of an entire
meal and digestive system. A range of environments
are encountered from formulation or growth to absorption.  
Initially the relevant chemistry relates to copper in food
and supplements while it is mixed, stored, or cooked.
Later, the cemistry of the GI tract notably the stomach
is important for interactions with the total intake. 
Surprisingly the literature from remediation and leeching
may be relevant here. 

A central topic  is copper speciation or oxidation state.
Perhaps four copper oxidation states have been 
considered in the biological context. Solid Cu(0),
Cu(I), Cu(II), and rarely Cu(III). Cu(I) and Cu(II)
are probably the most relevant to food effects. 
Solid copper or Cu(0) has been claimed to have
environmental remidiation capabilities
\cite{deSousa2019}.
Cu(iii) is thought to be an intermediary in the
copper Fenton reaction at near neutral pH with
bicaronate and sodum chloride present
\cite{Pham_Xing_Miller_Fenton_like_copper_redox_2013}.
Cu(iii) production for degratdation was also enhanced
with histidine
\cite{Park_Lee_Kim_Trivalent_Copper_Mediated_2024}.


The first  concern,   the formation of insolubles that
have decreased availability of the copper, is most clearly seen
with sulfides but many other stable copper compounds
are known. Copper minerals represent some compounds that
exhibit low solubility in their native settings. Wikipedia lists
several cooper ore minerals \cite{Wikimediaprojects_List_copper_ores_2024}
continaing sulfur, iron, antimony,zinc, carbonate, hydroxide, aluminum,
silicon, hydrogen, arsenic, and water.  Of course an infinite
number of organics could be formed with variable stability.
Early XPS work suggested many copper minerals contain copper
in the +1 state \cite{NAKAI_IZAWA_SUGITANI_photoelectron_spectroscopic_1976}
although work continues to clarify specific issues
\cite{Goh_Buckley_Lamb_oxidation_states_2006}.
Generally the minerals can be leeched in acid to recover the copper
and that chemistry may be useful here.  

It has been noted that Cu(II) is converted to Cu(I) prior
to transport and therefore some benefit may be
obtained by making the conversion from Cu(II) earlier.
However, as with some other nutrients, this may
be counterproductivex when normal mechanisms are operational.
CTR1 for example will convert prior to transport
\cite{Kar_Sen_Maji_Copper_import_2022}.
This is likely similar to the situation with iron
\cite{Sanchez_Sabio_Galvez_Iron_chemistry_2017}
\begin{quote}
The two drawbacks of iron chemical characteristics, that is, Fe31 insolubility
and toxic radical production by free iron, must be jointly addressed in
some biological processes, such as the process of iron absorption by enterocytes
at the duodenum. Iron is oxidized by stomach gastric juices, and most iron
reaches the duodenum in the Fe31 form (32). As the stomach has a pH <3,
Fe31 remains soluble until it passes to the duodenum where the pH is significantly
higher. Thus, to enter into the enterocyte Fe31 must be reduced to Fe21.
The absorption of the as-formed Fe21 must be done quickly to avoid the formation
of hydroxyl radicals. To overcome this inconvenience, humans have developed
a very efficient mechanism based on a protein pair. Iron transfer through
the enterocyte membrane occurs due to the combined activities of two proteins:
the DMT1 Fe21 transporter and the conjugated DcytB, which can reduce Fe31
to Fe21 (33). The DcytB/DMT1 pair is required for iron absorption since
iron enters the small intestine lumen mainly as Fe31. DcytB, an ironregulated
ferrireductase protein is highly expressed at duodenal enterocytes and reduces
the Fe31. Upon reduction, Fe21 is transferred across the apical membrane
of enterocytes by divalent metal transporters, mainly DMT1 (Fig. 6).
\end{quote}

The same work
\cite{Sanchez_Sabio_Galvez_Iron_chemistry_2017}
may be a good background for understanding copper more generally
as a lot of issues with iron may be similar. 



Reactions of Cu(I) may reduce net uptake due to precipitate
formation that requires re-dissolution to absorb.


 Copper "Fenton" reactions have been extensively explored
for pollution remediation by destruction of toxic
chemicals. Many of these reactions occur under mild
nearly physiological conditions suggesting they may
be able to degrade nutrients. 
In particular, bicarbonate in the 4mM range improves
the degradation rates for a Fenton like Cu system 
\cite{Peng_Zhang_Zhang_Enhanced_mediated_2019}.
In fact, carbonate species 
are thought to be the active products not hydroxyl
in neutral or alkali media 
\cite{PMID30458276}. Bicaronbate is being investigated
for pollutant degradation optimization \cite{PMID39708608}
as it apparently is quite active. 
\mjmtol{
The photograph in \mjmreffig{baking}  shows the side of
a baking pan on the right which was represntative of the surface
to the left prior to heating for a while in baking soda water.
The carbonate derived species may easily react with or
make soluble the oxidized food grease which I thought was
just a "like dissolves like" issue but may be as stated here.
An ammonium bicarbonate electrochemical etch apaprently
optimized the mechanical properties of carbon fibers
\cite{Kim_Jeong_An_study_2019}.
}


 % https://www.linkedin.com/posts/peter-dingle-709206_bicarb-therapy-for-acidosis-activity-7072474628274679808-BevO/
\mjmpicture{bakingpan.jpg}{  A previously brown baking pan with baked on carbonized food largely grease. Heating in oven cleaned the metal submerged in baking soda water while leaving the metal above the water line dark. See if bicarbonate has any relevance here or save for another work lol.  \label{fig:baking} }{bakingpan}

Cu(II) is the more common species but it can be reduced
under relevant conditions by ascorbate, riboflavin leading to
DNA damage \cite{PMID8349205}( although maybe assisted with
light \cite{PMID8791088} )  or reducing sugare
at elevated temperatures which may occur during food processing.
It can be reduced and complexed with niacin
\cite{charvv324ska_charvv322y_Interaction_niacin_with_2013}.
It may be worth noting however even the interaction with ascorbate
is controversial with redox competing with a catalytic
mechanism leaving the Cu unchanged
\cite{Shen_Griffiths_Campbell_Ascorbate_oxidation_iron_2021}.
Generally copper absorption has been reported to decrease with
higher dietary ascorbate with a confusing interaction with iron
\cite{Johnson1988} which may be easy to rationalize a increased
copper reduction prior to uptake. 


Copper can be reduced by several things.


\cee{ 2Cu^{++}  + 4I^-  -> 2CuI +  I_2}



Once Cu(I) is formed, it can undergo underirable reacions.

\cite{SAMUNI_ARONOVITCH_GODINGER_cytotoxicity_1983}

\cee{ Cu^+ + H_2O_2  -> Cu^{++} +  OH^- +  OH }

disproportionation,

\cee{ 2Cu^+ (sq)  -> Cu^{++}(aq) +  Cu(s) }

Its also worth mentioning that several "electroless copper"
formulations exist and there may be similar mixtures
that occur in food or in vivo. ZZ

\mjmtol{ Personally applying Cu saturated ammonia to the skin
can leave a light blue residue or wash off but topicl peroxide 
may produce a nice "copper tone" lol. }

It is quite reasonable to consider the possibility that
uptake is limited by stomach acid and chloride assuming
that once solbulble the copper can be stabilized on something
available. 


First its interesting to note that many high-school
level chemsitry examples are probably relevant. 
Copper is used to detect reducing sugars.
This is one reaction that may damage other components
such as sugars.
Interactions with iodine are textbook illustrations.


Once a copper sulfide is formed, the copper may be recovered
in processes such as commercial leaching that can go at
30C. In an acid environment both chloride and iron 
are important to getting optimal copper extraction from 
sulfide ore 
\cite{Salinas_Herreros_Torres_Leaching_Primary_Copper_2018}.

And some have observed chloride may be generally better
for extractions of metals from sulfides, 
\cite{Herreros_Vinals_Leaching_sulfide_copper_2007}
\begin{quote}
Generally, the leaching of the sulfides occurs more easily in solutions of chloride rather than sulfate (Lu et al., 2000a, Lu et al., 2000b, Fisher, 1994, Fisher et al., 1992, Chu and Lawson, 1991a, Chu and Lawson, 1991b). 
\end{quote}

Some leaching reactions include 
\cite{Schlesinger_King_Sole_Hydrometallurgical_Copper_Extraction_2011}


\begin{figure}[ht]
\begin{align}
\cee{Cu_{2}S +Fe_2(SO_4)_3  & -> Cu^{++} + SO^{--}_4 + CuS + 2FeSO_4  } \\
\cee{Cu_{2}S +.5O_2+JH_2(SO_4)  & -> Cu^{++} + SO^{--}_4 + CuS + H_2O  } \\
%\cee{H2O &<=> H^{+}_{(aq)} + OH^{-}_{(aq)}} \\
%\cee{Cl_{2} + 2OH^{-}  & <=> Cl^{-} + O-Cl + H_{2}O_{l} \label{r:ecbleach} } \\
%\cee{Cl_{2} + 2NaOH & -> NaCl + NaClO + H_{2}O \label{r:hypochlorate}}\\
%\cee{NaCl + 3H_{2}O & <=> NaClO_3 + 3H_2 \label{r:chlorate}} \\
%\cee{ H_{2}O_{2} + NaClO & -> NaCl + H_{2}O + O_{2} \label{r:bleachperoxide}}\\
\end{align}
\caption{ \label{fig:leachers} Some leaching reactions
\cite{Schlesinger_King_Sole_Hydrometallurgical_Copper_Extraction_2011}
 at ambient or elevated temperatures  }
\end{figure}

Ultimately food and stomach contents form a complicated variable
mixture of chemical with varying ability to interact with copper
and a survey of such properties such as 
\cite{Manceau_Matynia_nature_2010} may be useful .



Reactions with "Reducing sugars"
and iodine are examples where nutrients are transformed
or made insoluble by interaction with copper. although
assay solutions typically are used at elevated temperatures
slower reactions may be observed at room temperature.

Details may be important for minimizing these unintended
or spoiling reactions but elucidation is currently incomplete
and may be quite context specific. 


A brief theoretical digression may also seem out of place
but it will help to inoke literature often ignored in this
context. 
Copper's outer shell is $4s^13d^{10}$
( compared to iron $4s^23d^6$
and off hand you may expect filled or emptied s or d orbitals
to be more stable. 
 

A discussion of the phase diagram and related materials
issues of the copper-sulfur system may be a bit tangential
but it also picks up issues with metastability and cooking.
Relatively high concentrations of copper ions may occur during
mixing or segregation.
It may be possible to stop formation of the worst phases,
those that sequester copper in organisms or make it
mobile in integrated circuits, even if sulfur can not
be well eliminated. 

Copper sulfides can form varying large unit cell crystals 
with correrpsondingly high integer stoichiometry. 
They apparently are not molecular crystals.
Unit cells may contain 24 or 62 distinct copper atoms and some phases
contain a sulphur HCP structure with interstial Cu that become
fluid above 100C \cite{Evans_crystal_structures_}.

This may appear unrelated to anything biologically relevant
but a number of unknowns exist around the interaction of copper
with mitochondrial ligand CuL. For example
\cite{Fitisemanu_PadillaBenavides_Emerging_perspectives_copper_2024},
\begin{quote}
Mobilization of Cu into the mitochondrial matrix was also shown to occur
bound to a non-proteinaceous fluorescent ligand, termed CuL {[}110, 113,
114{]}. CuL levels were found to correlate with cellular Cu levels and COX
assembly {[}110, 115{]}. CuL, suggested to be coordinated by citrate and
oxaloacetate, migrates with a mass of 13 000 Da, hinting at larger ligands.
Fluorescence anisotropy experiments showed that the CuL complex is transported
by SLC25A3 and MRS3 transporters {[}115117{]}. Although the molecular identity
of CuL and the mechanism for exiting the matrix and delivering Cu to apo-COX17
in the intermembrane space remain unclear, its biophysical properties suggest
it contributes to buffering cytosolic Cu and facilitating Cu uptake into
mitochondria {[}110, 113{]}
\end{quote}

If these are not resolved in simple well known structures they could
involve unusual interactions with things like citrate or phosphate.


A phase diagrom aroun Cu2S was published in 1981
showing phase transformations above 72C
\cite{Evans1981CopperCI} which is too high to be biologically
relevant except for cooking but may be important
in IC failure modes. 
A more complete phase diagram from CMU 1983 is also available
\cite{Chakrabarti_Laughlin_Copper_1983}.
Metasbability may be common due to need to rearrange hcp.


Most well studied crystals have simple unit cells containing
only a few atoms. To make these crystals, rather than amorphous
materials, state information must be propagated through
many bonds. This could be due to various electron configuration
issue or long range forces. 


\begin{comment}
\begin{figure}[ht]
\begin{align}
\cee{H2O &<=> H^{+}_{(aq)} + OH^{-}_{(aq)}} \\
\cee{Cl_{2} + 2OH^{-}  & <=> Cl^{-} + O-Cl + H_{2}O_{l} \label{r:ecbleach} } \\
\cee{Cl_{2} + 2NaOH & -> NaCl + NaClO + H_{2}O \label{r:hypochlorate}}\\
\cee{NaCl + 3H_{2}O & <=> NaClO_3 + 3H_2 \label{r:chlorate}} \\
\cee{ H_{2}O_{2} + NaClO & -> NaCl + H_{2}O + O_{2} \label{r:bleachperoxide}}\\
\end{align}
\caption{ \label{fig:reacts} test to see how chemfig works.  wtf is outer par mode?  Some reactions thought to be relevant to generation, decay, and
efficacy of electrolyzed water solutions. }
\end{figure}

\end{comment}

Dexter, who may be copper sensitive, has a varying prefernce
for turkey and this was eliminated from Happy's diet for a while
on the concern it reduces copper uptake.




\mjmtol{ collection of facts thoughts and citations 
coming together well but needs more time for organization }

Perhaps a more comprehensive summary of copper chemistry
could have been included earlier but the following is specialized
to issues that involve the uptake of copper due to diet and
GI attributes and the issues discovered during data analysis. 

Initially the relevant chemistry relates to copper in food
and supplements while it is mixed, stored, or cooked.
Later, the cemistry of the GI tract notably the stomach
is important for interactions with the total intake. 

First its interesting to note that many high-school
level chemsitry examples are probably relevant. 
Copper is used to detect reducing sugars.
This is one reaction that may damage other components
such as sugars.
Interactions with iodine are textbook illustrations.

A brief theoretical digression may also seem out of place
but it will help to inoke literature often ignored in this
context. 
Copper's outer shell is $s^1d^{10}$
and off hand you may expect filled or emptied s or d orbitals
to be more stable. 
Perhaps four copper oxidation states have been 
considered in the biological context. Solid Cu(0),
Cu(I), Cu(II), and rarely Cu(III). Cu(I) and Cu(II)
are probably the most relevant to food effects. 

It has been noted that copper is converted to Cu(I) prior
to transport and therefore some benefit may be
obtained by making the conversion from Cu(II) earlier.
However, as with some other nutrients, this may
be counterproductive. 
Reactions of Cu(I) may reduce net uptake due to precipitate
formation that requires re-dissolution to absorb.


 Copper "Fenton" reactions have been extensively explored
for pollution remediation by destruction of toxic
chemicals. Many of these reactions occur under mild
nearly physiological conditions suggesting they may
be able to degrade nutrients. 
In particular, bicaronate in the 4mM range improves
the degradation rates for a Fenton like Cu system 
\cite{Peng_Zhang_Zhang_Enhanced_mediated_2019}.
See \mjmreffig{baking}.

Cu(II) is the more common species but it can be reduced
under relevant conditions by ascorbate, riboflavin leading to
DNA damage \cite{PMID8349205}( although maybe assisted with
light \cite{PMID8791088} )  or reducing sugare
at elevated temperatures which may occur during food processing.
It can be reduced and complexed with niacin
\cite{charvv324ska_charvv322y_Interaction_niacin_with_2013}.
Copper can be reduced by several things.


\cee{ 2Cu^{++}  + 4I^-  -> 2CuI +  I_2}



Once Cu(I) is formed, it can undergo underirable reacions.

\cite{SAMUNI_ARONOVITCH_GODINGER_cytotoxicity_1983}

\cee{ Cu^+ + H_2O_2  -> Cu^{++} +  OH^- +  OH }

disproportionation,

\cee{ 2Cu^+ (sq)  -> Cu^{++}(aq) +  Cu(s) }



It is quite reasonable to consider the possibility that
uptake is limited by stomach acid and chloride assuming
that once solbulble the copper can be stabilized on something
available. 
Once a copper sulfide is formed, the copper may be recovered
in processes such as commercial leaching that can go at
30C. In an acid environment both chloride and iron 
are important to getting optimal copper extraction from 
sulfide ore 
\cite{Salinas_Herreros_Torres_Leaching_Primary_Copper_2018}.

And some have observed chloride may be generally better
for extractions of metals from sulfides, 
\cite{Herreros_Vinals_Leaching_sulfide_copper_2007}
\begin{quote}
Generally, the leaching of the sulfides occurs more easily in solutions of chloride rather than sulfate (Lu et al., 2000a, Lu et al., 2000b, Fisher, 1994, Fisher et al., 1992, Chu and Lawson, 1991a, Chu and Lawson, 1991b). 
\end{quote}






Reactions with "Reducing sugars"
and iodine are examples where nutrients are transformed
or made insoluble by interaction with copper. although
assay solutions typically are used at elevated temperatures
slower reactions may be observed at room temperature.

Details may be important for minimizing these unintended
or spoiling reactions but elucidation is currently incomplete
and may be quite context specific. 


 

A discussion of the phase diagram and related materials
issues of the copper-sulfur system may be a bit tangential
but it also picks up issues with metastability and cooking.
Relatively high concentrations of copper ions may occur during
mixing or segregation.
It may be possible to stop formation of the worst phases,
those that sequester copper in organisms or make it
mobile in integrated circuits, even if sulfur can not
be well eliminated. 

Copper sulfides can form varying large unit cell crystals 
with correrpsondingly high integer stoichiometry. 
They apparently are not molecular crystals.
Unit cells may contain 24 or 62 distinct copper atoms and some phases
contain a sulphur HCP structure with interstial Cu that become
fluid above 100C \cite{Evans_crystal_structures_}.

A phase diagrom aroun Cu2S was published in 1981
showing phase transformations above 72C
\cite{Evans1981CopperCI} which is too high to be biologically
relevant except for cooking but may be important
in IC failure modes. 
A more complete phase diagram from CMU 1983 is also available
\cite{Chakrabarti_Laughlin_Copper_1983}.
Metasbability may be common due to need to rearrange hcp.


Most well studied crystals have simple unit cells containing
only a few atoms. To make these crystals, rather than amorphous
materials, state information must be propagated through
many bonds. This could be due to various electron configuration
issue or long range forces. 

 % https://www.linkedin.com/posts/peter-dingle-709206_bicarb-therapy-for-acidosis-activity-7072474628274679808-BevO/
\mjmpicture{bakingpan.jpg}{ See if bicarbonate has any relevance here or save for another work lol.  edge on right partially cleaned by heating baking soda water\label{fig:baking} }{bakingpan}



\begin{comment}
\begin{figure}[ht]
\begin{align}
\cee{H2O &<=> H^{+}_{(aq)} + OH^{-}_{(aq)}} \\
\cee{Cl_{2} + 2OH^{-}  & <=> Cl^{-} + O-Cl + H_{2}O_{l} \label{r:ecbleach} } \\
\cee{Cl_{2} + 2NaOH & -> NaCl + NaClO + H_{2}O \label{r:hypochlorate}}\\
\cee{NaCl + 3H_{2}O & <=> NaClO_3 + 3H_2 \label{r:chlorate}} \\
\cee{ H_{2}O_{2} + NaClO & -> NaCl + H_{2}O + O_{2} \label{r:bleachperoxide}}\\
\end{align}
\caption{ \label{fig:reacts} test to see how chemfig works.  wtf is outer par mode?  Some reactions thought to be relevant to generation, decay, and
efficacy of electrolyzed water solutions. }
\end{figure}

\end{comment}

Dexter, who may be copper sensitive, has a varying prefernce
for turkey and this was eliminated from Happy's diet for a while
on the concern it reduces copper uptake.



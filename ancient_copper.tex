
\mjmtol{ always introduce new topic at the end lol }

Given copper's lustre and malleability it may not be surprising
that there is a long history of special treatment for copper in
past civilizations. The point of raising this experience
towards the end of this work is to put the apparent risks
and benefits into perspective. That is, it sounds like
only a very sophisticated Western audience could encounter copper
and actually benefit but that may not be the case either due to
luck, hidden knowledge, or a  more favorable risk benefit profile
from copper exposure. These "macro" observations may help put the
"micro" observations into a better range of implications. "Micro"
results may seem more precise and scientific but are by design
isolated and miss a lot of important context.

It sounds as if copper may require substantial care to supplement
but as a final thought its worth remembering that many older civilizations
appeared to recognize copper as particularly "good" in some ways
from cookware
\cite{Bansal_Zaina_Parihar_REVIEW_HEALTH_IMPACT_2020},
ritual uses, to medicinal\cite{Sugunthan_Shailaja_Parthasarathy_review_usage_2022}. Many appeared to persist for quite some
time but it would be interesting to see if they had signs of copper
toxicity or generally better health than those ignoring copper.

Copper is mentioned in some of the earliest known papyrus,
we used by many historical "great" civilizations such as Egyptions, Greeks,
Romans, and Aztecs with rediscovery in the 19th century to
be replcaed in medicine by antiiovtics in 1932
\cite{Grass_Rensing_Solioz_Metallic_Copper_2011}.
Even today, copper sulfate is burned in Buddhist and Hindu buildings for
good luck or religious reasons
\cite{Gamakaranage_Rodrigo_Weerasinghe_Complications_management_2011}.
If exposure contributed to the downfall of any of these
societies it is generally not explicit. 
There is some interest in revival of ancient 
antibiotics such as described in the  Leechbook
\cite{DiazGuerrero_CastilloJuarez_Zurabian_Reviving_Past_2024}
  previously mentioned but we don't know how effective these
are in real world conditions.
Copper combinations also arise in the study of ancient medicinal
plants
\cite{Blakeney_TeixidorToneu_Kjesrud_Finding_Medieval_Medicine_Through_2025}
.
A review of scholarly literature may be biased with conemporary
expectations. 
Health effects are difficutl to infer  but would have to be 
compared to competing exposures and impurities such as 
arsenic that may have been common in a given community.
Apparently arsenic was present in the area of Peruvian Andes
\cite{Lechtman_Klein_Production_Copper_1999}.
Given the adoption of modern antibiotics, they likely had clear
superiority to known copper formulations. 


Its also worth noting the intake of various acid foods in other
cultures and how this may relate to copper and other nutrient uptake.
Many health fads such as the Mediterranean diet contain acidici components
such as vinegar but these are rarely highlighted probably because of
confirmation bias issues, see the rcurring fallacy apendix.
Fermented foods are also common. 
Fermented foods are defined by the FDA as achieving a pH of less than 
4.6 
\cite{Caffrey_Perelman_Ward_Unpacking_Food_Fermentation_Clinically_2025}
 clearly the acid content is considered important.
Currently however there is the populat notion of a 
 "dietary acid load" of unknown significance tends to make people think they should avoid acid intake although the actual issue is thought to be renal
acid load making intake of organic acids just fine  \cite{PMC11006742}.
Despite vinegar having likely  benefits for glucose metabolism
\cite{Santos_deMoraes_daSilva_Vinegar_acetic_acid_intake_2019}
it is almost never mentioned in the context of the Mediterranean diet.





% https://tex.stackexchange.com/questions/231551/using-ifdefined-on-csname-macros
 \ifdefined\mjmstandalone%
%
\else%
\newcommand{\mjmstandalone}[1]{%
  \ifdefined\MJMTEXFRAG%
% \ifcsname\MJMTEXFRAG\endcsname%
   #1%
  \fi%
}
  \fi%


\begin{table}[H] \centering
%\begin{tabular}{r|r|c|r} \hline \multicolumn{4}{c}{Title}\\ \hline \end{tabular}
\begin{enumerate}
\item{ Meal sizes and food ingredient  quantities are missing or arbitrary. }
\item{ Not all meals are routinely recorded due to initial logistic issues with multiple people feeding the dogs and later a desire to maintain consistency.  }
\item{ Some preparation habits may not have been described completely. Metals were generally added last as mentioned in the text but other things like keeping the supplements dry and nominally unexposed to light could be important. }
\item{ Some items combined with isolated supplements for palitability and acceptance were not included.  }
\item{  Initially MUQED unit conversions were not used and weights were entered even as in almost all cases bulk supplements were measured by volume. Some were eventually converted to actual volume or numbers available to the "chef" while others continued to use a measured density.   }  
\item{ The "MgCitrate" entries reflect actual amount of the supplement not elemental magnesium as opposed to most other metal supplements. . This was retained but an issue early on making decisions about MUQED conventions. } 
\end{enumerate}
\caption{Known caveats with the MUQED diet data. Most of these limitations should not effect interpretations given here but due to interactions and complexity of copper all details may matter. Many of these issues can be fixed with better planning and the use of a more complete MUQED system. }
\label{tab:caveats}
\end{table}

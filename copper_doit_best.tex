Its expected after all of this that benefit or harm from
copper can depend on details of the diet, preparation, storage, and
supplement. Side reactions have to be considered at first contact
between copper and anything else and in general minimizing time
and temperature may be beneficial.  Time-release or limited
solubility forms may be useful for pre-mixing or copper could be
given in isolation from any other food intake.

Common supplement forms include copper sulfate, copper gluconate,
copper glycinate, and copper citrate.  Glycinate appears to be popular
due to solubility and reputed bioavailablity. Sulate appears to be
common due to cost \cite{Wu_Tan_Shi_vitro_bioaccessibility_2024}. 
Gluconate was used in a lot of the latter
part of this work as it was the only bulk form available in 
reasonable price and quantity. A priori and currently there is some
reason to consider citrate as desirable. This was eventually
purchased in kilogram quantities from China and had been used
early on as it is a common form in small supplement capsules.

Sulfate, glycinate, and proteinate effects were compared
in pigs and sulfate was found to be less absorbed and associated
with reduced microbial diversity \cite{PMC8718720}. 


Citrate salts have been well studied in a variety of contexts.
Calcium citrate in particular  has been explored as
it may have benefits for calcium handling in 
biological settings 
\cite{Liu_Skibsted_Citrate_calcium_transport_2023}.
It appears to become more soluble
with excess citrate although the Ca activity is reduced
\cite{Vavrusova_Skibsted_Aqueous_solubility_calcium_2016}.
As most of the snacks formulated here included significant
citric acid, the 'citrate dissolves citrate" notion may be
applicable although the effects of complexing are not 
known. Reduced reactivity may be a benefit however.

Citric acid appeared to improve biological activity
of copper as an algeacide in the presence of bicarbonate
even when derived from copper sulfate
\cite{Swader_Chan_Citric_acid_enhancement_1975}.

Studies have also shown that mixing copper citrate
with amino acids improves solubility
\cite{Sobel_Haigney_Kim_complexation_aqueous_}
which may be relevant here as most snacks contained
many amino acid supplements. 


Generally copper citrate is probably a good choice but in any
case avoiding high temperatures and storage times may be a good
idea. 
